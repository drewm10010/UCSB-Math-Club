\documentclass{article}
\usepackage[utf8]{inputenc}
\usepackage[utf8]{inputenc}
\usepackage{hyperref}
\usepackage{pdfrender,xcolor}
\usepackage{graphicx}
\usepackage{mathrsfs}
\usepackage{eqnarray}
\usepackage{xfrac}
\usepackage{dsfont}



\usepackage{amsmath,amsthm,amssymb}
\usepackage{mathtools,graphicx,tikz-cd}
\usepackage{blindtext}
\usepackage[margin = 1 in]{geometry}
\usepackage{enumitem}

\usepackage{fancyhdr,accents,lastpage}
\pagestyle{fancy}
\setlength{\headheight}{25pt}

\newtheorem{theorem}{Theorem}[section]
\newtheorem{corollary}{Corollary}[theorem]
\newtheorem{lemma}[theorem]{Lemma}

\theoremstyle{definition}
\newtheorem{definition}{Definition}[section]

\theoremstyle{remark}
\newtheorem*{remark}{Remark}
\newtheorem{exercise}{Exercise}
\newtheorem{example}{Example}


\DeclareMathOperator{\ab}{ab}
\DeclareMathOperator{\area}{area}
\DeclareMathOperator{\Aut}{Aut}
\DeclareMathOperator{\BGL}{BGL}
\DeclareMathOperator{\Br}{Br}
\DeclareMathOperator{\card}{card}
\DeclareMathOperator{\ch}{ch}
\DeclareMathOperator{\Char}{char}
\DeclareMathOperator{\CHur}{CHur}
\DeclareMathOperator{\Cl}{Cl}
\DeclareMathOperator{\coker}{coker}
\DeclareMathOperator{\Conf}{Conf}
\DeclareMathOperator{\disc}{disc}
\DeclareMathOperator{\End}{End}
\DeclareMathOperator{\et}{\text{\'et}}
\DeclareMathOperator{\Fix}{Fix}
\DeclareMathOperator{\Gal}{Gal}
\DeclareMathOperator{\GL}{GL}
\DeclareMathOperator{\Hom}{Hom}
\DeclareMathOperator{\Hur}{Hur}
\DeclareMathOperator{\im}{im}
\DeclareMathOperator{\Ind}{Ind}
\DeclareMathOperator{\Inn}{Inn}
\DeclareMathOperator{\Irr}{Irr}
\DeclareMathOperator{\lcm}{lcm}
\DeclareMathOperator{\Mor}{Mor}
\DeclareMathOperator{\ord}{ord}
\DeclareMathOperator{\Out}{Out}
\DeclareMathOperator{\Perm}{Perm}
\DeclareMathOperator{\PGL}{PGL}
\DeclareMathOperator{\Pin}{Pin}
\DeclareMathOperator{\PSL}{PSL}
\DeclareMathOperator{\rad}{rad}
\DeclareMathOperator{\sgn}{sgn}
\DeclareMathOperator{\SL}{SL}
\DeclareMathOperator{\SO}{SO}
\DeclareMathOperator{\Sp}{Sp}
\DeclareMathOperator{\Spec}{Spec}
\DeclareMathOperator{\Spin}{Spin}
\DeclareMathOperator{\St}{St}
\DeclareMathOperator{\Surj}{Surj}
\DeclareMathOperator{\Syl}{Syl}
\DeclareMathOperator{\tame}{tame}
\DeclareMathOperator{\Tr}{Tr}

\newcommand{\eps}{\varepsilon}
\newcommand{\QED}{\hspace{\stretch{1}} $\blacksquare$}
\renewcommand{\AA}{\mathbb{A}}
\newcommand{\CC}{\mathbb{C}}
\newcommand{\EE}{\mathbb{E}}
\newcommand{\FF}{\mathbb{F}}
\newcommand{\HH}{\mathbb{H}}
\newcommand{\NN}{\mathbb{N}}
\newcommand{\OO}{\mathbb{O}}
\newcommand{\PP}{\mathbb{P}}
\newcommand{\QQ}{\mathbb{Q}}
\newcommand{\RR}{\mathbb{R}}
\newcommand{\ZZ}{\mathbb{Z}}
\newcommand{\bfm}{\mathbf{m}}
\newcommand{\mcA}{\mathcal{A}}
\newcommand{\mcC}{\mathcal{C}}
\newcommand{\mcG}{\mathcal{G}}
\newcommand{\mcH}{\mathcal{H}}
\newcommand{\mcM}{\mathcal{M}}
\newcommand{\mcN}{\mathcal{N}}
\newcommand{\mcO}{\mathcal{O}}
\newcommand{\mcP}{\mathcal{P}}
\newcommand{\mcQ}{\mathcal{Q}}
\newcommand{\mfa}{\mathfrak{a}}
\newcommand{\mfb}{\mathfrak{b}}
\newcommand{\mfI}{\mathfrak{I}}
\newcommand{\mfM}{\mathfrak{M}}
\newcommand{\mfm}{\mathfrak{m}}
\newcommand{\mfo}{\mathfrak{o}}
\newcommand{\mfO}{\mathfrak{O}}
\newcommand{\mfP}{\mathfrak{P}}
\newcommand{\mfp}{\mathfrak{p}}
\newcommand{\mfq}{\mathfrak{q}}
\newcommand{\mfz}{\mathfrak{z}}
\newcommand{\AGL}{\mathbb{A}\GL}
\newcommand{\Qbar}{\overline{\QQ}}
\renewcommand{\qedsymbol}{$\blacksquare$}
\newcommand{\geqs}{\geqslant}
\newcommand{\leqs}{\leqslant}


\pdfrender{StrokeColor=black,TextRenderingMode=2,LineWidth=0.4pt}

\lhead{\pdfrender{StrokeColor=black,TextRenderingMode=2,LineWidth=0.5pt}Constructing Spaces}   
\chead{\pdfrender{StrokeColor=black,TextRenderingMode=2,LineWidth=0.5pt}Math Club}
\rhead{\pdfrender{StrokeColor=black,TextRenderingMode=2,LineWidth=0.5pt}Spring 2021} 


\begin{document}
\section{Introduction}
How to construct new spaces from the old ones has been a fundamental issue in topology and metric geometry. A few simple and classical examples are: 
\begin{enumerate}
\item The direct product of $\mathbb{R}^{n}$ and $\mathbb{R}^{m}$ is $\mathbb{R}^{n + m}$
\item The quotient of $[0, 1]$ by the relation $0 \sim 1$ gives $S^{1}$
\item The quotient of $S^{2}$ by the relation $x \sim -x$ gives $\mathbb{R}P^{2}$
\item The cone over a line segment is a plane sector
\item The suspension of $S^{n}$ becomes $S^{n + 1}$ 
\end{enumerate}
Today, we delve into operations on topological spaces with a focus on the change of their CW structures.  

\section{Resources}
Some books that have further reading on these topics are
\begin{itemize}
    \item \emph{Introduction to Topology} by Theodore W. Gamelin and Robert Everist Greene 
    \item \emph{Algebraic Topology} by Allen Hatcher 
    \item \emph{Differential Topology} by Victor Guillemin and Alan Pollack 
    \item \emph{A Course in Metric Geometry} by Dmitri Burago, Yuri Burago, and Sergei Ivanov 
    \item \emph{Metric Spaces of Non-positive Curvature} by Martin R. Bridson and Andr\'{e} Haefliger 
\end{itemize}

\section{Basic Definitions} 
\begin{definition}
A metric on a set $X$ is a real-valued function $d$ on $X \times X$ that has the following properties: 
\begin{eqnarray}
d(x, y) \geq 0, x, y \in X,\\
d(x, y) = 0 \text{ if and only if $x = y$,}\\
d(x, y) = d(y, x), x, y \in X, \\
d(x, z) \leq d(x, y) + d(y, z), x, y, z \in X. 
\end{eqnarray}
\end{definition}

\begin{definition}
A metric space $(X, d)$ is a set $X$ equipped with a metric $d$ on $X$. 
\end{definition}

\begin{definition}
Let $X$ be a set. A family $\mathscr{T}$ of subsets of $X$ is a topology for $X$ if $\mathscr{T}$ has the following three properties: 
\begin{enumerate}
    \item Both $X$ and the empty set belong to $\mathscr{T}$.
    \item Any union of sets in $\mathscr{T}$ belongs to $\mathscr{T}$.
    \item Any finite intersection of sets in $\mathscr{T}$ belongs to $\mathscr{T}$. 
\end{enumerate}
\end{definition}

\begin{definition}
A topological space is a pair $(X, \mathscr{T})$, where $X$ is a set and $\mathscr{T}$ is a topology for $X$. The sets in $\mathscr{T}$ are called open sets.  
\end{definition}

\begin{definition}
Let $\pi: X \rightarrow Y = \sfrac{X}{\sim}$ be the projection map sending $x$ to the equivalence class $[x]$. The quotient topology of $Y$ is given by $\mathscr{T} = \{U \subseteq Y \:|\: \pi^{-1}(U) \text{ is an open subset of $X$}\}$.
\end{definition}

\begin{definition}
A space $X$ constructed in the following way is called a cell complex or CW complex:
\begin{enumerate}
    \item Start with a discrete set $X_{0}$, whose points are regarded as $0$-cells.
    \item Inductively, form the $n$-skeleton $X^{n}$ from $X^{n - 1}$ by attaching $n$-cells $e_{\alpha}^{n}$ via maps $\varphi_{\alpha}: S^{n - 1} \rightarrow X^{n - 1}$. This means that $X^{n}$ is the quotient space of the disjoint union $X^{n - 1} \sqcup_{\alpha} D_{\alpha}^{n}$ of $X^{n - 1}$ with a collection of $n$-disks $D_{\alpha^{n}}$ under the identifications $x \sim \varphi_{\alpha}(x)$ for $x \in \partial D_{\alpha}^{n}$. Thus as a set, $X^{n} = X^{n - 1} \sqcup_{\alpha} e_{\alpha}^{n}$ where each $e_{\alpha}^{n}$ is an open $n$-disk.  
    \item One can either stop this inductive process at a finite step, setting $X = X^{n}$ for some $n < \infty$, or one can continue indefinitely, setting $X = \cup_{n} X^{n}$. In the latter case $X$ is given the weak topology: A set $A \subset X$ is open iff $A \cap X^{n}$ is open in $X^{n}$ for each $n$. 
\end{enumerate}
\end{definition}

\begin{remark}
The similar idea leads to the construction of simplicial complex and cubicial complex.  
\end{remark}

\section{Problems}
\begin{exercise}
The Euler characteristic, which for a cell complex with finitely many cells, is defined to be the number of even-dimensional cells minus the number of odd-dimensional cells. Compute the Euler characteristics for $S^{1}$, $S^{2}, T^{2}$, and the surface of genus two. 
\end{exercise}

\begin{exercise}(The line with two origins)
Let $X$ be the set of all points $(x, y) \in \mathbb{R}^{2}$ such that $y = \pm{1}$, and let $M$ be the quotient of $X$ by the equivalence relation generated by $(x, -1) \sim (x, 1)$ for all $x \neq 0$. Show that $M$ is locally Euclidean and second-countable, but not Hausdorff. 
\end{exercise}

\begin{exercise}
What are cell structures for $\mathbb{R}P^{n}$ and $\mathbb{C}P^{n}$? 
\end{exercise}

\begin{exercise}
Given positive integers $v$, $e$, and $f$ satisfying $v - e + f = 2$, construct a cell structure on $S^{2}$ having $v$ $0$-cells, $e$ $1$-cells, and $f$ $2$-cells. 
\end{exercise}

\begin{exercise}
Show that $S^{1} * S^{1} = S^{3}$, and more generally, $S^{m} * S^{n} = S^{m + n + 1}$. 
\end{exercise}
\end{document}