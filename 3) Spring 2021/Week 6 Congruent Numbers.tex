\documentclass{article}
\usepackage[utf8]{inputenc}
\usepackage[utf8]{inputenc}
\usepackage{hyperref}
\usepackage{pdfrender,xcolor}
\usepackage{graphicx}
\usepackage{mathrsfs}
\usepackage{eqnarray}

\usepackage{amsmath,amsthm,amssymb}
\usepackage{mathtools,graphicx,tikz-cd}
\usepackage{blindtext}
\usepackage[margin = 1 in]{geometry}
\usepackage{enumitem}

\usepackage{fancyhdr,accents,lastpage}
\pagestyle{fancy}
\setlength{\headheight}{25pt}

\newtheorem{theorem}{Theorem}%[section]
\newtheorem{corollary}{Corollary}[theorem]
\newtheorem{lemma}[theorem]{Lemma}

\theoremstyle{definition}
\newtheorem{definition}{Definition}%[section]

\theoremstyle{remark}
\newtheorem*{remark}{Remark}
\newtheorem{exercise}{Exercise}
\newtheorem{example}{Example}


\DeclareMathOperator{\ab}{ab}
\DeclareMathOperator{\area}{area}
\DeclareMathOperator{\Aut}{Aut}
\DeclareMathOperator{\BGL}{BGL}
\DeclareMathOperator{\Br}{Br}
\DeclareMathOperator{\card}{card}
\DeclareMathOperator{\ch}{ch}
\DeclareMathOperator{\Char}{char}
\DeclareMathOperator{\CHur}{CHur}
\DeclareMathOperator{\Cl}{Cl}
\DeclareMathOperator{\coker}{coker}
\DeclareMathOperator{\Conf}{Conf}
\DeclareMathOperator{\disc}{disc}
\DeclareMathOperator{\End}{End}
\DeclareMathOperator{\et}{\text{\'et}}
\DeclareMathOperator{\Fix}{Fix}
\DeclareMathOperator{\Gal}{Gal}
\DeclareMathOperator{\GL}{GL}
\DeclareMathOperator{\Hom}{Hom}
\DeclareMathOperator{\Hur}{Hur}
\DeclareMathOperator{\im}{im}
\DeclareMathOperator{\Ind}{Ind}
\DeclareMathOperator{\Inn}{Inn}
\DeclareMathOperator{\Irr}{Irr}
\DeclareMathOperator{\lcm}{lcm}
\DeclareMathOperator{\Mor}{Mor}
\DeclareMathOperator{\ord}{ord}
\DeclareMathOperator{\Out}{Out}
\DeclareMathOperator{\Perm}{Perm}
\DeclareMathOperator{\PGL}{PGL}
\DeclareMathOperator{\Pin}{Pin}
\DeclareMathOperator{\PSL}{PSL}
\DeclareMathOperator{\rad}{rad}
\DeclareMathOperator{\sgn}{sgn}
\DeclareMathOperator{\SL}{SL}
\DeclareMathOperator{\SO}{SO}
\DeclareMathOperator{\Sp}{Sp}
\DeclareMathOperator{\Spec}{Spec}
\DeclareMathOperator{\Spin}{Spin}
\DeclareMathOperator{\St}{St}
\DeclareMathOperator{\Surj}{Surj}
\DeclareMathOperator{\Syl}{Syl}
\DeclareMathOperator{\tame}{tame}
\DeclareMathOperator{\Tr}{Tr}

\newcommand{\eps}{\varepsilon}
\newcommand{\QED}{\hspace{\stretch{1}} $\blacksquare$}
\renewcommand{\AA}{\mathbb{A}}
\newcommand{\CC}{\mathbb{C}}
\newcommand{\EE}{\mathbb{E}}
\newcommand{\FF}{\mathbb{F}}
\newcommand{\HH}{\mathbb{H}}
\newcommand{\NN}{\mathbb{N}}
\newcommand{\OO}{\mathbb{O}}
\newcommand{\PP}{\mathbb{P}}
\newcommand{\QQ}{\mathbb{Q}}
\newcommand{\RR}{\mathbb{R}}
\newcommand{\ZZ}{\mathbb{Z}}
\newcommand{\bfm}{\mathbf{m}}
\newcommand{\mcA}{\mathcal{A}}
\newcommand{\mcC}{\mathcal{C}}
\newcommand{\mcG}{\mathcal{G}}
\newcommand{\mcH}{\mathcal{H}}
\newcommand{\mcM}{\mathcal{M}}
\newcommand{\mcN}{\mathcal{N}}
\newcommand{\mcO}{\mathcal{O}}
\newcommand{\mcP}{\mathcal{P}}
\newcommand{\mcQ}{\mathcal{Q}}
\newcommand{\mcT}{\mathcal{T}}
\newcommand{\mfa}{\mathfrak{a}}
\newcommand{\mfb}{\mathfrak{b}}
\newcommand{\mfI}{\mathfrak{I}}
\newcommand{\mfM}{\mathfrak{M}}
\newcommand{\mfm}{\mathfrak{m}}
\newcommand{\mfo}{\mathfrak{o}}
\newcommand{\mfO}{\mathfrak{O}}
\newcommand{\mfP}{\mathfrak{P}}
\newcommand{\mfp}{\mathfrak{p}}
\newcommand{\mfq}{\mathfrak{q}}
\newcommand{\mfz}{\mathfrak{z}}
\newcommand{\AGL}{\mathbb{A}\GL}
\newcommand{\Qbar}{\overline{\QQ}}
\renewcommand{\qedsymbol}{$\blacksquare$}
\newcommand{\geqs}{\geqslant}
\newcommand{\leqs}{\leqslant}


%\pdfrender{StrokeColor=black,TextRenderingMode=2,LineWidth=0.4pt}

%\lhead{\pdfrender{StrokeColor=black,TextRenderingMode=2,LineWidth=0.5pt}Congruent Number Problem}   
%\chead{\pdfrender{StrokeColor=black,TextRenderingMode=2,LineWidth=0.5pt}Math Club}
%\rhead{\pdfrender{StrokeColor=black,TextRenderingMode=2,LineWidth=0.5pt}Spring 2021} 
\lhead{Congruent Number Problem}
\chead{Math Club}
\rhead{Spring 2021}

\begin{document}

Welcome to week 6 of Math Club!
Today we will discuss the Pythagorean theorem and an ``open'' problem in the area of number theory known as Diophantine geometry, the congruent number problem.
For each exercise, you may use anything stated prior to that exercise.
Exercises marked with an asterisk are more difficult.

\section{Pythagorean Triples}

We are all familiar with the famous theorem of Pythagoras: if we have a right triangle with side lengths \(a,b,c\), with \(c\) the hypotenuse, then \(a,b,c\) must satisfy \(a^2+b^2=c^2\).

\begin{definition}
	A \textbf{Pythagorean triple} consists of three positive integers \(a,b,c\) that satisfy \(a^2+b^2=c^2\).
	Some examples include \((3,4,5)\) and \((5,12,13)\).
\end{definition}

\begin{exercise}
	Show that there are infinitely many Pythagorean triples.
\end{exercise}

\begin{definition}
	A \textbf{primitive Pythagorean triple} is one in which \(a,b,c\) are coprime.
	That is, \(a,b,c\) share no common factors.
\end{definition}

If \((a,b,c)\) is a primitive Pythagorean triple, then the point \((a/c,b/c)\) is a rational point on the unit circle.

\begin{theorem}
	If \((a,b,c)\) is a primitive Pythagorean triple, then one of \(a\) or \(b\) is even and the other is odd.
	Taking \(b\) to be even, 
	\[a=k^2-\ell^2,\quad b=2k\ell,\quad c=k^2+\ell^2\]
	for coprime integers \(k>\ell>0\) with different parity.
	Conversely, for such integers \(k,\ell\), the above formulas yield a primitive Pythagorean triple.
\end{theorem}

See \url{https://kconrad.math.uconn.edu/blurbs/ugradnumthy/pythagtriple.pdf} for a more in depth discussion of this result, its derivation, and more on Pythagorean triples.

\begin{exercise}
	Find all primitive Pythagorean triples with \(a\) odd and \(c=a+2\).
	Explicitly calculate one such triple with \(c>1000\).
\end{exercise}

\section{The Congruent Number Problem}

\begin{definition}
    A right triangle is called \textbf{rational} if all its side lengths are rational.
    For example, the triangle with sides \((3,4,5)\) is rational and the triangle with sides \((3/2,20/3,41/6)\) is rational.
    You can verify that the Pythagorean theorem holds.
\end{definition}

\begin{definition}
    A positive rational number \(n\) is called a \textbf{congruent number} if there exists a rational right triangle with area \(n\).
    Equivalently, if there are rational \(a,b,c\) such that 
   \[a^2+b^2=c^2\quad\text{and}\quad \frac{1}{2}ab=n.\]
\end{definition}

The congruent number problem asks: ``which rational numbers \(n\) are congruent numbers?''
You can check that the examples of rational triangles we gave above are triangles of area 6 and 5, respectively, so these are both congruent numbers.

\begin{exercise}
    If \(n\) is a congruent number, prove that \(k^2n\) is also a congruent number for any \(k\in\QQ\).
\end{exercise}

Thus it suffices to only consider squarefree integer, integers that are not divisible by the square of any integers, in our discussion of congruent numbers. Why?

\begin{exercise}*
    Show that 1 is not a congruent number.
    {\it Hint: Argue by contradiction and apply infinite descent.}
\end{exercise}

\begin{exercise}
    Show that \(\sqrt{2}\) is irrational.
	There are many ways to do this, but in the spirit of today's topic we suggest that you use the previous exercise.
\end{exercise}

\begin{exercise}
    Let \(q\in\QQ\).
    Show that \(q^2\) is not a congruent number.
\end{exercise}

There is an important theorem that relates arithmetic progressions of squares to congruent numbers:

\begin{theorem}
	Let \(n\in\ZZ_{>0}\).
	There is a one-to-one correspondence between right triangles with area \(n\) and 3-term arithmetic progressions of squares with common difference \(n\): the sets
	\[\{(a,b,c):a^2+b^2=c^2,\ ab/2=n\}\quad\text{and}\quad\{(r,s,t):s^2-r^2=n,\ t^2-s^2=n\}\]
	are in one-to-one correspondence by
	\[(a,b,c)\longmapsto((b-a)/2,c/2,(b+a)/2)\quad\text{and}\quad(r,s,t)\longmapsto(t-r,t+r,2s).\]
\end{theorem}

\begin{exercise}
	Prove Theorem 2.
\end{exercise}

Congruent numbers are also closely related to the solvability of the equation \(y^2=x^3-n^2x\) in the integers (given an integer \(n\)).

\begin{theorem}
	Let \(n\in\ZZ_{>0}\).
	There is a one-to-one correspondence between the following two sets:
	\[\{(a,b,c):a^2+b^2=c^2,\ ab/2=n\}\quad\text{and}\quad\{(x,y):y^2=x^3-n^2x,\ y\neq 0\}.\]
	The correspondence is given by 
	\[(a,b,c)\longmapsto\left(\frac{nb}{c-a},\frac{2n^2}{c-a}\right)\quad\text{and}\quad(x,y)\mapsto\left(\frac{x^2-n^2}{y},\frac{2nx}{y},\frac{x^2+n^2}{y}\right).\]
\end{theorem}

\begin{exercise}
	Prove Theorem 3.
\end{exercise}

\begin{exercise}
	Find a non-trivial solution to the equation \(y^2=x(x^2-36)\).
	That is, a solution where \(y\neq 0\).
\end{exercise}

\begin{exercise}
	Prove that \(7\) is a congruent number.
\end{exercise}

\begin{theorem}[Tunnell's Theorem]
	Let \(n\) be a squarefree integer.
	Set
	\begin{align*}
		f(n) &= \#\{(x,y,z)\in\ZZ^3:x^2+2y^2+8z^2=n\}\\
		g(n) &= \#\{(x,y,z)\in\ZZ^3:x^2+2y^2+32z^2=n\}\\
		h(n) &= \#\{(x,y,z)\in\ZZ^3:x^2+2y^2+8z^2=n/2\}\\
		k(n) &= \#\{(x,y,z)\in\ZZ^3:x^2+2y^2+32z^2=n/2\}
	\end{align*}
	For odd \(n\), if \(n\) is congruent, then \(f(n)=2g(n)\).
	For even \(n\), if \(n\) is congruent, then \(h(n)=2k(n)\).
	Moreover, if the weak Birch and Swinnerton-Dyer conjecture is true for the curve \(y^2=x^3-n^2x\), then the reverse implication is also true.
\end{theorem}

The Birch and Swinnerton-Dyer (BSD) conjecture is one of the Millennium Prize problems, one of the most important conjectures in math.

\begin{exercise}
	Using Tunnell's Theorem, show that the following numbers are not congruent: 1, 2, 9, 10.
\end{exercise}

\end{document}