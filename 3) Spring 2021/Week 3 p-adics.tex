\documentclass[10pt]{beamer}

\usetheme[progressbar=frametitle]{metropolis}
\usepackage{appendixnumberbeamer}
\usepackage{booktabs}
\usepackage[scale=2]{ccicons}
\usepackage{xspace}
\usepackage{physics}
\usepackage{amsmath}

\title{A Hands-on Exploration of \(\mathbb{Q}_p\)}
\subtitle{Math Club, Spring 2021}
\date{Week 3}
\author{}
\institute{University of California, Santa Barbara}

\begin{document}

\maketitle

\begin{frame}
    \frametitle{What are numbers anyway?}

    Some examples of numbers:
    \begin{itemize}
        \item \(3.1415926...=\pi\)
        \item \(0.3333333...=1/3\)
        \item \(1.4142135=\sqrt{2}\)
    \end{itemize}
    This notation is shorthand for a sum:
    \[\sqrt{2}=1\cdot\left(\frac{1}{10}\right)^0 + 4\cdot\left(\frac{1}{10}\right)^1 + 1\cdot\left(\frac{1}{10}\right)^2 + 4\cdot\left(\frac{1}{10}\right)^3 + \cdots\]

\end{frame}

\begin{frame}
    \frametitle{What are numbers anyway?}

    \begin{center}
        \Huge Is this acceptable? Why?
    \end{center}

\end{frame}

\begin{frame}
    \frametitle{Zeno's Paradox}

    Suppose Ancient Greek track legend Atalanta has to run a mile.
    First she runs half a mile.
    Then she must run an additional quarter mile.
    Then an eighth.
    Then a sixteenth.
    And so on.

\end{frame}

\begin{frame}
    \frametitle{The Math Involved}

    We know that Atalanta reaches her goal of a mile run.
    We can represent this as the series
    \[\frac{1}{2}\sum_{i=0}^\infty\left(\frac{1}{2}\right)^i,\]
    which we evaluate with the standard formula
    \[\sum_{i=1}^\infty r^i=\frac{1}{1-r},\quad \abs{r}<p\]

\end{frame}

\begin{frame}
    \frametitle{Zeno's \(p\)-aradox}

    Let \(p\) be a prime number and consider the series
    \[1+p+p^2+p^3+\cdots.\]
    Suppose \(p\)-Ancient Greek track legend \(p\)-Atalanta has to run a race, starting at 0, running first to the point 4, then 20 miles to the point \(4+4\cdot 5^1\), then 100 miles to the point \(4+4\cdot 5^1+4\cdot 5^2\), etc, culminating in the series
    \[4\cdot 5^0 + 4\cdot 5^1 + 4\cdot 5^2 + 4\cdot 5^3 + 4\cdot 5^4 + 4\cdot 5^5 + \cdots.\]
    This is divergent.

    Right?

\end{frame}

\begin{frame}
    \frametitle{Basic Recap of Number Theory}

    \begin{definition}
        For two integers \(a,b\), we say that \(a\) {\bf divides} \(b\) if there exists \(n\in\mathbb{Z}\) such that \(b=na\).
        In this case, we write \(a\mid b\).
        Examples: \(2\mid -6\), \(25\mid 625\), \(k\mid 0\) for all \(k\in\mathbb{Z}\), \(k\neq 0\).
    \end{definition}

    \begin{definition}
        For two integers \(a,b\) and \(n\in\mathbb{N}\), we say that \(a\) is {\bf congruent} to \(b\) modulo \(n\), and write \(a\equiv b\pmod{n}\) if \(n\mid(b-a)\).
        Examples: \(4\equiv -1\pmod{5}\), \(3\equiv 10\pmod{7}\), \(10\equiv 108\pmod 7^2\).
    \end{definition}

\end{frame}

\begin{frame}
    \frametitle{A New Perspective}

    Fix a prime \(p\).
    We know that two rational numbers \(a,b\) are equal iff \(p^k\) divides the numerator of the reduced form of \(a-b\) for all positive integers \(k\) (why?).

    Let's relax this and say that \(a\) and \(b\) are ``\(p\)-adically close'' if \(p^k\) divides the numerator of the reduced form of \(a-b\) for ``many'' \(k\).

    In particular, a number is small \(p\)-adically if it is highly divisible by \(p\).
    For example, if \(p=5\), \(625\) is much smaller than \(10\) when considered \(5\)-adically.

\end{frame}

\begin{frame}
    \frametitle{Back to \(p\)-Atalanta}

    With this new perspective, \(p\)-Atalanta's steps are getting smaller! (\(p\)-adically)
    \[4\cdot 5^0 + 4\cdot 5^1 + 4\cdot 5^2 + 4\cdot 5^3 + 4\cdot 5^4 + 4\cdot 5^5 + \cdots\]

\end{frame}

\begin{frame}
    \frametitle{\(p\)-adic Numbers}

    \begin{definition}
        We define the set of \(p\)-adic numbers to be the set
        \[\mathbb{Q}_p = \left\{\sum_{i=n_0}^\infty b_ip^i:n_0\in\mathbb{Z}\text{ and } b_i\in\{0,1,2,\ldots,p-1\}\right\}.\]
        Similarly, define the \(p\)-adic integers to be the set
        \[\mathbb{Z}_p = \left\{\sum_{i=0}^\infty b_ip^i:b_i\in\{0,1,2,\ldots,p-1\}\right\}.\]
    \end{definition}

\end{frame}

\begin{frame}
    \frametitle{Exercise 1}

    Show that \(\mathbb{Q}_p\) is uncountable.

    Hint: one may follow Cantor's diagonalization argument.

\end{frame}

\begin{frame}
    \frametitle{Arithmetic}

    We now have a new system of numbers, so a natural thing to do is arithmetic with these new numbers.

    First we need to recall the notion of writing a number in base \(p\).
    That is, we write a natural number \(n\) in the form \(\sum b_ip^i\).
    Some examples when \(p=7\), 
    \[77 = 0\cdot 7^0 + 4\cdot 7^1 + 1\cdot 7^2 + 0\cdot 7^3 + \cdots,\]
    \[37 = 2\cdot 7^0 + 5\cdot 7^1 + 0\cdot 7^2 + 0\cdot 7^3 + \cdots,\]
    \[113 = 1\cdot 7^0 + 2\cdot 7^1 + 3\cdot 7^2 + 0\cdot 7^3 + \cdots.\]

\end{frame}

\begin{frame}
    \frametitle{Addition}

    Q: How do we add?
    
    A: Like we learned in elementary school!
    We add each \(p\)igit (\(p\)-digit) and carry over terms as necessary.
    For example:
    \begin{align*}
        &0\cdot 7^0 + 4\cdot 7^1 + 1\cdot 7^2 + 0\cdot 7^3 + \cdots = 77\\
        +\ &2\cdot 7^0 + 5\cdot 7^1 + 0\cdot 7^2 + 0\cdot 7^3 + \cdots = 37\\
        =\ &2\cdot 7^0 + 2\cdot 7^1 + 3\cdot 7^2 + 0\cdot 7^3 + \cdots = 114
    \end{align*}

\end{frame}

\begin{frame}
    \frametitle{\(p\)-Atalanta's run}

    Let's add 1 to \(p\)-Atalanta's journey:
    \begin{align*}
        &4\cdot 5^0 + 4\cdot 5^1 + 4\cdot 5^2 + 4\cdot 5^3 + 4\cdot 5^4 + \cdots\\
        +\ &1\cdot 5^0 + 0\cdot 5^1 + 0\cdot 5^2 + 0\cdot 5^3 + 0\cdot 5^4 + \cdots\\
        =\ &0\cdot 5^0 + 0\cdot 5^1 + 0\cdot 5^2 + 0\cdot 5^3 + 0\cdot 5^4 + \cdots\\
        =\ &0
    \end{align*}

    If we let 
    \[a=4\cdot 5^0 + 4\cdot 5^1 + 4\cdot 5^2 + 4\cdot 5^3 + 4\cdot 5^4 + \cdots,\]
    then \(a+1 = 0\), so it must be that \(a=-1\), the additive inverse of 1 in \(\mathbb{Q}_p\).

\end{frame}

\begin{frame}
    \frametitle{Multiplication}

    Multiplication in \(\mathbb{Q}_p\) works analogously to multiplication with regular decimals.
    We write our numbers as sums and multiply them out term-by-term.
    Take it as a fact that multiplication works as intended.

\end{frame}

\begin{frame}
    \frametitle{Exercise 2}

    Show that every \(p\)-adic number \(a\in\mathbb{Q}_p\) has an additive inverse.
    What are the \(p\)igits of this inverse \(-a\)?

    Hint: We know that multiplication works.
    How might we use this to construct an additive inverse?

\end{frame}

\begin{frame}
    \frametitle{Subtraction}

    Addition but negative.

\end{frame}

\begin{frame}
    \frametitle{Division}

    Division takes some more work.
    Let's find the \(5\)-adic expansion of \(1/3\) (if it even exists).
    This number is of the form \(b_0\cdot 5^0+b_1\cdot 5^1+b_2\cdot 5^2+\cdots\) that, when multiplied by 3, gives 1:
    \[3(b_0\cdot 5^0+b_1\cdot 5^1+b_2\cdot 5^2+\cdots) = 1\cdot 5^0 + 0\cdot 5^1 + 0\cdot 5^2 + \cdots\]
    So how do we do this?

    Long division!

\end{frame}

\begin{frame}
    \frametitle{Division}

    \[3(b_0\cdot 5^0+b_1\cdot 5^1+b_2\cdot 5^2+\cdots) = 1\cdot 5^0 + 0\cdot 5^1 + 0\cdot 5^2 + \cdots\]
    Solve \(3b_0\equiv 1\pmod{5}\) for \(b_0\in\{0,\ldots,4\}\). 
    Get \(b_0=2\).
    Then subtract \(3b_0\cdot 5^0=1\cdot 5^0 + 1\cdot 5^1\) from both sides of the equation:
    \[3(b_0\cdot 5^0+b_1\cdot 5^1+b_2\cdot 5^2+\cdots)-3b_0\cdot 5^0 = 0\cdot 5^0 - 1\cdot 5^1 + 0\cdot 5^2 + \cdots\]
    \[3(b_1\cdot 5^1+b_2\cdot 5^2+b_3\cdot 5^3+\cdots)=0\cdot 5^0+4\cdot 5^1+4\cdot 5^2+4\cdot 5^3+\cdots,\]
    where we used our knowledge of the value of \(-1\).

    We can continue this indefinitely to find the inverse of any nonzero \(p\)-adic number.

\end{frame}

\begin{frame}
    \frametitle{Algebra}

    \begin{center}
        \huge \(\mathbb{Q}_p\) is a field.        
    \end{center}

    \normalsize This roughly means that we can add, subtract, multiply, and divide (by nonzero numbers), all of which work as expected.

\end{frame}

\begin{frame}
    \frametitle{Exercise 3}

    \begin{itemize}
        \item Calculate \(\sqrt{2}\) in \(\mathbb{Q}_5\).
        \item Calculate \(\sqrt{-1}\) in \(\mathbb{Q}_5\).
        \item For which primes \(p\) does \(\mathbb{Q}_p\) have a square root of \(-1\)?
    \end{itemize}

    Hint: These will take some computation.
    Be slow and methodical when writing things out.
    The third question assumes some knowledge of number theory (quadratic residues, etc).

\end{frame}

\section{Why do we care?}

\begin{frame}
    \frametitle{Answer}

    \Huge Because they're fun!
    
    \normalsize Okay for real...

\end{frame}

\begin{frame}
    \frametitle{Local-Global Principle}

    One can study integer solutions to an equation by studying its behavior in \(\mathbb{R}\) and \(\mathbb{Q}_p\) for each prime \(p\).

    More formally, 

    \begin{theorem}[Hasse-Minkowski]
        If \(F(X_1,\ldots,X_n)\in\mathbb{Q}[X_1,\ldots,X_n]\) is a quadratic form, then 
        \[F(X_1,\ldots,X_n)=0\]
        has nontrivial solutions in \(\mathbb{Q}\) if and only if it has nontrivial solutions in \(\mathbb{Q}_p\) for all \(p\leq\infty\).
    \end{theorem}

\end{frame}

\begin{frame}
    \frametitle{Research}

    There is lots of exciting and current research out there that concerns \(p\)-adic numbers.
    In particular, Peter Scholze's work with \(p\)-adics won him the most recent Fields Medal back in 2018!

\end{frame}

\begin{frame}
    \frametitle{Credit Where Credit is Due}

    Most of this presentation was inspired by the lovely people at the Arizona Winter School.
    In particular, Renee Bell's and Charlotte Chan's lectures were the motivation behind this content.

\end{frame}

\begin{frame}
    \frametitle{Exercises}

    \begin{enumerate}
        \item Show that \(\mathbb{Q}_p\) is uncountable.
        \item Show that every \(p\)-adic number \(a\in\mathbb{Q}_p\) has an additive inverse.
        What are the \(p\)igits of this inverse \(-a\)?
        \item \begin{itemize}
            \item Calculate \(\sqrt{2}\) in \(\mathbb{Q}_5\).
            \item Calculate \(\sqrt{-1}\) in \(\mathbb{Q}_5\).
            \item For which primes \(p\) does \(\mathbb{Q}_p\) have a square root of \(-1\)?
        \end{itemize}
        \item Fix a prime \(p\) and let \(x,y\in\mathbb{Z}_p\).
        Prove that if \(x,y\neq 0\), then \(xy\neq 0\).
        \item Pick your favorite rational number \(a/b\) and calculate \(a/b\) in \(\mathbb{Q}_p\).
    \end{enumerate}

\end{frame}

\end{document}