\documentclass{article}
\usepackage[utf8]{inputenc}
\usepackage[utf8]{inputenc}
\usepackage{hyperref}
\usepackage{pdfrender,xcolor}
\usepackage{graphicx}
\usepackage{mathrsfs}
\usepackage{eqnarray}

\usepackage{amsmath,amsthm,amssymb}
\usepackage{mathtools,graphicx,tikz-cd}
\usepackage{blindtext}
\usepackage[margin = 1 in]{geometry}
\usepackage{enumitem}

\usepackage{fancyhdr,accents,lastpage}
\pagestyle{fancy}
\setlength{\headheight}{25pt}

\newtheorem{theorem}{Theorem}%[section]
\newtheorem{corollary}{Corollary}[theorem]
\newtheorem{lemma}[theorem]{Lemma}

\theoremstyle{definition}
\newtheorem{definition}{Definition}%[section]

\theoremstyle{remark}
\newtheorem*{remark}{Remark}
\newtheorem{exercise}{Exercise}
\newtheorem{example}{Example}


\DeclareMathOperator{\ab}{ab}
\DeclareMathOperator{\area}{area}
\DeclareMathOperator{\Aut}{Aut}
\DeclareMathOperator{\BGL}{BGL}
\DeclareMathOperator{\Br}{Br}
\DeclareMathOperator{\card}{card}
\DeclareMathOperator{\ch}{ch}
\DeclareMathOperator{\Char}{char}
\DeclareMathOperator{\CHur}{CHur}
\DeclareMathOperator{\Cl}{Cl}
\DeclareMathOperator{\coker}{coker}
\DeclareMathOperator{\Conf}{Conf}
\DeclareMathOperator{\disc}{disc}
\DeclareMathOperator{\End}{End}
\DeclareMathOperator{\et}{\text{\'et}}
\DeclareMathOperator{\Fix}{Fix}
\DeclareMathOperator{\Gal}{Gal}
\DeclareMathOperator{\GL}{GL}
\DeclareMathOperator{\Hom}{Hom}
\DeclareMathOperator{\Hur}{Hur}
\DeclareMathOperator{\im}{im}
\DeclareMathOperator{\Ind}{Ind}
\DeclareMathOperator{\Inn}{Inn}
\DeclareMathOperator{\Irr}{Irr}
\DeclareMathOperator{\lcm}{lcm}
\DeclareMathOperator{\Mor}{Mor}
\DeclareMathOperator{\ord}{ord}
\DeclareMathOperator{\Out}{Out}
\DeclareMathOperator{\Perm}{Perm}
\DeclareMathOperator{\PGL}{PGL}
\DeclareMathOperator{\Pin}{Pin}
\DeclareMathOperator{\PSL}{PSL}
\DeclareMathOperator{\rad}{rad}
\DeclareMathOperator{\sgn}{sgn}
\DeclareMathOperator{\SL}{SL}
\DeclareMathOperator{\SO}{SO}
\DeclareMathOperator{\Sp}{Sp}
\DeclareMathOperator{\Spec}{Spec}
\DeclareMathOperator{\Spin}{Spin}
\DeclareMathOperator{\St}{St}
\DeclareMathOperator{\Surj}{Surj}
\DeclareMathOperator{\Syl}{Syl}
\DeclareMathOperator{\tame}{tame}
\DeclareMathOperator{\Tr}{Tr}

\newcommand{\eps}{\varepsilon}
\newcommand{\QED}{\hspace{\stretch{1}} $\blacksquare$}
\renewcommand{\AA}{\mathbb{A}}
\newcommand{\CC}{\mathbb{C}}
\newcommand{\EE}{\mathbb{E}}
\newcommand{\FF}{\mathbb{F}}
\newcommand{\HH}{\mathbb{H}}
\newcommand{\NN}{\mathbb{N}}
\newcommand{\OO}{\mathbb{O}}
\newcommand{\PP}{\mathbb{P}}
\newcommand{\QQ}{\mathbb{Q}}
\newcommand{\RR}{\mathbb{R}}
\newcommand{\ZZ}{\mathbb{Z}}
\newcommand{\bfm}{\mathbf{m}}
\newcommand{\mcA}{\mathcal{A}}
\newcommand{\mcC}{\mathcal{C}}
\newcommand{\mcG}{\mathcal{G}}
\newcommand{\mcH}{\mathcal{H}}
\newcommand{\mcM}{\mathcal{M}}
\newcommand{\mcN}{\mathcal{N}}
\newcommand{\mcO}{\mathcal{O}}
\newcommand{\mcP}{\mathcal{P}}
\newcommand{\mcQ}{\mathcal{Q}}
\newcommand{\mcT}{\mathcal{T}}
\newcommand{\mfa}{\mathfrak{a}}
\newcommand{\mfb}{\mathfrak{b}}
\newcommand{\mfI}{\mathfrak{I}}
\newcommand{\mfM}{\mathfrak{M}}
\newcommand{\mfm}{\mathfrak{m}}
\newcommand{\mfo}{\mathfrak{o}}
\newcommand{\mfO}{\mathfrak{O}}
\newcommand{\mfP}{\mathfrak{P}}
\newcommand{\mfp}{\mathfrak{p}}
\newcommand{\mfq}{\mathfrak{q}}
\newcommand{\mfz}{\mathfrak{z}}
\newcommand{\AGL}{\mathbb{A}\GL}
\newcommand{\Qbar}{\overline{\QQ}}
\renewcommand{\qedsymbol}{$\blacksquare$}
\newcommand{\geqs}{\geqslant}
\newcommand{\leqs}{\leqslant}


%\pdfrender{StrokeColor=black,TextRenderingMode=2,LineWidth=0.4pt}

%\lhead{\pdfrender{StrokeColor=black,TextRenderingMode=2,LineWidth=0.5pt}Congruent Number Problem}   
%\chead{\pdfrender{StrokeColor=black,TextRenderingMode=2,LineWidth=0.5pt}Math Club}
%\rhead{\pdfrender{StrokeColor=black,TextRenderingMode=2,LineWidth=0.5pt}Spring 2021} 
\lhead{Cryptography}
\chead{Math Club}
\rhead{Spring 2021}

\begin{document}

\section{Introduction}

    Welcome to our final meeting of Math Club this year and our final meeting online!
    Thanks for sticking around thus far.
    As a reminder, we are still taking applications for the role of math club officer until Friday.
    Please fill out the form on Slack if you are interested.

    Today we will be discussing cryptography, or the study and practice of securing digital information from adversarial attacks.
    In other words, cryptography looks at the mathematical ways in which we can safely and securely send messages from point \(A\) to point \(B\).
    With cryptography, the sender will scramble up the message in some way so that it looks like gibberish until it gets unscrambled by the receiver.

    Some definitions:
    \begin{definition}
        We call any set of mathematical tools used for cryptography a \textbf{cryptosystem}.
        A cryptosystem \(\Sigma\) is composed of the following:
        \begin{enumerate}
            \item A \textbf{message space} \(\mathcal{M}\) from which you choose your message,
            \item A \textbf{ciphertext space} \(\mathcal{C}\) where your scrambled message lived,
            \item A \textbf{key space} \(\mathcal{K}\) from which your key is chosen,
            \item An \textbf{encryption} function \(E:\mathcal{M}\to \mathcal{C}\) that does the scrambling using the key,
            \item A \textbf{decryption} function \(D:\mathcal{C}\to \mathcal{M}\) that takes a scrambled message and unscrambles it using the key.
        \end{enumerate}
        The encryption and decryption functions have the property that \(D(E(m))=m\), so that the decryption undoes the encryption, revealing the message.
    \end{definition}

    Most of cryptography follows a certain storyline.
    Alice wants to send some kind of message to Bob.
    She wants to do so in a way that nobody other than Bob can read this message.
    Eve, an adversary to Alice and Bob, wants to try and read the message Alice sends.
    The assumption in the story is that Eve has access to anything not kept completely secret.

\section{Resources}

    A great text for learning cryptography from a mathematical point of view (or maybe learning cryptography at all) is \textit{An Introduction to Mathematical Cryptography} by Hoffstein, Pipher, and Silverman.
    Its only prerequisite is some linear algebra knowledge (Math 4A is fine).
    
    For a more applied approach, for those of you looking to implement cryptography, check out \textit{An Introduction to Modern Cryptography} by Katz and Lindell.
    As a warning: don't roll your own crypto.
    Don't write your own encryption scheme and use it in a non-trivial way.
    If someone is willing enough, your scheme \textit{will} be broken.

\section{The RSA Cryptosystem}

    The RSA cryptosystem (named for the creators Rivest, Shamir, and Adleman) allows Alice and Bob to share a message using a public key (i.e., one that Eve has access to).

    In RSA, we have that \(\mathcal{K}=\NN\).
    For the key, Bob picks two primes \(p,q\) and a positive integer \(e\) such that \(\gcd(e,(p-1)(q-1))=1\).
    The message space and ciphertext space are both the set of integers modulo \(N=pq\).
    That is, \(\mathcal{M}=\mathcal{C}=\ZZ/N\ZZ\).

    Alice chooses a message \(m\), an integer between 0 and \(pq-1\) inclusive.
    To encrypt the message, she computes \(c\equiv m^e\pmod{N}\) and sends this over to Bob.
    Bob then computes, using the Euclidean algorithm, an integer \(d\) satisfying \(ed\equiv 1\pmod{(p-1)(q-1)}\) (he can do this because he has full knowledge of what \(p\) and \(q\) are).
    Then Bob computes \(m'\equiv c^d\pmod{N}\).
    Here, we have
    \[m' \equiv c^d \equiv \left(m^e\right)^d \equiv m^{ed} \equiv m\pmod{N}.\]
    This last equivalence follows from Euler's theorem. The following table gives an outline of how RSA is done.

    \begin{table}[h!]
        \centering
        \begin{tabular}{|p{5cm}|p{5cm}|}
            \hline
            \textbf{Alice} & \textbf{Bob}\\
            \hline\hline
            \multicolumn{2}{|c|}{\bf Key creation}\\
            \hline
            Choose secret primes \(p\) and \(q\).
            Choose encryption exponent \(e\) with \(\gcd(e,(p-1)(q-1))=1\).
            Publish \(N=pq\) &\\
            \hline
            \multicolumn{2}{|c|}{\bf Encryption}\\
            \hline
            & Choose plaintext \(m\).
            Use Bob's public key \((N,e)\) to compute \(c\equiv m^e\pmod{N}\).
            Send ciphertext \(c\) to Bob.\\
            \hline
            \multicolumn{2}{|c|}{\bf Decryption}\\
            \hline
            Compute \(d\) satisfying \(ed\equiv 1\pmod{(p-1)(q-1)}\).
            Compute \(m' \equiv c^d\pmod{N}\).
            Then \(m'\) equals the plaintext \(m\).&\\
            \hline
        \end{tabular}
    \end{table}

\section{Exercises}

    \begin{exercise}
        The Euler \(\phi\)-function has some marvelous properties.
        \begin{enumerate}
            \item[(a)] If \(p\) is a prime, what is the value of \(\phi(p)\)?
            \item[(b)] If \(p\) and \(q\) are distinct primes, how is \(\phi(pq)\) related to \(\phi(p)\) and \(\phi(q)\)?
            \item[(c)] If \(p\) is prime, what is the value of \(\phi(p^2)\)?
            What about \(\phi(p^k)\) in general?
            \item[(d)] Let \(M\) and \(N\) be integers such that \(\gcd(M,N)=1\).
            Prove that
            \[\phi(MN) = \phi(M)\phi(N).\]
            \item[(e)] Let \(p_1,\ldots,p_r\) be the distinct primes that divide an integer \(N\). 
            Use your previous results to show that 
            \[\phi(N) = N\prod_{i=1}^r\left(1-\frac{1}{p_i}\right).\]
        \end{enumerate}
    \end{exercise}

    \begin{exercise}
        Let \(N\), \(c\), and \(e\) be positive integers satisfying \(\gcd(N,c)=1\) and \(\gcd(e,\phi(N))=1\).
        Explain how to solve the congruence
        \[x^e \equiv c\pmod{N}\]
        assuming that you know the value of \(\phi(N)\).
    \end{exercise}

    \begin{exercise}
        This exercise shows that one should be careful in choosing the public key.
        Sometimes, there's a way to compute the message without having to factor.
        Suppose that the message \(m\) has been encrypted using the RSA cryptosystem using three public keys \((391,3)\), \((55,3)\), and \((87,3)\).
        The resulting ciphertexts are \(208\), \(38\), and \(87\), respectively.
        Determine \(m\) without using the easy-to-find prime factors of \(391\), \(55\), and \(87\).
    \end{exercise}
    \iffalse
    Solution:
    We are trying to find a solution to the system
    \begin{align*}
        38 &\equiv m^3 \pmod{55}\\
        32 &\equiv m^3 \pmod{87}\\
        208 &\equiv m^3 \pmod{391}.
    \end{align*}
    Apply Chinese Remainder Theorem to get a value for \(x = m^3\) modulo \(55\cdot 87\cdot 391 = 1870935\).
    Lots of computation shows that \(x \equiv 103823\pmod{1870935}\).
    Taking the cube root gets \(m = \sqrt[3]{103823} = 47\).

    Note that this problem requires some fluency with the Chinese Remainder Theorem.
    \fi

    \begin{exercise}
        This exercise illustrates the dangers of reusing a key.
        Suppose Alice wants to use the RSA cryptosystem to receive message \(m\) from both Bob 1 and Bob 2.
        She picks a modulus \(N\).
        Suppose she gives Bob 1 the encryption exponent \(e_1\) and Bob 2 the encryption exponent \(e_2\).
        Moreover, suppose \(\gcd(e_1,e_2)=1\).
        If Bob 1 sends Alice \(c_1\) and Bob 2 sends Alice \(c_2\), show how Eve can use the values of \(c_1\), \(c_2\), \(e_1\), \(e_2\), and \(N\) to figure out \(m\).
    \end{exercise}
    \iffalse
    Solution: 
    Use Bezout's lemma to get \(e_1 x + e_2 y = 1\) and consider \(c_1^x c_2^y\).
    \fi

    \begin{exercise}
        Here is an example of a public key cryptosystem that was proposed at a cryptography conference.
        It was designed to be more efficient than RSA.

        Alice chooses two large primes \(p\) and \(q\) and she publishes \(N=pq\).
        It is assumed as usual that \(N\) is hard to factor.
        Alice also chooses three random numbers \(g\), \(r_1\), and \(r_2\) modulo \(N\) and computes 
        \[g_1 \equiv g^{r_1(p-1)}\pmod{N}\quad\textrm{and}\quad g_2 \equiv g^{r_2(q-1)}\pmod{N}.\]
        Alice's public key is the triple \((N,g_1,g_2)\) and her private key is the pair of primes \((p,q)\).

        Now Bob wants to send the message \(m\) to Alice, where \(m\) is taken modulo \(N\).
        He chooses two random integers \(s_1\) and \(s_2\) modulo \(N\) and computes
        \[c_1 \equiv mg_1^{s_1}\quad\textrm{and}\quad c_2 \equiv mg_2^{s_2}\pmod{N}.\]
        Bob sends the ciphertext \((c_1,c_2)\) to Alice.

        Decryption is fast and easy thanks to the Chinese Remainder Theorem.
        Alice has to solve
        \[x\equiv c_1\pmod{p}\quad\textrm{and}\quad x\equiv c_2\pmod{q}.\]

        \begin{enumerate}
            \item[(a)] Show that Alice's solution \(x\) is the same as Bob's plaintext \(m\).
            \item[(b)] Explain why the cryptosystem is not secure. 
        \end{enumerate}
    \end{exercise}

\section{Number Theory Reference Guide}

    \subsection{The Euler \(\phi\) Function}

        \begin{definition}
            The Euler \(\phi\) function, sometimes called Euler's totient function, is defined to by
            \[\phi(N) := \#\{0\leq k<N : \gcd(k,N) = 1\}.\]
            In other words, it counts how many numbers smaller than \(N\) are coprime to \(N\).
        \end{definition}

        This has a number of useful properties, some of which are outlines in the exercises above.

        \begin{theorem}[Euler's theorem]
            Let \(N\) be a positive integer.
            Then for all \(a\) with \(\gcd(a,N)=1\), we have
            \[a^{\phi(N)}\equiv 1\pmod{N}.\]
        \end{theorem}

    \subsection{Euclidean Algorithm}

        The \textbf{Euclidean algorithm} is an efficient way to compute the greatest common divisor of two integers.
        We illustrate it's use with an example.
        
        Suppose we want to compute \(\gcd(81,57)\).
        First, we take the larger number \(81\) and divide it by the smaller number, keeping track of the remainder:
        \[81 = 1\cdot 57 + 24.\]
        Note that \(\gcd(81,57)=\gcd(57,24)\).
        This fact is very important (convince yourself that this is true).
        Then we repeat the process, writing
        \[57 = 2\cdot 24 + 3\]
        \[24 = 8\cdot 3 + 0.\]
        The gcd we were searching for is the last nonzero remainder.
        In this case, \(\gcd(81,57) = 3\).

        A fact from number theory known as \textbf{Bezout's lemma} is that given integers \(a\) and \(b\), there exist integers \(x\) and \(y\) such that
        \[ax + by = \gcd(a,b).\]
        One can compute such \(x\) and \(y\) by the ``reverse'' Euclidean algorithm.
        Suppose we want to find integers \(x\) and \(y\) such that
        \[81x + 57y = \gcd(81,57) = 3.\]
        Then using the relationships we found in computing the gcd and lots of distributivity, we have
        \begin{align*}
            3 &= 9 - 1\cdot 6\\
            &= 9 - 1\cdot(24 - 2\cdot 9)\\
            &= 3\cdot 9 - 1\cdot 24\\
            &= 3\cdot(57 - 2\cdot 24) - 1\cdot 24\\ 
            &= 3\cdot 57 - 7\cdot 24\\
            &= 3\cdot 57 - 7\cdot (81 - 1\cdot 57)\\
            &= 10\cdot 57 - 7\cdot 81,
        \end{align*}
        so \(x = -7\) and \(y=10\).

        For cryptographic purposes, the Euclidean algorithm and the reverse process are both considered ``fast'' computations for a computer.
        Therefore, it is assumed that all parties (Alice, Bob, Eve, etc.) can efficiently compute these algorithms.

    \subsection{Chinese Remainder Theorem}

        \begin{theorem}[CRT]
            Let \(n_1,\ldots,n_k\) be integers greater than \(1\) and denote \(N=n_1n_2\cdots n_k\).
            Assume that \(\gcd(n_i,n_j)=1\) whenever \(i\neq j\) (i.e., any two of the \(n_i\)s are coprime).
            Let \(a_1,\ldots,a_k\) be integers.
            Then the system of congruences
            \begin{align*}
                x &\equiv a_1\pmod{n_1}\\
                x &\equiv a_2\pmod{n_2}\\
                &\vdots\\
                x &\equiv a_k\pmod{n_k}
            \end{align*}
            has a unique solution \(\pmod{N}\).
        \end{theorem}

        As an example, consider an ancient puzzle appearing in a book by 3rd century Chinese mathematician Sun-tzu:
        ``There are certain things whose number is unknown. 
        If we count them by threes, we have two left over; by fives, we have three left over; and by sevens, two are left over. 
        How many things are there?''
        In other words, we are being asked to solve the system
        \begin{align*}
            x &\equiv 2\pmod{3}\\
            x &\equiv 3\pmod{5}\\
            x &\equiv 2\pmod{7}.
        \end{align*}
        The CRT guarantees a unique solution \(\pmod{105}\), so here's how we compute it.
        We begin by noting that \(x\equiv 2\pmod 3\) means that \(x=2+3k\) for some \(k\).
        Substituting into the second congruence, we get 
        \[x=2+3k\equiv 3\pmod 5.\]
        This simplifies to
        \[3k\equiv 1\pmod 5.\]
        Multiplying both sides of the equivalence by 2, we get
        \[6k\equiv 2\pmod 5\implies k\equiv 2\pmod 5\implies k=2 + 5\ell\]
        Substituting this back into \(x\), we get 
        \[x = 2 + 3k = 2 + 3(2+5\ell) = 8 + 15\ell.\]
        Substituting \(x\) into the final equation, we get
        \[x = 8+15\ell\equiv 2\pmod 7\implies 1+\ell\equiv 2\pmod 7\implies \ell\equiv 1\pmod 7\implies \ell = 1+7m.\]
        Substituting this back into \(x\), we get
        \[x = 8 + 15\ell = 8 + 15(1+7m) = 23 + 105m.\]
        Thus the unique solution \(\pmod{105}\) is \(23\). 
        You can check that this does indeed satisfy the original congruences.

        From a cryptographic perspective, the CRT is considered a ``fast'' computation, so you may assume that all parties can efficiently compute solutions to systems of congruences like these.

\end{document}