\documentclass{article}
\usepackage[utf8]{inputenc}
\usepackage{hyperref}
\usepackage{pdfrender,xcolor}
\usepackage{graphicx}
\usepackage[T1]{fontenc}
\usepackage[charter]{mathdesign}



\usepackage{amsmath,amssymb,amsthm}
\usepackage{mathtools,graphicx,tikz-cd}
\usepackage{blindtext}
\usepackage[margin = 1.25 in]{geometry}
\usepackage{enumitem}

\usepackage{fancyhdr,accents,lastpage}
\pagestyle{fancy}
\setlength{\headheight}{25pt}

\linespread{1.3}

\newtheorem{theorem}{Theorem}[section]
\newtheorem{corollary}{Corollary}[theorem]
\newtheorem{lemma}[theorem]{Lemma}

\theoremstyle{definition}
\newtheorem{definition}{Definition}[section]

\theoremstyle{remark}
\newtheorem*{remark}{Remark}
\newtheorem{exercise}{Exercise}
\newtheorem{example}{Example}[definition]
\newtheorem*{solution}{Solution}


\DeclareMathOperator{\ab}{ab}
\DeclareMathOperator{\Aut}{Aut}
\DeclareMathOperator{\BGL}{BGL}
\DeclareMathOperator{\Br}{Br}
\DeclareMathOperator{\card}{card}
\DeclareMathOperator{\ch}{ch}
\DeclareMathOperator{\Char}{char}
\DeclareMathOperator{\CHur}{CHur}
\DeclareMathOperator{\Cl}{Cl}
\DeclareMathOperator{\coker}{coker}
\DeclareMathOperator{\Conf}{Conf}
\DeclareMathOperator{\disc}{disc}
\DeclareMathOperator{\End}{End}
\DeclareMathOperator{\et}{\text{\'et}}
\DeclareMathOperator{\Fix}{Fix}
\DeclareMathOperator{\Gal}{Gal}
\DeclareMathOperator{\GL}{GL}
\DeclareMathOperator{\Hom}{Hom}
\DeclareMathOperator{\Hur}{Hur}
\DeclareMathOperator{\im}{im}
\DeclareMathOperator{\Ind}{Ind}
\DeclareMathOperator{\Inn}{Inn}
\DeclareMathOperator{\Irr}{Irr}
\DeclareMathOperator{\lcm}{lcm}
\DeclareMathOperator{\Mor}{Mor}
\DeclareMathOperator{\ord}{ord}
\DeclareMathOperator{\Out}{Out}
\DeclareMathOperator{\Perm}{Perm}
\DeclareMathOperator{\PGL}{PGL}
\DeclareMathOperator{\Pin}{Pin}
\DeclareMathOperator{\PSL}{PSL}
\DeclareMathOperator{\rad}{rad}
\DeclareMathOperator{\sgn}{sgn}
\DeclareMathOperator{\SL}{SL}
\DeclareMathOperator{\SO}{SO}
\DeclareMathOperator{\Sp}{Sp}
\DeclareMathOperator{\Spec}{Spec}
\DeclareMathOperator{\Spin}{Spin}
\DeclareMathOperator{\St}{St}
\DeclareMathOperator{\Surj}{Surj}
\DeclareMathOperator{\Syl}{Syl}
\DeclareMathOperator{\tame}{tame}
\DeclareMathOperator{\Tr}{Tr}

\newcommand{\eps}{\varepsilon}
\newcommand{\QED}{\hspace{\stretch{1}} $\blacksquare$}
\renewcommand{\AA}{\mathbb{A}}
\newcommand{\CC}{\mathbb{C}}
\newcommand{\EE}{\mathbb{E}}
\newcommand{\FF}{\mathbb{F}}
\newcommand{\HH}{\mathbb{H}}
\newcommand{\NN}{\mathbb{N}}
\newcommand{\OO}{\mathbb{O}}
\newcommand{\PP}{\mathbb{P}}
\newcommand{\QQ}{\mathbb{Q}}
\newcommand{\RR}{\mathbb{R}}
\newcommand{\ZZ}{\mathbb{Z}}
\newcommand{\bfm}{\mathbf{m}}
\newcommand{\mcA}{\mathcal{A}}
\newcommand{\mcC}{\mathcal{C}}
\newcommand{\mcG}{\mathcal{G}}
\newcommand{\mcH}{\mathcal{H}}
\newcommand{\mcM}{\mathcal{M}}
\newcommand{\mcN}{\mathcal{N}}
\newcommand{\mcO}{\mathcal{O}}
\newcommand{\mcP}{\mathcal{P}}
\newcommand{\mcQ}{\mathcal{Q}}
\newcommand{\mfa}{\mathfrak{a}}
\newcommand{\mfb}{\mathfrak{b}}
\newcommand{\mfI}{\mathfrak{I}}
\newcommand{\mfM}{\mathfrak{M}}
\newcommand{\mfm}{\mathfrak{m}}
\newcommand{\mfo}{\mathfrak{o}}
\newcommand{\mfO}{\mathfrak{O}}
\newcommand{\mfP}{\mathfrak{P}}
\newcommand{\mfp}{\mathfrak{p}}
\newcommand{\mfq}{\mathfrak{q}}
\newcommand{\mfz}{\mathfrak{z}}
\newcommand{\AGL}{\mathbb{A}\GL}
\newcommand{\Qbar}{\overline{\QQ}}
\renewcommand{\qedsymbol}{$\blacksquare$}
\newcommand{\isp}[1]{\quad\text{#1}\quad}


\lhead{A Little Algebra} 
\chead{Math Club}
\rhead{Fall 2020} 


\begin{document}


\section{Introduction}
Today, we will be discussing algebra: the ``$4x^3+5x=10$, solve for $x$'' kind, also known as elementary algebra. But don't be fooled! Elementary does \emph{not} mean easy; it just means that it doesn't require any deep mathematical machinery that takes a semester to develop. We will be going over several clever algebraic tricks that are invaluable to any problem-solver's toolkit and may help you on the Putnam.

There are three main tricks to consider when proving algebraic identities:
\begin{enumerate}[label=\Roman*.]
    \item Adding and subtracting the same thing: \[a^4+4b^4 = a^4 + 4b^4 + 0 = a^4 + 4b^4 + 4b^2-4b^2 = (a^4+ 4b^2 + 4b^4) - 4b^2\]
    This adding and subtracting of $4b^2$ might seem useless, but it is actually very powerful, because both terms on the right hand side are squares, so it combines very well with the next trick.
    \item Factoring and Expanding: 
    \[(a^4 + 4b^2 + 4b^4) - 4b^2 = (a^2+2b^2)^2 - (2b)^2 = (a^2 + 2b^2 + 2b)(a^2+2b^2-2b)\]
    First we used the square-of-sum expansion, $(x+y)^2 = x^2 + 2xy + y^2$, with $x=a^2$, $y=2b^2$, and then we used the difference-of-squares factorization, $x^2-y^2 = (x+y)(x-y)$, with $x=a^2+2b^2$ and $y=2b$. This concludes the proof of the \emph{Sophie-Germain identity}: $a^4+4b^4 = (a^2+2b^2 + 2b)(a^2+2b^2-2b)$.
    
    An essential part of this kind of algebraic manipulation is \emph{knowing how to create and recognize squares and cubes}. Examples are given.
    \begin{enumerate}[label=(\alph*)]
    \item $x^n - y^n = (x-y)(x^{n-1}+x^{n-2}y + \cdots + xy^{n-2}+y^{n-1})$ 
    \item If $n$ is odd, $x^n + y^n=(x+y)(x^{n-1}-x^{n-2}y+x^{n-3}y^2-\cdots + x^2y^{n-3}-xy^{n-2}+y^{n-1})$
    \item $x^{4} + 4y^{4} = (x^2 + 2y^2 + 2y)(x^2 + 2y^2-2y)$ 
    \item $x^3 + y^3 + z^3 - 3xyz = (x + y + z)(x^2 + y^2 + z^2 -xy - xz - yz)$\\$ = \frac{1}{2}(x+y+z)\left[ (x-y)^2 + (y-z)^2 + (x-z)^2\right]$ 
    \item $(x+y)^2 =x^2+2xy+y^2$
    \item $(x-y)^2 = x^2 -2xy+y^2$
    \item $(x+y)^3 = x^3 + 3x^2y +3xy^2 + y^3 = x^3 + y^3 +3xy(x+y)$
    \item $(x-y)^3 = x^3 - 3x^2y+3xy^2-y^3 = x^3 - y^3 -3xy(x-y)$
    \item (The Binomial Theorem) $(x+y)^n = \sum_{k=0}^n \binom{n}{k} x^{n-k}y^k$
\end{enumerate}
    \item Substitution and Simplification
    This final trick is often used to vastly simplify complicated-looking equations. The heuristic here is, \emph{always move in the direction of greater beauty or simplicity}. In other words, \emph{be lazy!} For example, supposed you're asked to find the product of the solutions to the equation \begin{equation}x^2+18x+30 = 2\sqrt{x^2 + 18x + 45}\end{equation} While you could certainly square both sides and solve the resulting quartic (ugh) equation, it would be much easier to let $y=\sqrt{x^2 + 18x + 45}$, and find all possible solutions for $y$. The equation would become
    \[y^2-15 = 2y\]
    Now it's easy: solve the quadratic equation for $y$ (by factoring or the quadratic formula), and once you have the value of $y$, you can plug back into $y=\sqrt{x^2 + 18x + 45}$ to find the solutions for $x$. \emph{We essentially just reduced the problem of solving a quartic equation into that of solving two quadratic equations}.
\end{enumerate}

\section{Easy-Mediumish}

To the problems! Each exercise relies on some technique mentioned above.

\begin{exercise}[]
If $x+y = 4$ and $x^2 + y^2 = 3$, then find $xy$.
\end{exercise}

\begin{exercise}[]
If $xy=x+y = 3$, find $x^3+y^3$.
\end{exercise}

\begin{exercise}[]
Simplify
\[\left(\sqrt{5}+\sqrt{6}+\sqrt{7}\right)\left(\sqrt{5}+\sqrt{6}-\sqrt{7}\right)\left(\sqrt{5}-\sqrt{6}-\sqrt{7}\right)\left(\sqrt{5}-\sqrt{6}+\sqrt{7}\right)\]
\end{exercise}

\begin{exercise}[]
Given $x^2 + y^2 + z^2 = 1$, find the minimum value of $xy+xz+yz$. No calculus!
\end{exercise}

\begin{exercise}[]
Solve the system of equations (you don't need row reduction here!)
\begin{align*}
    2x_1+x_2+x_3+x_4+x_5&=6 \\
    x_1+2x_2+x_3+x_4+x_5&=12 \\
    x_1+x_2+2x_3+x_4+x_5&=24 \\
    x_1+x_2+x_3+2x_4+x_5&=48 \\
    x_1+x_2+x_3+x_4+2x_5&=96 \\
\end{align*}
\end{exercise}

\begin{exercise}
How many integer solutions \((a,b)\) does \(ab-3b-2a=7\) have?
\end{exercise}

\begin{exercise}
Verify that
\[\sqrt[3]{20+14\sqrt{2}}+\sqrt[3]{20-14\sqrt{2}}=4\]
\end{exercise}

\begin{exercise}
If the expression 
\[(x^3-x^2y+xy^2+y^3)^5\] 
is expanded and simplified, what is the sum of all the coefficients of the resulting polynomial?
\end{exercise}

\section{Hard}

\begin{exercise}
Find all triples $x,y,z$ of integers such that 
\begin{equation*}
    x^3 + y^3 + z^3 -3xyz = p
\end{equation*}
where $p$ is a prime strictly greater than 3.
\end{exercise}

\begin{exercise}
Solve for $x$:
\[\sqrt[3]{x-1} + \sqrt[3]{x}  + \sqrt[3]{x+1} = 0\]
\end{exercise}

\begin{exercise}
Suppose that $a,b,c$ are distinct real numbers. Show that 
\begin{equation*}
    \sqrt[3]{a - b} + \sqrt[3]{b - c}  + \sqrt[3]{c - a} \neq 0
\end{equation*}
\end{exercise}

\begin{exercise}
Show that for no positive integer $n$ can both $n+3$ and $n^2+3n+3$ be perfect cubes.
\end{exercise}

\begin{exercise}
Prove that for any nonnegative integer $n$, the number
\[5^{5^{n+1}}+5^{5^n} +1\] is not prime.
\end{exercise}

\begin{exercise}
Prove that the number
\[\frac{5^{125}-1}{5^{25}-1}\]
is not prime.
\end{exercise}

\begin{exercise}[2019 A1]
Determine all possible values of the expression
\[A^3+B^3+C^3-3ABC,\]
where \(A\), \(B\), and \(C\) are nonnegative integers. Note that solving exercise 7 first would be helpful.
\end{exercise}

\begin{exercise}[1977 A2]
Determine all solutions in real numbers $x,y,z,w$ of the system
\begin{align*}
    x+y+z&= w \\
    \frac{1}{x}+\frac{
    1}{y}+\frac{1}{z} &= \frac{1}{w}
\end{align*}
\end{exercise}

\pagebreak

\section{Hints}

\begin{enumerate}
    \item Expand $(x+y)^2$ and plug in the given information to find the quantity you want.
    \item Expand $(x+y)^3$ and plug in the given information to find the quantity you want. 
    \item Can you use the difference of squares factorization $x^2-y^2 = (x-y)(x+y)$ here?
    \item Squares are positive, so $(x+y+z)^2\geq 0$. Now expand.
    \item Add!
    \item 6 is an important number
    \item You can do this one.
    \item Don't expand this. Your time is worth more than that.
\end{enumerate}

\end{document}