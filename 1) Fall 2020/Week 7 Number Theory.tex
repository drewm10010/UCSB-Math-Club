\documentclass{article}
\usepackage[utf8]{inputenc}
\usepackage{hyperref}
\usepackage{pdfrender,xcolor}
\usepackage{graphicx}
\usepackage[T1]{fontenc}
\usepackage[charter]{mathdesign}



\usepackage{amsmath,amssymb,amsthm}
\usepackage{mathtools,graphicx,tikz-cd}
\usepackage{blindtext}
\usepackage[margin = 1.25 in]{geometry}
\usepackage{enumitem}

\usepackage{fancyhdr,accents,lastpage}
\pagestyle{fancy}
\setlength{\headheight}{25pt}

\linespread{1.3}

\newtheorem{theorem}{Theorem}[section]
\newtheorem{corollary}{Corollary}[theorem]
\newtheorem{lemma}[theorem]{Lemma}

\theoremstyle{definition}
\newtheorem{definition}{Definition}[section]

\theoremstyle{remark}
\newtheorem*{remark}{Remark}
\newtheorem{exercise}{Exercise}
\newtheorem{example}{Example}[definition]
\newtheorem*{solution}{Solution}


\DeclareMathOperator{\ab}{ab}
\DeclareMathOperator{\Aut}{Aut}
\DeclareMathOperator{\BGL}{BGL}
\DeclareMathOperator{\Br}{Br}
\DeclareMathOperator{\card}{card}
\DeclareMathOperator{\ch}{ch}
\DeclareMathOperator{\Char}{char}
\DeclareMathOperator{\CHur}{CHur}
\DeclareMathOperator{\Cl}{Cl}
\DeclareMathOperator{\coker}{coker}
\DeclareMathOperator{\Conf}{Conf}
\DeclareMathOperator{\disc}{disc}
\DeclareMathOperator{\End}{End}
\DeclareMathOperator{\et}{\text{\'et}}
\DeclareMathOperator{\Fix}{Fix}
\DeclareMathOperator{\Gal}{Gal}
\DeclareMathOperator{\GL}{GL}
\DeclareMathOperator{\Hom}{Hom}
\DeclareMathOperator{\Hur}{Hur}
\DeclareMathOperator{\im}{im}
\DeclareMathOperator{\Ind}{Ind}
\DeclareMathOperator{\Inn}{Inn}
\DeclareMathOperator{\Irr}{Irr}
\DeclareMathOperator{\lcm}{lcm}
\DeclareMathOperator{\Mor}{Mor}
\DeclareMathOperator{\ord}{ord}
\DeclareMathOperator{\Out}{Out}
\DeclareMathOperator{\Perm}{Perm}
\DeclareMathOperator{\PGL}{PGL}
\DeclareMathOperator{\Pin}{Pin}
\DeclareMathOperator{\PSL}{PSL}
\DeclareMathOperator{\rad}{rad}
\DeclareMathOperator{\sgn}{sgn}
\DeclareMathOperator{\SL}{SL}
\DeclareMathOperator{\SO}{SO}
\DeclareMathOperator{\Sp}{Sp}
\DeclareMathOperator{\Spec}{Spec}
\DeclareMathOperator{\Spin}{Spin}
\DeclareMathOperator{\St}{St}
\DeclareMathOperator{\Surj}{Surj}
\DeclareMathOperator{\Syl}{Syl}
\DeclareMathOperator{\tame}{tame}
\DeclareMathOperator{\Tr}{Tr}

\newcommand{\eps}{\varepsilon}
\newcommand{\QED}{\hspace{\stretch{1}} $\blacksquare$}
\renewcommand{\AA}{\mathbb{A}}
\newcommand{\CC}{\mathbb{C}}
\newcommand{\EE}{\mathbb{E}}
\newcommand{\FF}{\mathbb{F}}
\newcommand{\HH}{\mathbb{H}}
\newcommand{\NN}{\mathbb{N}}
\newcommand{\OO}{\mathbb{O}}
\newcommand{\PP}{\mathbb{P}}
\newcommand{\QQ}{\mathbb{Q}}
\newcommand{\RR}{\mathbb{R}}
\newcommand{\ZZ}{\mathbb{Z}}
\newcommand{\bfm}{\mathbf{m}}
\newcommand{\mcA}{\mathcal{A}}
\newcommand{\mcC}{\mathcal{C}}
\newcommand{\mcG}{\mathcal{G}}
\newcommand{\mcH}{\mathcal{H}}
\newcommand{\mcM}{\mathcal{M}}
\newcommand{\mcN}{\mathcal{N}}
\newcommand{\mcO}{\mathcal{O}}
\newcommand{\mcP}{\mathcal{P}}
\newcommand{\mcQ}{\mathcal{Q}}
\newcommand{\mfa}{\mathfrak{a}}
\newcommand{\mfb}{\mathfrak{b}}
\newcommand{\mfI}{\mathfrak{I}}
\newcommand{\mfM}{\mathfrak{M}}
\newcommand{\mfm}{\mathfrak{m}}
\newcommand{\mfo}{\mathfrak{o}}
\newcommand{\mfO}{\mathfrak{O}}
\newcommand{\mfP}{\mathfrak{P}}
\newcommand{\mfp}{\mathfrak{p}}
\newcommand{\mfq}{\mathfrak{q}}
\newcommand{\mfz}{\mathfrak{z}}
\newcommand{\AGL}{\mathbb{A}\GL}
\newcommand{\Qbar}{\overline{\QQ}}
\renewcommand{\qedsymbol}{$\blacksquare$}
\newcommand{\isp}[1]{\quad\text{#1}\quad}


\lhead{Number Theory} 
\chead{Math Club}
\rhead{Fall 2020} 


\begin{document}


\section{Introduction}

    Welcome back to Math Club in week 7! (not 5040). 
    Today we will be playing with some concepts in elementary number theory. 
    We will cover modular arithmetic as well as some stuff about primes and divisibility, all of which will be helpful in not just math competitions but \textit{every single one of your math classes moving forward}. 

    First we will introduce the concept of modular arithmetic.

    \begin{definition}
        Let \(x\), \(y\), and \(n\) be integers. 
        We say that \(x\) is congruent (or equivalent) to \(y\) modulo (or mod) \(n\) if \(x\) and \(y\) differ by a multiple of \(n\). 
        We write \(x\equiv y\pmod{n}\).
    \end{definition}

    \begin{example}
        Find the last digit of \(2^{1234}\).
    \end{example}

    Next we will discuss the idea of divisibility of integers.

    \begin{definition}
        Let \(a,b\in\ZZ\). 
        We say that $b$ divides $a$, in symbols $b \mid a$, if there exists an integer $k$ such that $a=kb$. We write $b\nmid a$ to mean ``$b$ does not divide $a$.''
    \end{definition}

    \begin{definition}
        Let \(a,b\in\NN\). 
        Then their greatest common divisor, written \((a,b)\) or \(\gcd(a,b)\), is the largest integer that divides both \(a\) and \(b\). 
        Their least common multiple, written \(\lcm(a,b)\), is the smallest positive integer that is a multiple of both \(a\) and \(b\).

        If \(\gcd(a,b)=1\), we say that \(a\) and \(b\) are coprime or relatively prime.
    \end{definition}
    
    Here are some basic facts that can be proven about divisibility (you can try the first two, but for the last two, consult any introductory number theory text).
    
    \begin{theorem}
        The following hold:
        \begin{enumerate}
            \item If \(g\mid a\) and \(g\mid b\), then \(g\mid ax+by\) for any integers \(x,y\)
            \item \(\gcd(a+b,b)=\gcd(a,b)\)
            \item \(\gcd(a,b)\cdot\lcm(a,b)=a\cdot b\)
            \item If \(p\) is prime and \(p\mid ab\), then \(p\mid a\) or \(p\mid b\)
        \end{enumerate}
    \end{theorem}
    
    \begin{theorem}[Division Algorithm]
    Let $a, b\in \ZZ$ with $b\neq 0$. There exist unique integers $q,r$ such that $a=bq+r$ and $0 \leq r < |b|$.
    \end{theorem}

    Here are some useful results that will help you in your problem-solving endeavors.

    \begin{theorem}[Fermat's Little Theorem]
        Let \(a\in\ZZ\) and let \(p\) be a prime number. 
        Then
        \[a^p\equiv a\pmod{p}.\]
        If \(p\nmid a\), then 
        \[a^{p-1}\equiv 1\pmod{p}\]
    \end{theorem}

    \begin{theorem}[Bezout's Identity/Lemma]
        Let \(a,b\in\NN\). 
        Then there exist \(x,y\in\ZZ\) such that
        \[ax+by=\gcd(a,b).\]
        and moreover, $\gcd(a,b)$ is the least positive integer expressible in this form.
    \end{theorem}
    
    \begin{theorem}[Fundamental Theorem of Arithmetic]
    Every positive integer $n>2$ can be represented in exactly one way as the product of prime powers
    \[n=p_1^{k_1}p_2^{k_2}\cdots p_r^{k_r}=\prod_{i=1}^r p_i^{k_i}\] where $p_1<p_2<\cdots<p_r$ are primes and the $k_i$ are positive integers.
    \end{theorem}

    As usual, resources for these problems (and many more) can be found on the Slack page. 
    Happy problem solving!
    Today's problems are not split into categories, but instead arranged roughly in order of difficulty.

\section{Problems}

    Try to prove some of the results given in the introduction section.

    \begin{exercise}
        Prove that there are infinitely many prime numbers.
    \end{exercise}

    \begin{exercise}
        Let \(n\) be an odd positive integer not divisible by 3. Show that \(n^2-1\) is divisible by 24.
    \end{exercise}

    \begin{exercise}
        A \textit{palindrome} is a positive integer that reads the same forward and backward, like 2552 or 1991. 
        Find a positive integer greater than 1 that divides all four-digit palindromes.
    \end{exercise}

    \begin{exercise}
        Find and integer \(c\) such that \(x^2+18x+c\) is a perfect square for all integers \(x\). 
        Prove that this choice of \(c\) is unique. 
    \end{exercise}

    \begin{exercise}[Due to Sun-tzu, 3rd century]
        There are certain things whose number is unknown. 
        If we count them by threes, we have two left over; by fives, we have three left over; and by sevens, two are left over. 
        How many things are there?
    \end{exercise}

    \begin{exercise}
        Show that any two consecutive Fibonacci numbers are relatively prime. 
        Recall that the Fibonacci numbers are defined recursively by \(F_1=F_2=1\) and \(F_n=F_{n-1}+F_{n-2}\) for \(n\geq 3\).
    \end{exercise}

    \begin{exercise}
        Determine whether there exist three positive integers \(a,b,c\) such that \(a+b\), \(b+c\), and \(a+c\) are all pairwise distinct prime numbers.
    \end{exercise}

    \begin{exercise}
        Let \(k=2020^2+2^{2020}\).
        What is the last digit of
        \[2^k+k^2\mathord{?}\]
    \end{exercise}

    \begin{exercise}
        Prove that there are infinitely many primes of the form \(4k+3\), where \(k\in\NN\).
    \end{exercise}

    \begin{exercise}
        Show that if \(a^2+b^2=c^2\), then \(3\mid ab\).
    \end{exercise}

    \begin{exercise}
        Show that the fraction \(\dfrac{21n+4}{14n+3}\) is irreducible for all positive integers \(n\).
    \end{exercise}

    \begin{exercise}
        Prove that the number \(n=1,280,000,401\) is composite.
    \end{exercise}

    \begin{exercise}
        Let \(N\) be a number with nine distinct non-zero digits, such that, for each \(k\) from 1 to 9 inclusive, the first \(k\) digits of \(N\) form a number that is divisible by \(k\). 
        Find \(N\).
    \end{exercise}

    \begin{exercise}[2000 Putnam B2]
        Prove that the expression
        \[\frac{\gcd(m,n)}{n}\binom{n}{m}\]
        is an integer for all pairs of integers \(n\geq m\geq 1\).
    \end{exercise}
    
    \begin{exercise}
        If $a\in \NN$ and $p$ is a prime number for which $p$ divides $(a^7-1)/(a-1)$, prove that either $p \equiv 1 \pmod{7}$ or $p=7$.
    \end{exercise}
    
    \begin{exercise}
        Find all integer solutions of the equation
        \[\frac{a^7-1}{a-1} = b^5-1\]
    \end{exercise}
    
    \begin{exercise}
        If $a\equiv b \pmod{n}$ prove that $a^n \equiv b^n \pmod{n^2}$.
    \end{exercise}
    
    \begin{exercise}
        Let $p$ be a prime. Show that there are infinitely many positive integers $n$ such that $p$ divides $2^n-n$.
    \end{exercise}
    
    \begin{exercise}
        Let $n>1$ be a positive integer. Prove that 
        \[1+\frac{1}{2}+\frac{1}{3}+\cdots + \frac{1}{n}\] is not an integer.
    \end{exercise}

\newpage
\section{Hints}

    \begin{enumerate}
        \item Suppose there are finitely many (say, \(n\)) and consider \(N=p_1p_2\cdots p_n+1\).
        \item \(24=2^3\cdot 3\).
        \item Place value.
        \item Complete the square.
        \item Formulate in the language of modular arithmetic.
        \item Induction is generally good on recursive sequences.
        \item Two cases: First, one of the primes is equal to 2. 
        Second, all the primes are greater than 2.
        \item Reduce \(k\) modulo 4 and then you can get the last digit of \(2^k\). 
        Expand \(k^2\).
        \item Do exercise 1 first.
        Similarly, consider the number \(N=4p_1p_2\cdots p_n-1\).
        \item Look at the values of perfect squares modulo 3.
        \item If \(\dfrac{a}{b}\) is reducible, then so is \(\dfrac{2a}{3b}\).
        \item Set \(n=x^7+x^3+x^0\) and find \(x\).
        \item Know (or derive) your divisibility rules!
        \item Use Bezout's lemma.
    \end{enumerate}

\end{document}