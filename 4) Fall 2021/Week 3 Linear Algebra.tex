\documentclass{article}

\usepackage{../mathclub}

\title{Linear Algebra - Week 3}
\author{}
\date{October 11, 2021}

\begin{document}

\section{Introduction}

Welcome to week 3 of Math Club!
This week, we're going to learn some critical tools for dealing with linear-alegbraic-type problems.
Linear algebra concerns the study of vector spaces, linear maps, matrices, and other similar objects.
Today, we will not concern ourselves much with the theory, but rather how we can use these tools from linear algebra to solve problems.

First, given two matrices, we can multiply them together with the following formula:

\[AB = \begin{bmatrix}
a_{11}&a_{12}&\cdots&a_{1n}\\
a_{21}&a_{22}&\cdots&a_{2n}\\
\vdots&\vdots&\ddots&\vdots\\
a_{m1}&a_{m2}&\cdots&a_{mn}
\end{bmatrix}\begin{bmatrix}
b_{11}&b_{12}&\cdots&b_{1p}\\
b_{21}&b_{22}&\cdots&b_{2p}\\
\vdots&\vdots&\ddots&\vdots\\
b_{n1}&b_{n2}&\cdots&b_{np}
\end{bmatrix} = \begin{bmatrix}
c_{11}&c_{12}&\cdots&c_{1p}\\
c_{21}&c_{22}&\cdots&c_{2p}\\
\vdots&\vdots&\ddots&\vdots\\
c_{m1}&c_{m2}&\cdots&c_{mp}
\end{bmatrix} = C.
\]
such that $c_{ij} = \sum_{k = 1}^{n} a_{ik}b_{kj}$ for $i = 1, 2, \ldots, m$ and $j \in 1, 2, \ldots, p$.
In particular, for \(2\times 2\) matrices, we have
\[\begin{bmatrix}a&b\\c&d\end{bmatrix}\begin{bmatrix}e&f\\g&h\end{bmatrix} = \begin{bmatrix}ae+bg&af+bh\\ce+dg&cf+dh\end{bmatrix}.\]
Note that matrix multiplication is not always commutative. (Try this yourself! Come up with a pair of matrices that don't commute!)

The \textbf{determinant} of an $n \times n$ matrix $A = (a_{ij})$, denoted by $\det A$, has the formula 
\[\det A = a_{11} \det A_{11} - a_{12} \det A_{12} + \cdots + (-1)^{n + 1} \det A_{1n}, 
\]
where $A_{ij}$ is the $(n - 1) \times (n - 1)$ matrix obtained by deleting the $i$-th row and $j$-th column. Some properties for the determinant: 
\begin{itemize}
    \item $\det(AB) = \det(A) \cdot \det(B)$ 
    \item $\det(A^{T}) = \det(A)$
    \item The determinant of a diagonal of triangular matrix is the product of the entries on the main diagonal.
    \item There exists an \(n\times n\) matrix \(A^{-1}\) such that \(AA^{-1}=A^{-1}A=I\) if and only if \(\det A\neq 0\).
    Here, \(I\) is the identity matrix which has 1s in the main diagonal and 0s elsewhere.
    In this case, we say \(A\) is \textbf{invertible} and \(A^{-1}\) is the \textbf{inverse} of \(A\).
\end{itemize}

The \textbf{trace} of a square matrix \(\Tr(A)\) is the sum of the entries on the main diagonal.
Some properties of trace include:
\begin{itemize}
    \item $\Tr(A + cB) = \Tr(A) + c\Tr(B)$
    \item $\Tr(AB) = \Tr(BA)$
    \item $\Tr(A) = \Tr(A^{T})$
    \item $\Tr(ABC) = \Tr(CAB)$ but $\Tr(ABC) = \Tr(ACB)$ does not always hold true
\end{itemize}
Of course, we haven't listed every property of the determinant or trace, so feel free to derive your own results!

\section{Problems}

\begin{exercise}
    Let \(A\) be a skew-symmetric.
    That is, \(A^T=-A\).
    Find \(\det A\).
\end{exercise}

\begin{exercise}
            Show that if $A$, $B$ are similar square matrices ($A = QBQ^{-1}$ for some invertible matrix $Q$), then $A$ and $B$ have the same determinant.   
        \end{exercise}

        \begin{exercise}
            Do there exist square matrices $A, B$ such that $AB - BA = I_{n}?$  
        \end{exercise}
        
        \begin{exercise}
            Show that if $A, B$ are square matrices such that $A + B = AB$, then $AB = BA$. 
        \end{exercise}
        
        \begin{exercise}
        The Fibonacci sequence $(F_{n})$ is defined by $F_{0} = 0$, $F_{1} = 1$, $F_{n} = F_{n - 1} + F_{n - 2}$. Prove that 
        \[\begin{bmatrix}
            1&1\\
            1&0\\
        \end{bmatrix}^{n} = \begin{bmatrix}
            F_{n + 1}&F_{n}\\
            F_{n}&F_{n - 1}\\
            \end{bmatrix}.
        \]
        \end{exercise}
        
        \begin{exercise}
        Show that 
        \[F_{n + 1}F_{n - 1} - F_{n}^{2} = (-1)^{n} \quad \text{for}\quad n \geq 1. 
        \]
        Try proving this in two ways: once by induction, and a second time using linear algebra.
        \end{exercise}
        
        \begin{exercise}
            Let \(A\) and  \(B\) be matrices of size \(n\times n\) with complex entries that satisfy.
            \[A^2 = B^2 = (AB)^2 = I.\]
            Prove that \(A\) and \(B\) commute.
        \end{exercise}
        
        % \begin{exercise}
        % Prove that 
        % \[
        % \det \begin{bmatrix}
        % (x^{2} + 1)^{2}&(xy + 1)^{2}&(xz + 1)^{2}\\
        % (xy + 1)^{2}&(y^{2} + 1)^{2}&(yz + 1)^{2}\\
        % (xz + 1)^{2}&(yz + 1)^{2}&(z^{2} + 1)^{2}
        % \end{bmatrix} = 2(y - z)^{2}(z - x)^{2}(x - y)^{2}.\]
        % \end{exercise}
        
        \begin{exercise}
        For any $n \times n$ matrix $A$ with real entries,
        \[
        \det(I_{n} + A^{2}) \geq 0.
        \]
        \end{exercise}
        
        \begin{exercise}
        (Shoelace formula) Show that if a triangle in the plane has coordinates $(x_{1}, y_{1})$, $(x_{2}, y_{2})$, and $(x_{3}, y_{3})$, then its area is the absolute value of:
        \[\frac{1}{2} \det \begin{bmatrix}
        x_{1}&y_{1}&1\\
        x_{2}&y_{2}&1\\
        x_{3}&y_{3}&1\\
        \end{bmatrix}.
        \]
        \end{exercise}

        \begin{exercise}
            Let $A$ and $B$ be $n \times n$ matrices with real entries satisfying 
            \[\Tr(A A^{T} + B B^{T}) = \Tr(AB + A^{T}B^{T}).  
            \]
            Prove that $A = B^{T}$. 
            \end{exercise}
            
            \begin{exercise}
                Let $A, B, C$ be $n \times n$ matrices, $n \geq 1$, satisfying 
            \[ABC + AB + BC + AC + A + B + C = 0.
            \]
            Prove that $A$ and $B + C$ commute if and only if $A$ and $BC$ commute. 
            \end{exercise}
            
            % \begin{exercise}
            % Let $p < m$ be two positive integers. Prove that 
            % \[\det \begin{bmatrix}
            % \binom{m}{0}&\binom{m}{1}&\cdots&\binom{m}{p}\\
            % \binom{m + 1}{0}&\binom{m + 1}{1}&\cdots&\binom{m + 1}{p}\\
            % \vdots&\vdots&\ddots&\vdots\\
            % \binom{m + p}{0}&\binom{m + p}{1}&\cdots&\binom{m + p}{p} 
            % \end{bmatrix} = 1.
            % \]
            % \end{exercise}
            
            % \begin{exercise}
            % (Vandermonde Matrices) Show that the matrix (where $x_{1}, \ldots, x_{n} \in \mathbb{R}$)
            % \[\begin{bmatrix}
            % 1&1&\cdots&1\\
            % x_{1}&x_{2}&\cdots&x_{n}\\
            % x_{1}^{2}&x_{2}^{2}&\cdots&x_{n}^{2}\\
            % \vdots&\vdots&\ddots&\vdots\\
            % x_{1}^{n - 1}&x_{2}^{n - 1}&\cdots&x_{n}^{n - 1}\\
            % \end{bmatrix}
            % \]
            % is invertible if and only if $x_{i} \neq x_{j}$ whenever $i \neq j$. 
            % \end{exercise}

            % \begin{exercise}
            %     Let $M_n$ be the $(2n+1)\times (2n+1)$ matrix for which \[(M_n)_{ij}=\begin{cases} 0 & i=j \\
            %     1 & i-j=1,\dots,n \pmod{2n+1} \\ -1 & i-j=n+1,\dots,2n \pmod{2n+1}\end{cases}\] Find the rank of $M_n$.
            % \end{exercise}

\begin{exercise}
            Let $n>2$. Prove that there exists an $n\times n$ matrix with entries in $\{\pm 1\}$ whose determinant is larger than $\sqrt{n!}$.
        \end{exercise}
        
        \begin{exercise}
            Let \(M_n\) be the \(n\times n\) matrix with entries as follows: for \(i=1,2,\ldots,n\) we have \(m_{i,i}=10\); for \(i=1,2,\ldots,n-1\) we have \(m_{i+1,i}=m_{i,i+1}=3\); all other entires are 0.
            Let \(D_n=\det M_n\).
            Then we can write
            \[\sum_{n=1}^\infty \frac{1}{8D_n+1} = \frac{p}{q}\in\QQ\]
            Find the integers \(p,q\) in lowest terms.
        \end{exercise}
        
        % \begin{exercise}
        %     Prove that there is an absolute constant $c>0$ with the following property. Let $A$ be an $n$ by $n$ matrix with pairwise distinct entries. Then there is a permutation of rows in $A$ such that no column in the permuted matrix contains an increasing subsequence of length at least $c\sqrt{n}$.
        % \end{exercise}
        
        \begin{exercise}[1991 A2]
        Let $A$ and $B$ be different $n \times n$ matrices with real entries. If $A^{3}$ = $B^{3}$ and $A^{2}B = B^{2}A$, can $A^{2} + B^{2}$ be invertible?
    \end{exercise}
    
        \begin{exercise}[2014 A2]
        Let $A$ be the $n \times n$ matrix whose entry in the $i$-th row and $j$-th column is
        \[\frac{1}{\min(i, j)}
        \]
        for $1 \leq i, j \leq n$. Compute $\det(A)$.
    \end{exercise}
    
    % \begin{exercise}[1990 A5]
    %     If $A$ and $B$ are square matrices of the same size such that $ABAB = 0$, does it follow that $BABA = 0$?
    % \end{exercise}
    
    % \begin{exercise}[1999 B5]
    %     For an integer $n\geqslant 3$, let $\theta = 2\pi / n$. Evaluate the determinant of the $n\times n$ matrix $I+A$, where $I$ is the $n\times n$ identity matrix and $A=(a_{jk})$ has entries $a_{jk} = \cos(j\theta+k\theta)$ for all $j,k$. 
    % \end{exercise}
    
    \begin{exercise}[2011 A4]
        For which positive integers $n$ is there an $n \times n$ matrix with integer entries such that every dot product of a row with itself is even, while every dot product of two different rows is odd?
    \end{exercise}
    
    \begin{exercise}[2016 B4]
        Let \(A\) be a \(2n\times 2n\) matrix with entries chosen at random.
        Each entry is chosen to be 0 or 1, each with probability \(1/2\).
        Find the expected value of \(\det(A-A^T)\) as a function of \(n\), where \(A^T\) denoted the transpose of \(A\).
    \end{exercise}

\end{document}