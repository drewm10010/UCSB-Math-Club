\documentclass{article}

\usepackage{../mathclub}

\title{Introduction to Math Club}
\author{}
\date{September 27, 2021}

\begin{document}

\section{Introduction}

    Welcome to UCSB's Math Club!
    Our goal as a math club is to improve our problem solving skills, learn some new math, and build a community of math-minded individuals.
    
    This fall, we do problem-solving with an eye towards the Putnam competition (December 4, 2021).
    In the competition, you will be given 12 problems to be done over 6 hours, and generally they are quite challenging. 
    If you can correctly answer 4 or more of the problems, you will generally be ranked in the top 200 in the nation!
    Of course, there is no obligation for you to take the Putnam! 
    It is simply a convenient excuse for us to try our hand at solving tough problems, and sharpen our analytical thinking skills. 
    The skills developed in Math Club will help you become a better mathematician in (in our opinion) a fun way!
    
    Today we will get to know our new and returning members and solve some problems together.
    Make sure you like our Facebook page and join our Slack channel to keep up with our communications.
    Happy problem solving!     
    
    Note: These problems are not in order of difficulty necessarily.

\section{Resources}

We are creative people, so naturally we stole these problems from various sources.
These sources include:
\begin{itemize}
    \item {\it Lateral Thinking Puzzles} by Paul Sloan
    \item {\it Putnam and Beyond} by Andreescu and Gelca
    \item {\it The Art and Craft of Problem Solving} by Paul Zeitz
\end{itemize}
All of our handouts from last year and this year may be found on Github: \url{https://github.com/drewm10010/UCSB-Math-Club}.
They may also be found in the Slack page, though you may have to scroll a bit.

\section{Problems}

\begin{exercise}
The governing body of the state of Lateralia was extremely concerned about the uneven distribution of wealth in the country. 
They thought it unfair that the richest man should have so much more than his poorer compatriots. 
They therefore instituted a wealth tax decreeing that each year the wealthiest man in the country had to give away his money by doubling the wealth of every other citizen, starting with the poorest and working up to the second wealthiest person if possible. 
The decree was carried out, and the richest person gave away his money by doubling the wealth of all other citizens. 
However, the governing body was shocked to find that this action had made no difference to the overall distribution of wealth nor to the relative wealth of the poorest and richest citizens. 
How could this be so?
Assume everyone has {\it some} wealth.
\end{exercise}

\begin{exercise}
Define $f(x)=1/(1-x)$ and denote $r$ iterations of the function $f$ by $f^r$, so \[f^r(x)=\underbrace{f(f(\cdots(f(x))\cdots))}_{r\text{ $f$'s}}\] 
Determine $f^{1999}(2000)$.
\end{exercise}

\begin{exercise}[Licking Frogs]
You are lost in the jungles of Brazil. 
After days of wandering, your food supplies dwindle, and you make a fatal mistake by eating a poisonous mushroom. 
You can feel the poison coursing through your veins, sure that you will collapse any second. 
But there is hope. 
The antidote to the poison is secreted by a certain species of frog found in this rainforest, and you can save yourself by licking one of these frogs. 
But, only the female frogs secret the antidote you need. 
The male and female frogs look identical, and they occur in equal numbers across the population. 
The only distinguishing feature is that the male frogs have a unique croak.

As your vision starts to blur, you look up and see one of these frogs sitting on a stump in front of you. 
You are about to make a mad dash to the frog, praying that it is female, when you hear the male frog's distinctive croak behind you. 
You turn around and see that there are two frogs on the grass in a clearing, just about as far away from you as the one on the stump. 
You do not know which one of the two frogs in the clearing croaked.

You only have time to reach the one frog on the stump, or the two frogs in the clearing (one of which croaked) before you pass out. 
Should you dash to the stump and lick the one frog, or into the clearing and lick the two?
\end{exercise}

\begin{exercise}[Scale Part I]
A shopkeeper wants to be able to dispense sugar in whole pounds ranging from one pound up to 40 pounds.
He has a standard, equal-arm balance weigh scale.
Being of an extremely economical outlook, he wants to use the least number of weights to enable him to weigh any number of pounds between 1 and 40.
How many weights does he need and what are they?
\end{exercise}

\begin{exercise}[Scale Part II]
You have an equal-arm balance scale and twelve solid balls. You are told that one of the balls has a different weight from all the others, but you do not know whether it is lighter or heavier. You can weigh the balls against each other in the scale balance. Can you find the odd ball and tell if it is lighter or heavier in only three weighings?
\end{exercise}

\begin{exercise}
Two missiles speed directly toward each other, one at 9,000 miles per hour and the other at 21,000 miles per hour. 
They start at 4,857 miles apart. 
Without using pencil and paper (or similar tools), calculate how far apart they are one minute before they collide.
\end{exercise}

\begin{exercise}
Find the minimum value of $(u-v)^2+(\sqrt{2-u^2}-\frac{9}{v})^2$ for $0<u<\sqrt{2}$ and $v>0$.
\end{exercise}

\begin{exercise}
Cover the plane with non-overlapping squares such that only two of them are the same size.
\end{exercise}

\begin{exercise}
All the students in a school are arranged in a rectangular array. 
After that, the tallest student in each row was chosen, and then among these Rufus T. Barleysheath happened to be the shortest. 
Then, in each column, the shortest student was chosen, and Ogbert the Nerd was the tallest of these. 
Who is taller: Rufus or Ogbert?
\end{exercise}

\begin{exercise}[Green-Eyed Dragons]
You visit a remote desert island inhabited by one hundred very friendly dragons, all of whom have green eyes. 
They haven’t seen a human for many centuries and are very excited about your visit. They show you around their island and tell you all about their dragon way of life (dragons can talk, of course).
    
They seem to be quite normal, as far as dragons go, but then you find out something rather odd. They have a rule on the island that states that if a dragon ever finds out that he/she has green eyes, then at precisely midnight at the end of the day of this discovery, he/she must relinquish all dragon powers and transform into a long-tailed sparrow. 
However, there are no mirrors on the island, and the dragons never talk about eye color, so they have been living in blissful ignorance throughout the ages.

Upon your departure, all the dragons get together to see you off, and in a tearful farewell you thank them for being such hospitable dragons. 
You then decide to tell them something that they all already know (for each can see the colors of the eyes of all the other dragons): You tell them all that at least one of them has green eyes. Then you leave, not thinking of the consequences (if any). 
Assuming that the dragons are (of course) infallibly logical, what happens? 
If something interesting does happen, what exactly is the new information you gave the dragons?
\end{exercise}


\end{document}