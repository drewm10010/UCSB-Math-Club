\documentclass{article}

\usepackage{../mathclub}
\usepackage{stix}
\def\rddots#1{\cdot^{\cdot^{\cdot^{#1}}}}

\title{Convergence Day - Week 6}
\author{}
\date{November 1, 2021}

\begin{document}

\section{Introduction}

Welcome to Week 6 of Math Club! Today we will be taking a stab at limits and convergence! For many of you, this will not be your first time dealing with convergence. Today we will formally reintroduce the topic, with a little $\epsilon$s, and $\delta$s here and there. 
\\\\
\emph{Definition of Convergence of Sequences} \\
A sequence $x_n$ is said to converge to a number $L$, if for any $\epsilon > 0$, there exists an integer $N$ such that for any $n > N$, we have $|x_n - L| < \epsilon$. 
\\\\
\emph{Cauchy Criterion for Sequence Convergence} \\
A sequence converges (in $\mathbb{R}^n$) if and only if for any $\epsilon > 0$, there exists an integer $N$ such that for any $n,m > N$, we have $|x_n - x_m| < \epsilon$.
\\\\
\emph{Definition of Convergence of Series} \\
A series $\sum a_{n}$ converges if the sequence of partial sums $S_{N} = \sum_{n = 1}^{N} a_{n}$ converges.  
\\\\
\emph{Cauchy's Convergence Test} \\
A series converges if and only if for any $\epsilon > 0$, there exists an integer $N$ such that for any $m \geq n > N$, $|\sum_{k = n}^{m} a_{k}| < \epsilon$.
\\\\
Examples of Taylor expansion of functions: 
\begin{itemize}
    \item $e^{x} = 1 + x + \cdots + \frac{x^{n}}{n!} + o(x^{n})$;
    \item $\sin x = x - \frac{x^{3}}{3!} + \cdots + (-1)^{k} \frac{x^{2k + 1}}{(2k + 1)!} + o(x^{2k + 1})$; 
    \item $\cos x = 1 - \frac{x^{2}}{2!} + \cdots + (-1)^{k} \frac{x^{2k}}{(2k)!} + o(x^{2k})$; 
    \item $\frac{1}{1 - x} = 1 + x + \cdots + x^{n} + o(x^{n})$;
\end{itemize}


\section{Problems}

\begin{exercise}
Verify using the formal definition of convergence that \[\lim_{n\to\infty} \frac{1}{n} = 0\] 
\end{exercise}

\begin{exercise}
Consider the following experiment with a fair coin: toss the coin twice, and consider it a \textit{success} if both tosses turn up heads.
The success probability of this experiment is \(\frac 14\).
Devise and experiment which only uses tosses of a fair coin, but which has success probability \(\frac 13\).
\end{exercise}

\begin{exercise}
Evaluate \[\lim_{x\to\infty} \left(\left(1+\frac{1}{x}\right)^xx-ex\right)\]
\end{exercise}

\begin{exercise}
Prove that the equation \(x^{x^{x^{\rddots{}}}}=2\) is satisfied by \(x=\sqrt 2\), but \(x^{x^{x^{\rddots{}}}}=4\) has no solution.
What is the ``break point''?
\end{exercise}

\begin{exercise}[Forward Cauchy Criterion]
    Let $x_n$ be a convergent sequence. Prove that for any $\epsilon > 0$, there exists an integer $N$ such that for any $n,m > N$, we have $|x_n - x_m| < \epsilon$. (It helps to recall that $|x+y| \leq |x| + |y|$)
\end{exercise}

\begin{exercise}
Let $\{x_n\}$ be a sequence satisfying \[\lim_{n\to\infty} (x_n-x_{n-1}) = 0\] Prove that \[\lim_{n\to\infty} \frac{x_n}{n}= 0\]
\end{exercise}

\begin{exercise}
Let $\{x_n\}$ be a sequence of real numbers such that \[ \lim_{n\to\infty} (2x_{n+1}-x_n)=L\] Prove that $\{x_n\}$ converges and its limit is $L$.
\end{exercise}

\begin{exercise}
Compute
\[\lim_{n\to\infty}\left(\sum_{k=1}^n \frac{n}{k^2+n^2}\right)\] 
\end{exercise}

\begin{exercise}
Determine if 
\[\sum_{n = 1}^{\infty} \left(e - (1 + \frac{1}{n})^{n}\right)
\]
converges. 
\end{exercise}

        \begin{exercise}
            Evaluate
            \[1+\frac{1}{1+\frac{1}{1+\frac{1}{1+\cdots}}}\]
            That is, compute the limit of the sequence $\{x_n\}$, where $x_1 = 1$ and $x_{n+1}=1+\frac{1}{x_n}$.
        \end{exercise}

\begin{exercise}
            Set, recursively, \((x_0,y_0)\) with \(0<x_0<y_0\) and
            \[x_{n+1}=\frac{x_n+y_n}{2}\quad\textrm{and}\quad y_{n+1}=\sqrt{x_{n+1}y_n}.\]
            Moreover, we are given (although one may derive it) that
            \[x_n<y_n\implies x_{n+1}<y_{n+1}\quad\textrm{and}\quad y_{n+1}-x_{n+1}<\frac{y_n-x_n}{4}\text{ for all }n.\]
            Find the common limit \(\displaystyle\lim_{n\to\infty}x_n=\lim_{n\to\infty}y_n=x=y\).
        \end{exercise}
        
        \begin{exercise}
    Prove that a sequence of positive numbers, each of which is less than the average of the previous two,
is convergent.
\end{exercise}

\begin{exercise}
If $\sum a_{n}^{2}$ and $\sum b_{n}^{2}$ converge, prove that $\sum a_{n}b_{n}$ converges. 
\end{exercise}

\begin{exercise}
$\sum a_{n}$ diverges, $a_{n} > 0$. Let $S_{n} = a_{1} + \cdots + a_{n}$. Show that $\sum \frac{a_{n}}{S_{n}}, \sum \frac{a_{n}}{1 + a_{n}}$ diverges. Show that $\sum \frac{a_{n}}{S_{n}^{2}}, \sum \frac{a_{n}}{1 + n^{2}a_{n}}$ converges. 
\end{exercise} 

\begin{exercise}[1970 B1]
Evaluate \[\lim_{n\to\infty}\left(\frac{1}{n^4} \prod_{i=1}^{2n} (n^2+i^2)^{1/n}\right)\]
\end{exercise}

\begin{exercise}[2020 A3]
Let $a_{0} = \frac{\pi}{2}$ and $a_{n} = \sin(a_{n - 1})$ for $n \geq 1$. Determine whether 
\[\sum_{n = 1}^{\infty} a_{n}^{2}\]
converges. 
\end{exercise}

\begin{exercise}
Prove that the product 
\[ \prod_{n=2}^\infty \left(1 + \frac{e^{in(\theta)}}{\log n}\right)\]
is not convergent for any rational value of $\frac{\theta}{\pi}$, but is convergent if $\frac{\theta}{\pi}$ is an algebraic number which is not rational. 
\end{exercise}


\end{document}