\documentclass{article}

\usepackage{../mathclub}

\title{Calculus - Week 4}
\author{}
\date{October 18, 2021}

\begin{document}

\section{Introduction}

Welcome all to week 4 of Math Club!
Today we'll discuss some topics from calculus and (real) analysis that might be helpful to you in your problem-solving endeavors.
Analysis, specifically real analysis, is essentially an extension of the things one would learn in calculus with an extra dash of \(\eps\)'s and \(\delta\)'s.
Hopefully today's resources will give you added confidence when you see analysis problems on the Putnam, should you take it.

We provide some important theorems below.
You have most likely encountered all of these in your calculus class, but they bear repeating:
\begin{itemize}
    \item Intermediate Value Theorem: if $f$ is continuous on $[a,b]$ then every value between $f(a)$ and $f(b)$ is of the form $f(c) $ for some $c\in [a,b]$.
    \item Extreme Value Theorem: if $f$ is continuous on $[a,b]$, there exists $y\in [a,b]$ such that $f(y)\leq f(x)$ for all $x\in [a,b]$.
    \item Mean Value Theorem: if $f$ is continuous on $[a,b]$ and differentiable on $(a,b)$, then there exists $u$ with $a < u < b$ such that $f'(u)=\frac{f(b)-f(a)}{b-a}$.
    \item Fundamental Theorem of Calculus: if $f$ is continuous at $c$, the function $F(x) = \int_a^x f(t)\ dt$ is differentiable at $c$ and $F'(c)=f(c)$.
\end{itemize}
    

\section{Problems}

\begin{exercise}
Recall the formula for integration by parts:
\[\int f\ \mathrm{d}g = fg - \int g\ \mathrm{d}f.\]
Applying this to \(f(x)=1/x\) and \(g(x)=x\), we get
\[\int\frac{1}{x}\mathrm{d}x = 1 + \int\frac{1}{x}\mathrm{d}x\]
from which we enthusiastically conclude that \(0=1\).
Why are we wrong? Or is math as we know it wrong?
\end{exercise}

\begin{exercise}
Suppose \(f:\RR\to\RR\) is differentiable and defined by
\[f(x) = f'(2)x^2 + x.\]
Find the value \(f(2)\).
\end{exercise}

\begin{exercise}
What is the 100th derivative of \(f(x)=e^x\cos(x)\) evaluated at \(x=\pi\)?
\end{exercise}

\begin{exercise}
Suppose \(a\) and \(b\) are real numbers such that 
\[\lim_{x\to 0}\frac{\sin^2(x)}{e^{ax}-bx-1} = \frac{1}{2}.\]
Determine all possible ordered pairs \((a,b)\).
\end{exercise}

\begin{exercise}
Suppose \(f(x) = e^{ax} + e^{bx}\) where \(a\neq b\) and that \(f''(x)-2f'(x)-15=0\) for all \(x\).
Give all possible ordered pairs \((a,b)\).
\end{exercise}

\begin{exercise}
A very tired audience of 9001 attends a concert of Haydn's Surprise Symphony, which lasts 20 minutes.
Members of the audience fall asleep at a continuous rate of \(6t\) people per minute, where \(t\) is the time in minutes since the symphony has begun.
The Surprise Symphony is named so because when \(t=8\) minutes, the orchestra plays exactly one very loud note, waking everyone in the audience up.
After that note, though, the audience continues to fall asleep at the same rate as before.
Once a member of the audience falls asleep, they will stay asleep except for the rude awakening at \(t=8\) minutes.
How many collective minutes does the audience sleep during the symphony?
\end{exercise}

\begin{exercise}
Compute the integral
\[\int_2^4\frac{\ln\sqrt{9-x}}{\ln\sqrt{9-x}+\ln\sqrt{x+3}}\mathrm{d}x.\]
\end{exercise}

\begin{exercise}[Extreme Value Theorem]
Suppose $f$ is continuous on $[a,b]$, and assume $f(x)>0$ for all $a\leq x \leq b$. Prove that there is a positive constant $c$ for which $c\leq f(x)$ for all $x\in [a,b]$.
\end{exercise}

\begin{exercise}[Intermediate Value Theorem]
Suppose $g:[0,1]\to [0,1]$ is a continuous function. Prove that $g$ has a fixed point in $[0,1]$, i.e., some $x\in [0,1]$ such that $g(x)=x$.
\end{exercise}

\begin{exercise}[Fundamental Theorem of Calculus]
Find all real-valued continuously differentiable functions on the real line such that for all $x$
\[(f(x))^2 = \int_0^x \left((f(t))^2 + (f'(t))^2\right)\ \mathrm{d}t + 1990\]
\end{exercise}

\begin{exercise}
Let \(f(x)=(x^2-1)^n\), where \(n\) is a positive integer.
Determine, in terms of \(n\) the number of distinct roots of \(f^{(n)}(x)\) (the \(n\)th derivative of \(f\)) in the intervals \((-\infty,-1)\), \((-1,1)\) and \((1,+\infty)\), respectively.
\end{exercise}

\begin{exercise}
Show that $4ax^3 + 3bx^2 + 2cx = a + b + c$ has at least one root between 0 and 1.
\end{exercise}

\begin{exercise}
The integral
\[\int_0^{\pi/2}\frac{x}{\tan(x)}\mathrm{d}x\]
can be written in the form \(a^b\pi\ln(c)\), where \(a,b,c\in\ZZ\) and \(c\) is as small as possible.
Compute \(a+b+c\).
Hint available upon request.
\end{exercise}
% The hint is to set the integral
% \[I(a) = \int_0^{\pi/2}\frac{\arctan(a\tan(x))}{\tan(x)}\mathrm{d}x\]
% and differentiate under the integral sign. Then with some clever algebraic manipulation, one can get
% \[I'(a) = \frac{\pi}{2(a+1)}\implies I(a) = \frac{\pi}{2}\ln(a+1)+C.\]
% Note that \(C = I(0)\). The integral in question is \(I(1)\).

\section{Beware, these Putnam problems may bite!}

\begin{exercise}[1968 A1]
    Prove that
    \[\frac{22}{7}-\pi = \int_0^1\frac{x^4(1-x)^4}{1+x^2}\ \mathrm{d}x.\]
\end{exercise}

\begin{exercise}[1994 B2]
    For which real numbers $c$ is there a straight line that intersects $y=x^4+9x^3+cx^2+9x+4$ in four distinct points?
\end{exercise}

\begin{exercise}[2000 B4]
    Let $f(x)$ be a continuous function such that $f(2x^{2} - 1) = 2xf(x)$ for all $x$. Show that $f(x) = 0$ for $-1 \leq x \leq 1$. 
\end{exercise}

\begin{exercise}[2019 A6]
Let $g$ be a real-valued function that is continuous on $[0,1]$ and twice differentiable on the open interval $(0,1)$. Suppose that for some real number $r>1$, 
\[\lim_{x\to 0^+} \frac{g(x)}{x^r}=0\] Prove that either 
\[\lim_{x\to 0^+} g(x)=0\quad\textrm{or}\quad\limsup_{x\to 0^+}x^r|g''(x)| = +\infty\]
\end{exercise}



\end{document}