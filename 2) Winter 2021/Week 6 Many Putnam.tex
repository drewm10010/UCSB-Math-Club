\documentclass{article}
\usepackage[utf8]{inputenc}
\usepackage{hyperref}
\usepackage{pdfrender,xcolor}
\usepackage{graphicx}

\usepackage{amsmath,amssymb,amsthm}
\usepackage{mathtools,graphicx,tikz-cd}
\usepackage{blindtext}
\usepackage[margin = 1.25 in]{geometry}
\usepackage{enumitem}
\usepackage{physics}

\usepackage{fancyhdr,accents,lastpage}
\pagestyle{fancy}
\setlength{\headheight}{25pt}

\newtheorem{theorem}{Theorem}[section]
\newtheorem{corollary}{Corollary}[theorem]
\newtheorem{lemma}[theorem]{Lemma}

\theoremstyle{definition}
\newtheorem{definition}{Definition}[section]

\theoremstyle{remark}
\newtheorem*{remark}{Remark}
\newtheorem{exercise}{Exercise}
\newtheorem{example}{Example}[definition]


\DeclareMathOperator{\ab}{ab}
\DeclareMathOperator{\area}{area}
\DeclareMathOperator{\Aut}{Aut}
\DeclareMathOperator{\BGL}{BGL}
\DeclareMathOperator{\Br}{Br}
\DeclareMathOperator{\card}{card}
\DeclareMathOperator{\ch}{ch}
\DeclareMathOperator{\Char}{char}
\DeclareMathOperator{\CHur}{CHur}
\DeclareMathOperator{\Cl}{Cl}
\DeclareMathOperator{\coker}{coker}
\DeclareMathOperator{\Conf}{Conf}
\DeclareMathOperator{\disc}{disc}
\DeclareMathOperator{\End}{End}
\DeclareMathOperator{\et}{\text{\'et}}
\DeclareMathOperator{\Fix}{Fix}
\DeclareMathOperator{\Gal}{Gal}
\DeclareMathOperator{\GL}{GL}
\DeclareMathOperator{\Hom}{Hom}
\DeclareMathOperator{\Hur}{Hur}
\DeclareMathOperator{\im}{im}
\DeclareMathOperator{\Ind}{Ind}
\DeclareMathOperator{\Inn}{Inn}
\DeclareMathOperator{\Irr}{Irr}
\DeclareMathOperator{\lcm}{lcm}
\DeclareMathOperator{\Mor}{Mor}
\DeclareMathOperator{\ord}{ord}
\DeclareMathOperator{\Out}{Out}
\DeclareMathOperator{\Perm}{Perm}
\DeclareMathOperator{\PGL}{PGL}
\DeclareMathOperator{\Pin}{Pin}
\DeclareMathOperator{\PSL}{PSL}
\DeclareMathOperator{\rad}{rad}
\DeclareMathOperator{\sgn}{sgn}
\DeclareMathOperator{\SL}{SL}
\DeclareMathOperator{\SO}{SO}
\DeclareMathOperator{\Sp}{Sp}
\DeclareMathOperator{\Spec}{Spec}
\DeclareMathOperator{\Spin}{Spin}
\DeclareMathOperator{\St}{St}
\DeclareMathOperator{\Surj}{Surj}
\DeclareMathOperator{\Syl}{Syl}
\DeclareMathOperator{\tame}{tame}

\newcommand{\eps}{\varepsilon}
\newcommand{\QED}{\hspace{\stretch{1}} $\blacksquare$}
\renewcommand{\AA}{\mathbb{A}}
\newcommand{\CC}{\mathbb{C}}
\newcommand{\EE}{\mathbb{E}}
\newcommand{\FF}{\mathbb{F}}
\newcommand{\HH}{\mathbb{H}}
\newcommand{\NN}{\mathbb{N}}
\newcommand{\OO}{\mathbb{O}}
\newcommand{\PP}{\mathbb{P}}
\newcommand{\QQ}{\mathbb{Q}}
\newcommand{\RR}{\mathbb{R}}
\newcommand{\ZZ}{\mathbb{Z}}
\newcommand{\bfm}{\mathbf{m}}
\newcommand{\mcA}{\mathcal{A}}
\newcommand{\mcC}{\mathcal{C}}
\newcommand{\mcG}{\mathcal{G}}
\newcommand{\mcH}{\mathcal{H}}
\newcommand{\mcM}{\mathcal{M}}
\newcommand{\mcN}{\mathcal{N}}
\newcommand{\mcO}{\mathcal{O}}
\newcommand{\mcP}{\mathcal{P}}
\newcommand{\mcQ}{\mathcal{Q}}
\newcommand{\mfa}{\mathfrak{a}}
\newcommand{\mfb}{\mathfrak{b}}
\newcommand{\mfI}{\mathfrak{I}}
\newcommand{\mfM}{\mathfrak{M}}
\newcommand{\mfm}{\mathfrak{m}}
\newcommand{\mfo}{\mathfrak{o}}
\newcommand{\mfO}{\mathfrak{O}}
\newcommand{\mfP}{\mathfrak{P}}
\newcommand{\mfp}{\mathfrak{p}}
\newcommand{\mfq}{\mathfrak{q}}
\newcommand{\mfz}{\mathfrak{z}}
\newcommand{\AGL}{\mathbb{A}\GL}
\newcommand{\Qbar}{\overline{\QQ}}
\renewcommand{\qedsymbol}{$\blacksquare$}
\renewcommand{\d}[1]{\ensuremath{\operatorname{d}\!{#1}}}

\pdfrender{StrokeColor=black,TextRenderingMode=2,LineWidth=0.5pt}

\lhead{\pdfrender{StrokeColor=black,TextRenderingMode=2,LineWidth=0.5pt}A Massive Load of Putnam Problems} 
\chead{\pdfrender{StrokeColor=black,TextRenderingMode=2,LineWidth=0.5pt}Math Club}
\rhead{\pdfrender{StrokeColor=black,TextRenderingMode=2,LineWidth=0.5pt}Winter 2021} 


\begin{document}

\section{Introduction}

    Welcome to this week of Math Club!
    The Putnam exam is approaching (Feb 20), so we will be spending some time preparing and practicing.
    Last week, our focus was on solving the ``easy'' problems on the Putnam, namely problems A1, B1, A2, and B2. This week, there are no restrictions: we're jumping all over the place. You will see a lot more problems today which will reflect the average difficulty of a Putnam problem, and you will certainly learn a lot from each problem on this worksheet, whether you solve it on your own, use a hint, or read the solution somewhere. It just goes to show how helpful Putnam preparation and problem-solving can be to your study of higher mathematics.
    
    You may have seen some of these problems before.
    If that is the case, we encourage you to work on those that are unfamiliar to you.
    The main point of today is to show you how to get full points on a question and then to practice doing so.
    As a reminder, don't feel bad at all if you're unable to completely solve these.
    After all, the median score on the Putnam is somewhere in the range of 0, 1, or 2 (each question is worth 10 points).
    
    We will work through the following problem as an example.

\section{Resources}
    There are several, but one very helpful Putnam competition websites is \url{https://kskedlaya.org/putnam-archive/}, compiled by Kiran Kedlaya of UCSD.
    This site has lots of problems and solutions (as well as results) from many previous exams.
    
    John Scholes compiled a collection of solutions to all the Putnam exams up to 2003. A copy of his site can be found here: \url{https://prase.cz/kalva/putnam.html}. 
    
    Highly recommended book: \emph{The William Lowell Putnam Mathematical Competition 1985-2000: Problems, Solutions, and Commentary} by Kedlaya, Poonen, and Vakil. Several Putnam problems are presented along with hints, detailed solutions, discussions of concepts and related problems. This book was made specifically for Putnam preparation.

\section{Matrices and Linear Algebra}

\begin{exercise}[1991 A2]
Let $A$ and $B$ be different $n\times n$ matrices with real entries. If $A^3 = B^3$ and $A^2B = B^2A$, can $A^2 +B^2$
be invertible? \emph{Hint: 13}
\end{exercise}

\begin{exercise}[1988 A6]
If a linear transformation $A$ on an $n$-dimensional vector space has $n+1$ eigenvectors such that any $n$ of them are linearly independent, does it follow that $A$ is a scalar multiple of the identity? Prove your answer. \emph{Hint: 8} 
\end{exercise}

\begin{exercise}[1999 B5]
For an integer $n\geqslant 3$, let $\theta = 2\pi / n$. Evaluate the determinant of the $n\times n$ matrix $I+A$, where $I$ is the $n\times n$ identity matrix and $A=(a_{jk})$ has entries $a_{jk} = \cos(j\theta+k\theta)$ for all $j,k$. \emph{Hint: 3}
\end{exercise}

\begin{exercise}[1985 B6]
Let $G$ be a finite set of real $n \times n$ matrices $\{M_i\}$, $1 \leqslant i \leqslant r$, which form a group under matrix multiplication. Suppose that $\sum_{i=1}^r \tr (M_i) = 0$, where $\tr (A)$ denotes the trace of the matrix $A$. Prove that $\sum_{i=1}^r M_i$ is the $n \times n$ zero matrix. \emph{Hint: 1}
\end{exercise}

\section{Calculus/Analysis}

\begin{exercise}[1989 A2]
Evaluate $\int_0^a \int_0^b e^{\max\{b^2x^2,a^2y^2\}}\ dy\ dx$, where $a,b>0$. \emph{Hint: 4} 
\end{exercise}

\begin{exercise}[1993 B4]
The function $K(x,y)$ is positive and continuous for $0\leqslant x \leqslant 1$, $0\leqslant y \leqslant 1$, and the functions $f(x)$ and $g(x)$ are positive and continuous for $0\leqslant x \leqslant 1$. Suppose that for all $x$, $0\leqslant x \leqslant 1$,
\[\int_0^1f(y)K(x,y)\ dy = g(x)\ \ \text{ and }\ \ \int_0^1 g(y)K(x,y)\ dy = f(x)\] Show that $f(x)=g(x)$ for $0\leqslant x \leqslant 1$. \emph{Hint: 12}
\end{exercise}

\begin{exercise}[1999 A5]
Prove that there is a constant $C$ such that, if $p(x)$ is a polynomial of degree $1999$, then
\[|p(0)|\leqslant C\int_{-1}^1 |p(x)|\ dx\]
\emph{Hint: 14}
\end{exercise}

\section{Integer Polynomials}

\begin{exercise}[2016 A1]
Find the smallest positive integer $j$ such that for every polynomial $p(x)$ with integer coefficients and for every integer $k$, the integer
\[p^{(j)}(k) = \frac{d^j}{dx^j}p(x)\bigg|_{x=k}\] (the $j$-th derivative of $p(x)$ at $k$) is divisible by 2016. \emph{Hint: 10}
\end{exercise}

\begin{exercise}[2007 B1]
Let $f$ be a nonconstant polynomial with positive integer coefficients. Prove that if $n$ is a positive integer, then $f(n)$ divides $f(f(n)+1)$ if and only if $n=1$. \emph{Hint: 5}
\end{exercise}

\section{Binary Operations and Group Theory}

\begin{exercise}[2012 A2]
Let $*$ be a commutative and associative binary operation on a set $S$. Assume that for every $x$ and $y$ in $S$, there exists $z$ in $S$ such that $x*z=y$. (This $z$ may depend on $x$ and $y$.) Show that if $a,b,c$ are in $S$ and $a*c=b*c$, then $a = b$.
\end{exercise}

\begin{exercise}[2007 A5]
Suppose that a finite group $G$ has exactly $n$ elements of order $p$, where $p$ is a prime. Prove that either $n=0$ or $p$ divides $n+1$. \emph{Hint: 11}
\end{exercise}

\begin{exercise}[1997 A4]
Let $G$ be a group with identity $e$ and $\phi: G \rightarrow G$ a function such that 
\[\phi(g_{1})\phi(g_{2})\phi(g_{3}) = \phi(h_{1})\phi(h_{2})\phi(h_{3})
\]
whenever $g_{1}g_{2}g_{3} = e = h_{1}h_{2}h_{3}$. Prove that there exists an element $a \in G$ such that $\psi(x) = a\phi(x)$ is a homomorphism. 
\end{exercise}

\section{Number Theory}

\begin{exercise}[2017 A4]
A class with $2N$ students took a quiz, on which the possible scores were $0, 1,\dots , 10$. Each of these scores occurred at least once, and the average score was exactly $7.4$. Show that the class can be divided into two groups of $N$ students in such a way that the average score for each group was exactly $7.4$. 
\end{exercise}

\begin{exercise}[1998 B5]
Let $N$ be the positive integer with $1998$ decimal digits, all of them $1$; that is, \[N=1111\cdots 1\] Find the thousandth digit after the decimal point of $\sqrt{N}$. \emph{Hint: 6}
\end{exercise}

\section{Geometry}

\begin{exercise}[2012 B2]
Let $P$ be a given (non-degenerate) polyhedron. Prove that there is a constant $c(P ) > 0$ with the following property: If a collection of $n$ balls whose volumes sum to $V$ contains the entire surface of $P$, then $n > c(P)/V^2$.
\end{exercise}

\begin{exercise}[2000 A5]
Three distinct points with integer coordinates lie on a circle of radius $r>0$. Show that two of these points are separated by a distance of at least $r^{1/3}$. \emph{Hint: 2} 
\end{exercise}

\section{Other Fun Problems}

\begin{exercise}[2002 A3]
Let $n\geqslant 2$ be an integer and $T_n$ be the number of non-empty subsets $S$ of $\{1, 2, 3, \dots, n\}$ with the property that the average of the elements of $S$ is an integer. Prove that $T_n - n$ is always even. \emph{Hint: 9}
\end{exercise}

\begin{exercise}[1993 B5]
Show that there do not exist four points in the Euclidean plane such that the pairwise distances between the points are all integers. \emph{Hint: 7} 
\end{exercise}

\begin{exercise}[2000 B4]
Let $f(x)$ be a continuous function such that $f(2x^{2} - 1) = 2xf(x)$ for all $x$. Show that $f(x) = 0$ for $-1 \leq x \leq 1$. 
\end{exercise}

\pagebreak
\section{Hints}
\begin{enumerate}[label=\textbf{\arabic*.}]
    \item What is $\left(\sum_{i=1}^r M_i\right)^2$
    \item Relate the sides, area, and circumradius of the triangle formed by the points.
    \item Product of the eigenvalues.
    \item Divide the rectangle into two parts by the diagonal line $ay=bx$ and integrate over each part separately.
    \item Write out the polynomial in terms of its coefficients and calculate $f(f(n)+1)$ modulo $f(n)$.
    \item Use Taylor series!
    \item Find a polynomial identity with integer coefficients satisfied by the six distances. Obtain a contradiction modulo a power of $2$. 
    \item The trace is independent of choice of basis.
    \item Pair up the subsets of $T_n$, except for the sets $\{1\},\{2\},\dots,\{n\}$.
    \item Write out the polynomial in terms of its coefficients and note that every coefficient in $p^{(j)}(x)$ is divisible by $k!$.
    \item Let $S$ be the set of all $p$-tuples of elements $(a_0,a_1,\dots,a_{p-1})\in G^p$ satisfying $a_0a_1\cdots a_{p-1}=e$. Consider the orbits of $S$ under cyclic permutation.
    \item Define $T$ to be a linear operator on the space of continuous functions $[0,1]\to [0,1]$ by $(Th)(x)=\int_0^1 h(y)K(x,y)\ dy$ and let $r$ be the minimum value of $f/g$ (note that it exists and is attained). Find a contradiction by considering $T^2(f-rg)=T(T(f-rg))$.
    \item Show that $A^2+B^2$ times something nonzero is zero. There is only one nonzero matrix in the problem statement.
    \item A continuous function on a compact set attains a minimum value.
\end{enumerate}
\end{document}