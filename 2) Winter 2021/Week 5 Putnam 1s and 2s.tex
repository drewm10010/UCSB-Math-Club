\documentclass{article}
\usepackage[utf8]{inputenc}
\usepackage{hyperref}
\usepackage{pdfrender,xcolor}
\usepackage{graphicx}



\usepackage{amsmath,amssymb,amsthm}
\usepackage{mathtools,graphicx,tikz-cd}
\usepackage{blindtext}
\usepackage[margin = 1.25 in]{geometry}
\usepackage{enumitem}
\usepackage{physics}

\usepackage{fancyhdr,accents,lastpage}
\pagestyle{fancy}
\setlength{\headheight}{25pt}

\newtheorem{theorem}{Theorem}[section]
\newtheorem{corollary}{Corollary}[theorem]
\newtheorem{lemma}[theorem]{Lemma}

\theoremstyle{definition}
\newtheorem{definition}{Definition}[section]

\theoremstyle{remark}
\newtheorem*{remark}{Remark}
\newtheorem{exercise}{Exercise}
\newtheorem{example}{Example}[definition]


\DeclareMathOperator{\ab}{ab}
\DeclareMathOperator{\area}{area}
\DeclareMathOperator{\Aut}{Aut}
\DeclareMathOperator{\BGL}{BGL}
\DeclareMathOperator{\Br}{Br}
\DeclareMathOperator{\card}{card}
\DeclareMathOperator{\ch}{ch}
\DeclareMathOperator{\Char}{char}
\DeclareMathOperator{\CHur}{CHur}
\DeclareMathOperator{\Cl}{Cl}
\DeclareMathOperator{\coker}{coker}
\DeclareMathOperator{\Conf}{Conf}
\DeclareMathOperator{\disc}{disc}
\DeclareMathOperator{\End}{End}
\DeclareMathOperator{\et}{\text{\'et}}
\DeclareMathOperator{\Fix}{Fix}
\DeclareMathOperator{\Gal}{Gal}
\DeclareMathOperator{\GL}{GL}
\DeclareMathOperator{\Hom}{Hom}
\DeclareMathOperator{\Hur}{Hur}
\DeclareMathOperator{\im}{im}
\DeclareMathOperator{\Ind}{Ind}
\DeclareMathOperator{\Inn}{Inn}
\DeclareMathOperator{\Irr}{Irr}
\DeclareMathOperator{\lcm}{lcm}
\DeclareMathOperator{\Mor}{Mor}
\DeclareMathOperator{\ord}{ord}
\DeclareMathOperator{\Out}{Out}
\DeclareMathOperator{\Perm}{Perm}
\DeclareMathOperator{\PGL}{PGL}
\DeclareMathOperator{\Pin}{Pin}
\DeclareMathOperator{\PSL}{PSL}
\DeclareMathOperator{\rad}{rad}
\DeclareMathOperator{\sgn}{sgn}
\DeclareMathOperator{\SL}{SL}
\DeclareMathOperator{\SO}{SO}
\DeclareMathOperator{\Sp}{Sp}
\DeclareMathOperator{\Spec}{Spec}
\DeclareMathOperator{\Spin}{Spin}
\DeclareMathOperator{\St}{St}
\DeclareMathOperator{\Surj}{Surj}
\DeclareMathOperator{\Syl}{Syl}
\DeclareMathOperator{\tame}{tame}

\newcommand{\eps}{\varepsilon}
\newcommand{\QED}{\hspace{\stretch{1}} $\blacksquare$}
\renewcommand{\AA}{\mathbb{A}}
\newcommand{\CC}{\mathbb{C}}
\newcommand{\EE}{\mathbb{E}}
\newcommand{\FF}{\mathbb{F}}
\newcommand{\HH}{\mathbb{H}}
\newcommand{\NN}{\mathbb{N}}
\newcommand{\OO}{\mathbb{O}}
\newcommand{\PP}{\mathbb{P}}
\newcommand{\QQ}{\mathbb{Q}}
\newcommand{\RR}{\mathbb{R}}
\newcommand{\ZZ}{\mathbb{Z}}
\newcommand{\bfm}{\mathbf{m}}
\newcommand{\mcA}{\mathcal{A}}
\newcommand{\mcC}{\mathcal{C}}
\newcommand{\mcG}{\mathcal{G}}
\newcommand{\mcH}{\mathcal{H}}
\newcommand{\mcM}{\mathcal{M}}
\newcommand{\mcN}{\mathcal{N}}
\newcommand{\mcO}{\mathcal{O}}
\newcommand{\mcP}{\mathcal{P}}
\newcommand{\mcQ}{\mathcal{Q}}
\newcommand{\mfa}{\mathfrak{a}}
\newcommand{\mfb}{\mathfrak{b}}
\newcommand{\mfI}{\mathfrak{I}}
\newcommand{\mfM}{\mathfrak{M}}
\newcommand{\mfm}{\mathfrak{m}}
\newcommand{\mfo}{\mathfrak{o}}
\newcommand{\mfO}{\mathfrak{O}}
\newcommand{\mfP}{\mathfrak{P}}
\newcommand{\mfp}{\mathfrak{p}}
\newcommand{\mfq}{\mathfrak{q}}
\newcommand{\mfz}{\mathfrak{z}}
\newcommand{\AGL}{\mathbb{A}\GL}
\newcommand{\Qbar}{\overline{\QQ}}
\renewcommand{\qedsymbol}{$\blacksquare$}
\renewcommand{\d}[1]{\ensuremath{\operatorname{d}\!{#1}}}

\pdfrender{StrokeColor=black,TextRenderingMode=2,LineWidth=0.5pt}

\lhead{\pdfrender{StrokeColor=black,TextRenderingMode=2,LineWidth=0.5pt}Putnam 1s and 2s} 
\chead{\pdfrender{StrokeColor=black,TextRenderingMode=2,LineWidth=0.5pt}Math Club}
\rhead{\pdfrender{StrokeColor=black,TextRenderingMode=2,LineWidth=0.5pt}Winter 2021} 


\begin{document}

\section{Introduction}

    Welcome to math club!
    Looking a few weeks ahead to the Putnam exam (Feb 20), we will be spending some time preparing and practicing.
    Today, our focus will be on solving the ``easy'' problems on the Putnam, usually problems 1A, 1B, 2A, and 2B.
    Of course, these aren't {\it easy} per se, but they are certainly more accessible than the later problems.
    
    You may have seen some of these problems before.
    If that is the case, we encourage you to work on those that are unfamiliar to you.
    The main point of today is to show you how to get full points on a question and then to practice doing so.
    As a reminder, don't feel bad at all if you're unable to completely solve these.
    After all, the median score on the Putnam is somewhere in the range of 0, 1, or 2 (each question is worth 10 points).
    
    We will work through the following problem as an example.
    
    \begin{exercise}[2009 B1]
        Show that every positive rational number can be written as a quotient of products of factorials of (not necessarily distinct) primes.
        For example,
        \[\frac{10}{9} = \frac{2!\cdot 5!}{3!\cdot 3!\cdot 3!}.\]
\end{exercise}

\section{Resources}

    The Holy Grail of Putnam competition websites is \url{https://kskedlaya.org/putnam-archive/}, compiled by Kiran Kedlaya of UCSD.
    This site has lots of problems and solutions (as well as results) from many previous exams.
    
    John Scholes compiled a collection of solutions to all the Putnam exams up to 2003. A copy of his site can be found here: \url{https://prase.cz/kalva/putnam.html}. 
    
    Highly recommended book: \emph{The William Lowell Putnam Mathematical Competition 1985-2000: Problems, Solutions, and Commentary} by Kedlaya, Poonen, and Vakil. Several Putnam problems are presented along with hints, detailed solutions, discussions of concepts and related problems. This book was made specifically for Putnam preparation.
\section{Easy}

\begin{exercise}[1984 A1]
Let $A$ be a solid $a\times b\times c $ rectangular brick in three dimensions, where $a,b,c>0$. Let $B$ be the set of all points which are a distance at most one from some point of $A$ (in particular, $B$ contains $A$). Express the volume of $B$ as a polynomial in $a$, $b$, and $c$.
\end{exercise}

\begin{exercise}[1983 A2]
    The hands of an accurate clock have lengths $3$ and $4$. Find the distance between the tips of the hands when that distance is increasing most rapidly.
\end{exercise}

\begin{exercise}[1968 A1]
    Prove that
    \[\frac{22}{7}-\pi = \int_0^1\frac{x^4(1-x)^4}{1+x^2}\d x.\]
\end{exercise}

\begin{exercise}[2014 B1]
    A {\it base 10 over-expansion} of a positive integer \(N\) is an expression of the form
    \[N = d_k10^k+d_{k-1}10^{k-1}+\cdots+d_010^0\]
    with \(d_i\in\{0,1,2,\ldots,10\}\) for all \(i\).
    For instance, the integer \(N=10\) has two base 10 over-expansions: \(10=10\cdot 10^0\) and \(10=1\cdot 10^1+0\cdot 10^0\).
    Which positive integers have a unique base 10 over-expansion?
\end{exercise}

\begin{exercise}[1994 B1]
Find all positive integers that are within $250$ of exactly $15$ perfect squares.
\end{exercise}

\begin{exercise}[1994 B2]
For which real numbers $c$ is there a straight line that intersects $y=x^4+9x^3+cx^2+9x+4$ in four distinct points?
\end{exercise}

\section{Medium}

\begin{exercise}[2019 B1]
    Denote by \(\ZZ^2\) the set of all points \((x,y)\) in the plane with integer coordinates.
    For each integer \(n\geq 0\), let \(P_n\) be the subset of \(\ZZ^2\) consisting of the point \((0,0)\) together with all points \((x,y)\) such that \(x^2+y^2=2^k\) for some integer \(k\leq n\).
    Determine, as a function of \(n\), the number of four point subsets of \(P_n\) whose elements are vertices of a square.
\end{exercise}

\begin{exercise}[2000 A2]
    Prove that there exist infinitely many positive integers \(n\) such that \(n\), \(n+1\), and \(n+2\) are each the sum of the squares of two integers.
\end{exercise}

\begin{exercise}[2002 A1]
    Given any five points on a sphere, show that some four of them must lie on a closed hemisphere.
\end{exercise}

\begin{exercise}[2004 B2]
    Let \(m\) and \(n\) be positive integers.
    Show that
    \[\frac{(m+n)!}{(m+n)^{m+n}}<\frac{m!}{m^m}\frac{n!}{n^n}\]
\end{exercise}

\begin{exercise}[2006 B2]
    Prove that, for every set \(X=\{x_1,x_2,\ldots,x_n\}\) of \(n\) real numbers, there exists a nonempty subset \(S\) of \(X\) and an integer \(m\) such that
    \[\abs{m+\sum_{s\in S}s}\leq \frac{1}{n+1}.\]
\end{exercise}

\begin{exercise}[2008 A1]
    Let \(f:\RR^2\to\RR\) be a function such that \(f(x,y)+f(y,z)+f(z,x)=0\) for all real numbers \(x\), \(y\), and \(z\).
    Prove that there exists a function \(g:\RR\to\RR\) such that \(f(x,y)=g(x)-g(y)\) for all real numbers \(x\) and \(y\).
\end{exercise}

\section{Tough}

\begin{exercise}[2019 A1]
    Determine all possible values of the expression
    \[A^3+B^3+C^3-3ABC,\]
    where \(A\), \(B\), and \(C\) are nonnegative integers.
\end{exercise}

\begin{exercise}[2019 A2]
    In the triangle \(\triangle ABC\), let \(G\) be the centroid, and let \(I\) be the center of the inscribed circle.
    Let \(\alpha\) and \(\beta\) be the angles at the vertices \(A\) and \(B\), respectively.
    Suppose that the segment \(IG\) is parallel to \(AB\) and that \(\beta=2\tan^{-1}(1/3)\).
    Find \(\alpha\).
\end{exercise}

\begin{exercise}[2019 B2]
    For all \(n\geq 1\), let
    \[a_n = \sum_{k=1}^{n-1}\frac{\sin\left(\frac{(2k-1)\pi}{2n}\right)}{\cos^2\left(\frac{(k-1)\pi}{2n}\right)\cos^2\left(\frac{k\pi}{2n}\right)}.\]
    Determine $\lim_{n\to\infty}a_n /n^3$.
\end{exercise}

\begin{exercise}[1990 B2]
Prove that for all $|x|<1$, $|z|>1$, 
\[1+\sum_{j=1}^\infty (1+x^j)\frac{(1-z)(1-zx)(1-zx^2)\cdots(1-zx^{j-1})}{(z-x)(z-x^2)(z-x^3)\cdots(z-x^j)} = 0\]
\end{exercise}

\begin{exercise}[1992 A2]
Define $C(\alpha)$ to be the coefficient of $x^{1992}$ in the power series expansion about $x=0$ of $(1+x)^\alpha$. Evaluate
\[\int_0^1 C(-y-1)\left(\frac{1}{y+1}+\frac{1}{y+2}+\frac{1}{y+3}+\cdots + \frac{1}{y+1992}\right)\ dy\]
\end{exercise}

\end{document}