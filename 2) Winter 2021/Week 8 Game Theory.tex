\documentclass{article}
\usepackage[utf8]{inputenc}
\usepackage{hyperref}
\usepackage{pdfrender,xcolor}
\usepackage{graphicx}

\usepackage{amsmath,amssymb,amsthm}
\usepackage{mathtools,graphicx,tikz-cd}
\usepackage{blindtext}
\usepackage[margin = 1.25 in]{geometry}
\usepackage{enumitem}
\usepackage{physics}

\usepackage{fancyhdr,accents,lastpage}
\pagestyle{fancy}
\setlength{\headheight}{25pt}

\newtheorem{theorem}{Theorem}[section]
\newtheorem{corollary}{Corollary}[theorem]
\newtheorem{lemma}[theorem]{Lemma}

\theoremstyle{definition}
\newtheorem{definition}{Definition}[section]

\theoremstyle{remark}
\newtheorem*{remark}{Remark}
\newtheorem{exercise}{Exercise}
\newtheorem{example}{Example}[definition]


\DeclareMathOperator{\ab}{ab}
\DeclareMathOperator{\area}{area}
\DeclareMathOperator{\Aut}{Aut}
\DeclareMathOperator{\BGL}{BGL}
\DeclareMathOperator{\Br}{Br}
\DeclareMathOperator{\card}{card}
\DeclareMathOperator{\ch}{ch}
\DeclareMathOperator{\Char}{char}
\DeclareMathOperator{\CHur}{CHur}
\DeclareMathOperator{\Cl}{Cl}
\DeclareMathOperator{\coker}{coker}
\DeclareMathOperator{\Conf}{Conf}
\DeclareMathOperator{\disc}{disc}
\DeclareMathOperator{\End}{End}
\DeclareMathOperator{\et}{\text{\'et}}
\DeclareMathOperator{\Fix}{Fix}
\DeclareMathOperator{\Gal}{Gal}
\DeclareMathOperator{\GL}{GL}
\DeclareMathOperator{\Hom}{Hom}
\DeclareMathOperator{\Hur}{Hur}
\DeclareMathOperator{\im}{im}
\DeclareMathOperator{\Ind}{Ind}
\DeclareMathOperator{\Inn}{Inn}
\DeclareMathOperator{\Irr}{Irr}
\DeclareMathOperator{\lcm}{lcm}
\DeclareMathOperator{\Mor}{Mor}
\DeclareMathOperator{\ord}{ord}
\DeclareMathOperator{\Out}{Out}
\DeclareMathOperator{\Perm}{Perm}
\DeclareMathOperator{\PGL}{PGL}
\DeclareMathOperator{\Pin}{Pin}
\DeclareMathOperator{\PSL}{PSL}
\DeclareMathOperator{\rad}{rad}
\DeclareMathOperator{\sgn}{sgn}
\DeclareMathOperator{\SL}{SL}
\DeclareMathOperator{\SO}{SO}
\DeclareMathOperator{\Sp}{Sp}
\DeclareMathOperator{\Spec}{Spec}
\DeclareMathOperator{\Spin}{Spin}
\DeclareMathOperator{\St}{St}
\DeclareMathOperator{\Surj}{Surj}
\DeclareMathOperator{\Syl}{Syl}
\DeclareMathOperator{\tame}{tame}

\newcommand{\eps}{\varepsilon}
\newcommand{\QED}{\hspace{\stretch{1}} $\blacksquare$}
\renewcommand{\AA}{\mathbb{A}}
\newcommand{\CC}{\mathbb{C}}
\newcommand{\EE}{\mathbb{E}}
\newcommand{\FF}{\mathbb{F}}
\newcommand{\HH}{\mathbb{H}}
\newcommand{\NN}{\mathbb{N}}
\newcommand{\OO}{\mathbb{O}}
\newcommand{\PP}{\mathbb{P}}
\newcommand{\QQ}{\mathbb{Q}}
\newcommand{\RR}{\mathbb{R}}
\newcommand{\ZZ}{\mathbb{Z}}
\newcommand{\bfm}{\mathbf{m}}
\newcommand{\mcA}{\mathcal{A}}
\newcommand{\mcC}{\mathcal{C}}
\newcommand{\mcG}{\mathcal{G}}
\newcommand{\mcH}{\mathcal{H}}
\newcommand{\mcM}{\mathcal{M}}
\newcommand{\mcN}{\mathcal{N}}
\newcommand{\mcO}{\mathcal{O}}
\newcommand{\mcP}{\mathcal{P}}
\newcommand{\mcQ}{\mathcal{Q}}
\newcommand{\mfa}{\mathfrak{a}}
\newcommand{\mfb}{\mathfrak{b}}
\newcommand{\mfI}{\mathfrak{I}}
\newcommand{\mfM}{\mathfrak{M}}
\newcommand{\mfm}{\mathfrak{m}}
\newcommand{\mfo}{\mathfrak{o}}
\newcommand{\mfO}{\mathfrak{O}}
\newcommand{\mfP}{\mathfrak{P}}
\newcommand{\mfp}{\mathfrak{p}}
\newcommand{\mfq}{\mathfrak{q}}
\newcommand{\mfz}{\mathfrak{z}}
\newcommand{\AGL}{\mathbb{A}\GL}
\newcommand{\Qbar}{\overline{\QQ}}
\renewcommand{\qedsymbol}{$\blacksquare$}
\renewcommand{\d}[1]{\ensuremath{\operatorname{d}\!{#1}}}

\pdfrender{StrokeColor=black,TextRenderingMode=2,LineWidth=0.5pt}

\lhead{\pdfrender{StrokeColor=black,TextRenderingMode=2,LineWidth=0.5pt}Game Theory} 
\chead{\pdfrender{StrokeColor=black,TextRenderingMode=2,LineWidth=0.5pt}Math Club}
\rhead{\pdfrender{StrokeColor=black,TextRenderingMode=2,LineWidth=0.5pt}Winter 2021} 


\begin{document}

\section{Introduction}

    Game theory is the study of strategic interactions among rational people who need to make decisions. 
    The fate of every person in a ``game'' will usually depend on the decisions of all ``players''.
    This week, we will explore many different types of games, discuss possible strategies, and try some problems to help you start thinking about these interactions like a game theorist.
    Often, you'll be presented with a game and be asked to find an ``optimal strategy'' for both players.
    In some cases, this strategy can be easily found; in others, many layers of theory are needed to find such a solution.
    Some games haven't even been solved!
    Game theory is also applicable to fields that deal with people and the decisions they make.
    In particular, many of the best game theorists were both mathematicians and economists and used their game theory skills to analyze the economic behavior of people.
    Anyway, we hope you find these games interesting.

\section{Resources}

    Here are some good reference from which you can learn more about game theory:
    \begin{enumerate}
        \item {\it Winning Ways for Your Mathematical Plays}, by Berlekamp, Conway, and Guy
        \item {\it A Course in Game Theory}, by Martin Osborne
    \end{enumerate}

\section{Nim}

    The game of Nim starts with three heaps, each containing any number of pebbles (usually the piles contain 3, 4, and 5 pebbles, but these parameters can be changed).
    The two players alternate taking any number of pebbles from any single one of the heaps.
    You win if you are the last person to take a pebble.

    \begin{exercise}
        If there are two equal rows left in the game, who will win: the first or the second player?
        What course will the game take?
    \end{exercise}

    \begin{exercise}
        Consider Nim with three piles, one containing 6 pebbles, a second containing \(n\) pebbles, and the third containing \(6-n\) pebbles, where \(0\leq n\leq 6\).
        For which \(n\) is this a losing position, i.e., the first player loses?
        For which \(n\) is \(\binom{6}{n}\) odd?
        Is this a coincidence?
    \end{exercise}

    \begin{exercise}
        Determine which positions in Nim are winning positions.
        What is the winning strategy?
    \end{exercise}

\section{Chomp}

    In Chomp, you begin with a \(4\times 6\) (or in general \(n\times m\)) chocolate bar, and players alternate turns by chomping a square out of the chocolate bar, along with any squares that are to the right and above.
    However, the square in the lower left is poisoned!
    The player forced to chomp it loses.

    You can play against a bot here: \url{https://www.math.ucla.edu/~tom/Games/chomp.html}.
    You can play against another person controlled by you here: \url{https://www.geogebra.org/m/HSwgnXjn}.

    \begin{figure}[hbt!]
        \small
        \centering
        \includegraphics[scale = 0.4]{Pics/chomp.png}
    \end{figure}

    \begin{exercise}
        Determine which player wins as well as the winning strategy.
    \end{exercise}

\section{Sim}

    In Sim, six dots are drawn.
    Each dot is connected to every other dot by a line.
    Gameplay proceeds as follows.
    Players take turns coloring any uncolored lines.
    One player uses red, while another uses blue.
    Each player is trying to avoid creating a triangle made solely from their color; the player who completes such a triangle loses instantly.
    Also, only triangles whose vertices are among the six original dots count; intersections of lines don't matter.

    You can play the game here: \url{https://wideaperture.net/sim/}.

    \begin{figure}[hbt!]
        \small
        \centering
        \includegraphics[scale = 0.4]{Pics/sim.png}
    \end{figure}

    \begin{exercise}
        Play the game and explore its properties.
        There are many connections between this game and Ramsey theory.
    \end{exercise}

\section{Miscellaneous}

    \begin{exercise}
        In a game of tic-tac-toe, suppose that the first player plays in the corner, and the second player does NOT play in the center.
        Prove that the first player can force a win.
    \end{exercise}

    \begin{exercise}[Bachet's Game]
        Initially there are \(n>0\) checkers on the table.
        The legal moves consist of removing at least one but not more than \(k<n\) checkers from the table.
        The winner is the one to take the last checker.
        For what values of \(n\) will the first person have the winning strategy?
        How about the second person?
        What are the losing positions?
    \end{exercise}

    \begin{exercise}
        Consider Bachet's Game again, but this time the legal moves consist of removing any power of 2 checkers.
        What is the winning strategy for the first and second player this time?
        What are the losing positions?
    \end{exercise}

    \begin{exercise}[2020 B2]
        Let \(k,n\in\ZZ\) with \(1\leq k<n\).
        Alice and Bob play a game with \(k\) pegs in a line of \(n\) holes.
        At the beginning of the game, the pegs occupy the \(k\) leftmost holes.
        A legal move consists of moving a single peg to any vacant hole that is further to the right.
        The players alternate moves, with Alice playing first.
        The game ends when the pegs are in the \(k\) rightmost holes, so whoever is next to play cannot move and therefore loses.
        For what values of \(n\) and \(k\) does Alice have a winning strategy?
    \end{exercise}

\end{document}