\documentclass{article}
\usepackage[utf8]{inputenc}
\usepackage{hyperref}
\usepackage{pdfrender,xcolor}
\usepackage{graphicx}

\usepackage{amsmath,amssymb,amsthm}
\usepackage{mathtools,graphicx,tikz-cd}
\usepackage{blindtext}
\usepackage[margin = 1.25 in]{geometry}
\usepackage{enumitem}

\usepackage{fancyhdr,accents,lastpage}
\pagestyle{fancy}
\setlength{\headheight}{25pt}

\newtheorem{theorem}{Theorem}[section]
\newtheorem{corollary}{Corollary}[theorem]
\newtheorem{lemma}[theorem]{Lemma}

\theoremstyle{definition}
\newtheorem{definition}{Definition}[section]

\theoremstyle{remark}
\newtheorem*{remark}{Remark}
\newtheorem{exercise}{Exercise}
\newtheorem{example}{Example}[definition]

\DeclareMathOperator{\ab}{ab}
\DeclareMathOperator{\Aut}{Aut}
\DeclareMathOperator{\BGL}{BGL}
\DeclareMathOperator{\Br}{Br}
\DeclareMathOperator{\card}{card}
\DeclareMathOperator{\ch}{ch}
\DeclareMathOperator{\Char}{char}
\DeclareMathOperator{\CHur}{CHur}
\DeclareMathOperator{\Cl}{Cl}
\DeclareMathOperator{\coker}{coker}
\DeclareMathOperator{\Conf}{Conf}
\DeclareMathOperator{\disc}{disc}
\DeclareMathOperator{\End}{End}
\DeclareMathOperator{\et}{\text{\'et}}
\DeclareMathOperator{\Fix}{Fix}
\DeclareMathOperator{\Gal}{Gal}
\DeclareMathOperator{\GL}{GL}
\DeclareMathOperator{\Hom}{Hom}
\DeclareMathOperator{\Hur}{Hur}
\DeclareMathOperator{\im}{im}
\DeclareMathOperator{\Ind}{Ind}
\DeclareMathOperator{\Inn}{Inn}
\DeclareMathOperator{\Irr}{Irr}
\DeclareMathOperator{\lcm}{lcm}
\DeclareMathOperator{\Mor}{Mor}
\DeclareMathOperator{\ord}{ord}
\DeclareMathOperator{\Out}{Out}
\DeclareMathOperator{\Perm}{Perm}
\DeclareMathOperator{\PGL}{PGL}
\DeclareMathOperator{\Pin}{Pin}
\DeclareMathOperator{\PSL}{PSL}
\DeclareMathOperator{\rad}{rad}
\DeclareMathOperator{\sgn}{sgn}
\DeclareMathOperator{\SL}{SL}
\DeclareMathOperator{\SO}{SO}
\DeclareMathOperator{\Sp}{Sp}
\DeclareMathOperator{\Spec}{Spec}
\DeclareMathOperator{\Spin}{Spin}
\DeclareMathOperator{\St}{St}
\DeclareMathOperator{\Surj}{Surj}
\DeclareMathOperator{\Syl}{Syl}
\DeclareMathOperator{\tame}{tame}
\DeclareMathOperator{\Tr}{Tr}

\newcommand{\eps}{\varepsilon}
\newcommand{\QED}{\hspace{\stretch{1}} $\blacksquare$}
\renewcommand{\AA}{\mathbb{A}}
\newcommand{\CC}{\mathbb{C}}
\newcommand{\EE}{\mathbb{E}}
\newcommand{\FF}{\mathbb{F}}
\newcommand{\HH}{\mathbb{H}}
\newcommand{\NN}{\mathbb{N}}
\newcommand{\OO}{\mathbb{O}}
\newcommand{\PP}{\mathbb{P}}
\newcommand{\QQ}{\mathbb{Q}}
\newcommand{\RR}{\mathbb{R}}
\newcommand{\ZZ}{\mathbb{Z}}
\newcommand{\bfm}{\mathbf{m}}
\newcommand{\mcA}{\mathcal{A}}
\newcommand{\mcC}{\mathcal{C}}
\newcommand{\mcG}{\mathcal{G}}
\newcommand{\mcH}{\mathcal{H}}
\newcommand{\mcM}{\mathcal{M}}
\newcommand{\mcN}{\mathcal{N}}
\newcommand{\mcO}{\mathcal{O}}
\newcommand{\mcP}{\mathcal{P}}
\newcommand{\mcQ}{\mathcal{Q}}
\newcommand{\mfa}{\mathfrak{a}}
\newcommand{\mfb}{\mathfrak{b}}
\newcommand{\mfI}{\mathfrak{I}}
\newcommand{\mfM}{\mathfrak{M}}
\newcommand{\mfm}{\mathfrak{m}}
\newcommand{\mfo}{\mathfrak{o}}
\newcommand{\mfO}{\mathfrak{O}}
\newcommand{\mfP}{\mathfrak{P}}
\newcommand{\mfp}{\mathfrak{p}}
\newcommand{\mfq}{\mathfrak{q}}
\newcommand{\mfz}{\mathfrak{z}}
\newcommand{\AGL}{\mathbb{A}\GL}
\newcommand{\Qbar}{\overline{\QQ}}
\renewcommand{\qedsymbol}{$\blacksquare$}

%\pdfrender{StrokeColor=black,TextRenderingMode=2,LineWidth=0.5pt}

%\lhead{\pdfrender{StrokeColor=black,TextRenderingMode=2,LineWidth=0.5pt}Probability} 
%\chead{\pdfrender{StrokeColor=black,TextRenderingMode=2,LineWidth=0.5pt}Math Club}
%\rhead{\pdfrender{StrokeColor=black,TextRenderingMode=2,LineWidth=0.5pt}Winter 2021}
\lhead{Probability}
\chead{Math Club}
\rhead{Winter 2021}

\begin{document}

\section{Introduction and History Lesson}

    Welcome to week 3 of Math Club!
    A \textbf{probability} is a value between 0 and 1 that represents the likelyhood of a certain event happening.
    We write the probability of event \(A\) happening as \(P(A)\).
    Convince yourself that the probability of \(A\) not happening is \(1-P(A)\).
    Probability theory is important in problem-solving, but it's even more important in the real world because it gives us tools to quantify problems that have uncertain results.

    Now for a brief history lesson: Chevalier de M\'er\'e, a gambler of the \(17^{\text{th}}\) century, observed while gambling that the odds of getting a six while rolling a die four times seem to be greater than \(\frac12\), while the odds of getting a double six when rolling two dice 24 times seem to be less than \(\frac12\).
    De M\'er\'e thought that this contradicted mathematics itself because \(\frac46=\frac{24}{36}\).
    He posed the question the Blaise Pascal and Pierre de Fermat.
    They answered the question and probability theory was born!
    Let us work through what Pascal and Fermat told de M\'er\'e.

    \begin{definition}
        Let \(A\) and \(B\) be events with \(P(B)\neq 0\).
        The \textbf{conditional probability} of \(A\) given \(B\) is
        \[P(A\mid B)=\frac{P(A\cap B)}{P(B)},\]
        where \(A\cap B\) represents both \(A\) and \(B\) happening.
        \(A\cup B\) represents either \(A\) or \(B\) (inclusive) happening.
    \end{definition}

    \begin{definition}
        Informally, two events \(A\) and \(B\) are called \textbf{independent} if they do not affect each other.
        More formally, \(A\) and \(B\) are independent if \(P(A\cap B)=P(A)P(B)\).
    \end{definition}

    \begin{theorem}
        Here are two classic and important formulas:
        \begin{itemize}
            \item Addition formula:
            \[P(A\cup B)=P(A)+P(B)-P(A\cap B)\]
            \item Multiplication formula:
            \[P(A\cap B)=P(A)P(B\mid A)\]
        \end{itemize}
    \end{theorem}

    \begin{theorem}[Bayes' Theorem]
        Let \(A\) and \(B\) be events with \(P(B)\neq 0\).
        Then
        \[P(A\mid B)=\frac{P(B\mid A)P(A)}{P(B)}.\]
        Bayes' theorem gives us a useful tool for reframing questions of conditional probability into problems that are more easily solved.
    \end{theorem}

\section{Resources}

    These problems were taken from a few different sources, including:
    \begin{itemize}
        \item Putnam and Beyond by Andreescu and Gelca
        \item \href{http://www.math.utoronto.ca/barbeau/putnamprob.pdf}{Ed Barbeau's website at UToronto}
    \end{itemize}

\section{Easy(ish)}

    \begin{exercise}[Monty Hall Problem]
        Suppose you're on a game show, and you're given the choice of three doors: Behind one door is a car; behind the others, goats. 
        You pick a door, say No. 1, and the host, who knows what's behind the doors, opens another door, say No. 3, which has a goat. 
        He then says to you, ``Do you want to pick door No. 2?" Is it to your advantage to switch your choice?
    \end{exercise}

    \begin{exercise}[Birthday Problem]
        Assume there are 365 days in the year and the chance of being born on a given day is the same for each day.
        What is the probability that in a room of 23 people, some pair of people have the same birthday?
        Feel free to use a calculator for this one.
        More generally, find the same probability for a room of \(n\) people.
    \end{exercise}

    \begin{exercise}[2014 AIME, Problem 2]
        An urn contains 4 green balls and 6 blue balls.
        A second urn contains 16 green balls and \(N\) blue balls.
        A single ball is drawn at random from each urn.
        The probability that both balls are of the same color is \(0.58\).
        Find \(N\).
    \end{exercise}

    \begin{exercise}
        One hundred people line up to board an airplane. 
        Each has a boarding pass with assigned seat. 
        However, the first person to board has lost his boarding pass and takes a random seat. 
        After that, each person takes the assigned seat if it is unoccupied, and one of unoccupied seats at random otherwise. 
        What is the probability that the last person to board gets to sit in his assigned seat?
    \end{exercise}

    \begin{exercise}
        Emmy writes down fifteen 1's in a row and randomly writes \(+\) or \(-\) between each pair of consecutive 1's.
        One such example is
        \[1+1+1-1-1+1-1+1-1+1-1-1-1+1+1.\]
        What is the probability that the value of the expression Emmy wrote down is 7?
    \end{exercise}

\section{Medium}

    \begin{exercise}
        The temperatures in Chicago and Detroit are \(x^\circ\) and \(y^\circ\), respectively.
        These temperatures are not assumed independent; namely we are given the following:
        \begin{enumerate}
            \item \(P(x^\circ=70^\circ)=a\), the probability that the temperature in Chicago is \(70^\circ\),
            \item \(P(y^\circ=70^\circ)=b\), and
            \item \(P(\max(x^\circ,y^\circ)=70^\circ)=c\).
        \end{enumerate}
        Determine \(P(\min(x^\circ,y^\circ)=70^\circ)\) in terms of \(a\), \(b\), and \(c\).
    \end{exercise}

    \begin{exercise}
        Mr. Knuth works on the \(13^{\text{th}}\) floor of a 15-floor building.
        The only elevator moves continuously through floors 1, 2,\dots, 15, 14,\dots, 2, 1, 2,\dots, except that it stops on a floor on which the button has been pressed.
        Assume the time spent loading and unloading passengers is negligible.

        Mr. Knuth complains that at 5pm, when he wants to go home, the elevator almost always goes up when it stops on his floor.
        What is the explanation for this?

        Now assume that the building has \(n\) elevators which move independently as previously described.
        Compute the proportion of time the first elevator on Mr. Knuth's floor moves up.
    \end{exercise}

    \begin{exercise}
        Prove the identity
        \[1+\frac{n}{m+n-1}+\cdots+\frac{n(n-1)\cdots 1}{(m+n-1)(m+n-2)\cdots m}=\frac{m+n}{m}.\]
        (ideally using probabilistic methods)
    \end{exercise}

\section{Putnam Practice}

    This section simulates one block of a Putnam exam, in which you are presented with 6 problems (labeled 1-6) and you are given 3 hours to complete them. 
    Of course, these are all probability questions, which would not happen on the actual exam.

    \begin{exercise}[2002 B1]
        Shanille O'Keal shoots free throws on a basketball court.
        She hits the first and misses the second, and thereafter the probability that she hits the next shot is equal to the proportion of shots she has hit so far.
        What is the probability that she hits exactly 50 of her first 100 shots?
    \end{exercise}

    \begin{exercise}[2001 A2]
        You have coins \(C_1,\ldots,C_n\).
        For each \(k\), coin \(C_k\) is biased so that, when tossed, it has probability \(1/(2k+1)\) of falling heads.
        If the \(n\) coins are tossed, what is the probability that the number of heads is odd?
        Express your answer as a rational function of \(n\).
    \end{exercise}

    \begin{exercise}[1993 B3]
        Two real numbers \(x\) and \(y\) are chosen at random in the interval \((0,1)\) with respect to the uniform distribution.
        What is the probability that the closest integer to \(x/y\) is even? 
        Express your answer in the form \(r+s\pi\), where \(r,s\in\QQ\).
    \end{exercise}

    \begin{exercise}[2016 B4]
        Let \(A\) be a \(2n\times 2n\) matrix with entries chosen at random.
        Each entry is chosen to be 0 or 1, each with probability \(1/2\).
        Find the expected value of \(\det(A-A^T)\) as a function of \(n\), where \(A^T\) denoted the transpose of \(A\).
    \end{exercise}

    \begin{exercise}[2004 A5]
        An \(m\times n\) checkerboard is colored randomly: each square is independently assigned red or black with probability \(1/2\).
        We say that two squares \(p\) and \(q\) are in the same connected monochromatic region if there is a sequence of squares, all of the same color, starting at \(p\) and ending at \(q\), in which successive squares in the sequence share a common side.
        SHow that the expected number of monochromatic regions is greater than \(mn/8\).
    \end{exercise}

    \begin{exercise}[2011 A6]
        Let \(G\) be an abelian group with \(n\) elements, and let
        \[\{g_1=e,g_2,\ldots,g_k\}\subseteq G\]
        be a (not necessarily minimal) set of distinct generators of \(G\).
        A special die, which randomly selects one of the elements \(g_1,\ldots,g_k\) with equal probability, is rolled \(m\) times and the selected elements are multiplied to produce an element \(g\in G\).

        Prove that there exists a real number \(b\in(0,1)\) such that
        \[\lim_{m\to\infty}\frac{1}{b^{2m}}\left(\text{Prob}(g=x)-\frac{1}{n}\right)^2\]
        is positive and finite.
    \end{exercise}

\end{document}