\documentclass{article}
\usepackage{../mathclub}

\title{Introduction to Math Club}
\author{UCSB}
\date{September 26, 2022}

\begin{document}

\section{Introduction}

Welcome to UCSB's Math Club! 
We are excited to have you here. 
In Math Club, we aim to improve our problem-solving skills, learn some new mathematics, and connect with a community of like-minded individuals.

Typically, in the fall quarter we work on some general problem-solving skills that you may find generally useful in your classes (even the non-math ones). 
For those who are interested, we also offer preparation for the Putnam competition, but we will cover this in more detail in the future.
For today, we want to get acquainted with you and for you to get to know your peers while working on some fun and interesting problems.

If you have not done so, please join our Slack channel at \href{UCSBmathclub.slack.com}{UCSBmathclub.slack.com}. 
This will serve as our main means of communication for the club.
Also, if you know the solution to one or more of these problems already, please don't spoil the answer for your neighbors.
Without further ado, let's solve some problems!

\section{Problems}

\begin{exercise}[The Pizza Problem]
    Suppose you have a pizza with 6 slices.
    Starting anywhere and label counterclockwise the slices 1 through 6.
    Place a pepperoni on slices 1 and 3.
    The rules of pizza-making are simple: as you add more pepperoni after the first two, you must only place pepperoni 2 at a time and on adjacent slices. 
    For example, you may place on slice 1 and 2, 6 and 1, etc.
    Can you make it such that each slice has the same number of pepperoni? 
    If so, how?
    If not, why not?
\end{exercise}

\begin{exercise}[Scale Part I]
    A shopkeeper wants to be able to dispense sugar in whole pounds ranging from one pound up to 40 pounds.
    He has a standard, equal-arm balance weigh scale.
    Being of an extremely economical outlook, he wants to use the least number of weights to enable him to weigh any number of pounds between 1 and 40.
    How many weights does he need and what are they?
\end{exercise}

\begin{exercise}[Scale Part II]
    You have an equal-arm balance scale and twelve solid balls. 
    You are told that one of the balls has a different weight from all the others, but you do not know whether it is lighter or heavier. 
    You can weigh the balls against each other in the scale balance. 
    Can you find the odd ball and tell if it is lighter or heavier in only three weighings?
\end{exercise}

\begin{exercise}
    Two missiles speed directly toward each other, one at 9,000 miles per hour and the other at 21,000 miles per hour. 
    They start at 4,857 miles apart. 
    Without using pencil and paper (or similar tools), calculate how far apart they are one minute before they collide.
\end{exercise}

\begin{exercise}
    Cover the plane with non-overlapping squares such that only two of them are the same size.
\end{exercise}

\begin{exercise}
    All the students in a school are arranged in a rectangular array. 
    After that, the tallest student in each row was chosen, and then among these Rufus T. Barleysheath happened to be the shortest. 
    Then, in each column, the shortest student was chosen, and Ogbert the Nerd was the tallest of these. 
    Who is taller: Rufus or Ogbert?
\end{exercise}

\begin{exercise}[Green-Eyed Dragons]
    You visit a remote desert island inhabited by one hundred very friendly dragons, all of whom have green eyes. 
    They haven't seen a human for many centuries and are very excited about your visit. They show you around their island and tell you all about their dragon way of life (dragons can talk, of course).
    
    They seem to be quite normal, as far as dragons go, but then you find out something rather odd. They have a rule on the island that states that if a dragon ever finds out that they has green eyes, then at precisely midnight at the end of the day of this discovery, they must relinquish all dragon powers and transform into a long-tailed sparrow. 
    However, there are no mirrors on the island, and the dragons never talk about eye color, so they have been living in blissful ignorance throughout the ages.

    Upon your departure, all the dragons get together to see you off, and in a tearful farewell you thank them for being such hospitable dragons. 
    You then decide to tell them something that they all already know (for each can see the colors of the eyes of all the other dragons): You tell them all that at least one of them has green eyes. Then you leave, not thinking of the consequences (if any). 
    Assuming that the dragons are (of course) infallibly logical, what happens? 
    If something interesting does happen, what exactly is the new information you gave the dragons?
\end{exercise}

\begin{exercise}
    The governing body of the state of Lateralia was extremely concerned about the uneven distribution of wealth in the country. 
    They thought it unfair that the richest man should have so much more than his poorer compatriots. 
    They therefore instituted a wealth tax decreeing that each year the wealthiest man in the country had to give away his money by doubling the wealth of every other citizen, starting with the poorest and working up to the second wealthiest person if possible. 
    The decree was carried out, and the richest person gave away his money by doubling the wealth of all other citizens.
    However, the governing body was shocked to find that this action had made no difference to the overall distribution of wealth nor to the relative wealth of the poorest and richest citizens. 
    How could this be so?
\end{exercise}

\begin{exercise}
    Mr. Smith planned to drive from Chicago to Detroit, then back again. 
    He wanted to average 60 miles an hour for the entire round trip. 
    After arriving in Detroit, he found that his average speed for the trip was only 30 miles an hour. 
    What must Smith's average speed be on the return trip in order to raise his average for the round trip to 60 miles an hour? 
    Recall that average speed is defined by the total distance travelled divided by the total time taken.
\end{exercise}

\begin{exercise}
    Find the minimum value of \((u-v)^2+\left(\sqrt{2-u^2}-\frac{9}{v}\right)^2\) for \(0<u<\sqrt{2}\) and \(v>0\).
\end{exercise}

\begin{exercise}[Choo Choo]
    \begin{enumerate}
        \item[(a)] A train starts in the station that has an infinite supply of fuel.
        A train can carry 500 units of fuel at a time.
        For every mile forwards or backwards, 1 unit of fuel is burned.
        The train can drop any amount of fuel on the track and can later pick up from this stash if it is on the same mile distance where the fuel was dropped. 
        What is the minimum amount of fuel required to reach a city located 800 miles away from the station?
        \item[(b)] A train starts in the station that has 1200 units of fuel. 
        The train can carry at most 300 units of fuel at a time. 
        For every mile forward or backwards, 1 unit of fuel is used. 
        The train can drop any amount of fuel on the track and can later pick up from this stash if it is on the same mile distance where the fuel was dropped. 
        How far can the train go?
    \end{enumerate}
\end{exercise}

\begin{exercise}
    \begin{enumerate}
        \item[(a)] Is it possible to arrange the numbers \(1,2,\ldots,16\) such that all adjacent pairs sum to a perfect square?
        \item[(b)] Is it possible to arrange the integers \(1,2\ldots,9\) such that all adjacent pairs sum to a prime number?
    \end{enumerate}
\end{exercise}

\begin{exercise}
    Define $f(x)=1/(1-x)$ and denote $r$ iterations of the function $f$ by $f^r$, so 
    \[f^r(x)=\underbrace{f(f(\cdots(f(x))\cdots))}_{r\text{ $f$'s}}\] 
    Determine $f^{1999}(2000)$.
\end{exercise}

\begin{exercise}
    Let $S$ be a finite set of at least two points in the plane. Assume that no three points of $S$ are colinear. 
    A windmill is a process that starts with a line $\ell$ going through a single point $P \in S$. 
    The line rotates clockwise about the pivot $P$ until the first time that the line meets some other point $Q$ belonging to $S$. 
    This point $Q$ takes over as the new pivot, and the line now rotates clockwise about $Q$, until it next meets a point of $S$.
    This process continues indefinitely. 
    Show that we can choose a point $P$ in $S$ and a line $\ell$ going through $P$ such that the resulting windmill uses each point of $S$ as a pivot infinitely many times.
\end{exercise}

\end{document}
