\documentclass{article}
\usepackage{../mathclub}

\title{Calculus}
\author{UCSB}
\date{October 24, 2022}

\begin{document}


\section{Introduction}
Welcome to the Week 5 Math Club meeting! Where did all the time go?! Today we'll be looking at CALCULUS, everyone's favorite subject
in high school. More accurately, we'll be giving a brief introduction to the more rigorous side of calculus. In other words, today 
we will be working on Analysis. Enjoy!

We provide some important theorems below.
You have most likely encountered all of these in your calculus class, but they bear repeating:
\begin{itemize}
    \item Intermediate Value Theorem: if $f$ is continuous on $[a,b]$ then every value between $f(a)$ and $f(b)$ is of the form $f(c) $ for some $c\in [a,b]$.
    \item Extreme Value Theorem: if $f$ is continuous on $[a,b]$, there exists $y\in [a,b]$ such that $f(y)\leq f(x)$ for all $x\in [a,b]$.
    \item Mean Value Theorem: if $f$ is continuous on $[a,b]$ and differentiable on $(a,b)$, then there exists $u$ with $a < u < b$ such that $f'(u)=\frac{f(b)-f(a)}{b-a}$.
    \item Fundamental Theorem of Calculus: if $f$ is continuous at $c$, the function $F(x) = \int_a^x f(t)\ dt$ is differentiable at $c$ and $F'(c)=f(c)$.
\end{itemize}
    
\section{Problems}
\begin{exercise}
Recall the formula for integration by parts:
\[\int f\ \mathrm{d}g = fg - \int g\ \mathrm{d}f.\]
Applying this to \(f(x)=1/x\) and \(g(x)=x\), we get
\[\int\frac{1}{x}\mathrm{d}x = 1 + \int\frac{1}{x}\mathrm{d}x\]
from which we enthusiastically conclude that \(0=1\).
Why are we wrong? Or is math as we know it wrong?
\end{exercise}

\begin{exercise}[A fun little integral]
        Compute
        \[\int_2^4 \frac{\log\sqrt{9-x}}{\log\sqrt{9-x}+\log\sqrt{x+3}}\ dx\] where $\log$ denotes the natural logarithm.
\end{exercise}
        
\begin{exercise}
Suppose \(f:\RR\to\RR\) is differentiable and defined by
\[f(x) = f'(2)x^2 + x.\]
Find the value \(f(2)\).
\end{exercise}

\begin{exercise}
What is the 100th derivative of \(f(x)=e^x\cos(x)\) evaluated at \(x=\pi\)?
\end{exercise}

\begin{exercise}[Fundamental Theorem of Calculus]
Find all real-valued continuously differentiable functions on the real line such that for all $x$
\[(f(x))^2 = \int_0^x \left((f(t))^2 + (f'(t))^2\right)\ \mathrm{d}t + 1990\]
\end{exercise}

\begin{exercise}[Extreme Value Theorem]
Suppose $f$ is continuous on $[a,b]$, and assume $f(x)>0$ for all $a\leq x \leq b$. Prove that there is a positive constant $c$ for which $c\leq f(x)$ for all $x\in [a,b]$.
\end{exercise}

\begin{exercise}[Intermediate Value Theorem]
Suppose $g:[0,1]\to [0,1]$ is a continuous function. Prove that $g$ has a fixed point in $[0,1]$, i.e., some $x\in [0,1]$ such that $g(x)=x$.
\end{exercise}

\begin{exercise} Let $f(x)=a_1\sin{x}+a_2\sin{2x}+\cdots + a_n\sin{nx}$, where $a_1,a_2,\dots,a_n$ are real numbers and $n$ is a positive integer. Given that             $|f(x)|\leq |\sin{x}|$ for all $x$, prove that $|a_1+2a_2+\cdots+na_n|<1$.
\end{exercise}

\begin{exercise}[1968 A1]
    Prove that
    \[\frac{22}{7}-\pi = \int_0^1\frac{x^4(1-x)^4}{1+x^2}\ \mathrm{d}x.\]
\end{exercise}

\begin{exercise}
            The integral
            \[\int_0^{\pi/2}\frac{x}{\tan(x)}\mathrm{d}x\]
            can be written in the form \(a^b\pi\ln(c)\), where \(a,b,c\in\ZZ\) and \(c\) is as small as possible.
            Compute \(a+b+c\).
        \end{exercise}

\begin{exercise}
            A very tired audience of 9001 attends a concert of Haydn's Surprise Symphony, which lasts 20 minutes.
            Members of the audience fall asleep at a continuous rate of \(6t\) people per minute, where \(t\) is the time in minutes since the symphony has begun.
            The Surprise Symphony is named so because when \(t=8\) minutes, the orchestra plays exactly one very loud note, waking everyone in the audience up.
            After that note, though, the audience continues to fall asleep at the same rate as before.
            Once a member of the audience falls asleep, they will stay asleep except for the rude awakening at \(t=8\) minutes.
            How many collective minutes does the audience sleep during the symphony?
        \end{exercise}

\begin{exercise}
            Suppose \(f(x) = e^{ax} + e^{bx}\) where \(a\neq b\) and that \(f''(x)-2f'(x)-15=0\) for all \(x\).
            Give all possible ordered pairs \((a,b)\).
        \end{exercise}

        \begin{exercise}
            Show that $4ax^3 + 3bx^2 + 2cx = a + b + c$ has at least one root between 0 and 1.
        \end{exercise}

\begin{exercise}
           Suppose $f$ is differentiable on $(-\infty, \infty)$ and that there is some constant $k<1$ for which $|f'(x)|\leq k$ for all real $x$. Prove that $f$ has fixed point. 
\end{exercise}

\begin{exercise}[2000 B4]
    Let $f(x)$ be a continuous function such that $f(2x^{2} - 1) = 2xf(x)$ for all $x$. Show that $f(x) = 0$ for $-1 \leq x \leq 1$. 
\end{exercise}

\begin{exercise}[2019 A6]
Let $g$ be a real-valued function that is continuous on $[0,1]$ and twice differentiable on the open interval $(0,1)$. Suppose that for some real number $r>1$, 
\[\lim_{x\to 0^+} \frac{g(x)}{x^r}=0\] Prove that either 
\[\lim_{x\to 0^+} g(x)=0\quad\textrm{or}\quad\limsup_{x\to 0^+}x^r|g''(x)| = +\infty\]
\end{exercise}

\end{document}