\documentclass{article}

\usepackage{../mathclub}

\title{Arithmetic Functions} 
\author{}
\date{April 18, 2022}

\begin{document}

Welcome to week 4 of Math Club!
Today we will be talking about some functions that tend to appear when studying analytic number theory.

\section{Arithmetic Functions}

These are functions \(f(n)\) which express some number-theoretic fact about \(n\).
We provide some examples of such functions
\begin{enumerate}
    \item \(\nu(n) =\) the number of distinct prime divisors of \(n\).
    \item \(\Omega(n) =\) the number of prime divisors of \(n\) counted with multiplcity (e.g., \(\Omega(4)=2\)).
    \item The \(s\)-fold divisor function \(\sigma_s(n) = \sum_{d\mid n}d^s\) for \(s\in\CC\). If \(s=0\), write \(\sigma_s(n)=d(n)\), the number of divisors of \(n\).
    \item The M\"obius function 
    \[\mu(n) = \begin{cases}
        (-1)^{\nu(n)} & \textrm{if }n\textrm{ is squarefree}\\
        0 & \textrm{otherwise,}
    \end{cases}\]
    along with \(\mu(1)=1\).
    \item Euler's totient function, or \(\phi\)-function
    \[\phi(n) = n\prod_{p\mid n}\left(1-\frac{1}{p}\right).\]
    \item The von Mangoldt function
    \[\Lambda(n) = \begin{cases}
        \log p & \textrm{if }n=p^\alpha\textrm{ for some }\alpha\geq 1\textrm{ and }p\textrm{ prime}\\
        0 & \textrm{otherwise.}
    \end{cases}\]
\end{enumerate}

\section{Dirichlet Series}

If \(f\) is an arithmetic function, the formal Dirichlet series attached to \(f\) is defined by
\[D(f,s) = \sum_{n=1}^\infty f(n)n^{-s},\]
where \(s\in\CC\).
We can define the sum of two Dirichlet series as
\[D(f,s) + D(g,s) = \sum_{n=1}^\infty\left(f(n)+g(n)\right)n^{-s}\]
and the product as
\[D(f,s)D(g,s) = \sum_{n=1}^\infty h(n)n^{-s},\quad\textrm{where}\quad h(n) = \sum_{de=n}f(d)g(e).\]
We sometimes write \(h=f*g\).
It is also helpful to write \(\delta(n)=1\) if \(n=1\) and \(\delta(n)=0\) if \(n\neq 1\), so that \(D(\delta,s)=1\).

\section{Exercises}

For each of these exercises, you may use the results from the previous exercises.

\begin{exercise}
    Prove that
    \[\sum_{d\mid n}\mu(d)=\begin{cases}
        1 & \textrm{if }n=1\\
        0 & \textrm{otherwise}
    \end{cases}\]
\end{exercise}

\begin{exercise}[M\"obius Inversion]
    Show that
    \[f(n) = \sum_{d\mid n}g(d)\qquad \forall n\in\NN\]
    if and only if
    \[g(n) = \sum_{d\mid n}\mu(d)f(n/d)\qquad \forall n\in\NN.\]
\end{exercise}

\begin{exercise}
    Show that
    \[\sum_{d\mid n}\phi(d) = n.\]
\end{exercise}

\begin{exercise}
    Show that 
    \[\sum_{d\mid n}\Lambda(d) = \log n.\]
    Deduce that
    \[\Lambda(n) = -\sum_{d\mid n}\mu(d)\log d.\]
\end{exercise}

\begin{exercise}
    Let \(f\) be multiplicative (i.e., \(f(nm)=f(n)f(m)\) for coprime \(n,m\)). 
    Suppose that
    \[n = \prod_{p^\alpha\mid\mid n}p^\alpha\]
    is the unique factorization of \(n\) into powers of distinct primes.
    Show that
    \[\sum_{d\mid n}f(d) = \prod_{p^{\alpha}\mid\mid n}\left(1 + f(p) + f(p^2) + \cdots + f(p^\alpha)\right).\]
    The notation \(p^\alpha\mid\mid n\) means that \(p^{\alpha}\) is the exact power of \(p\) dividing \(n.\)
\end{exercise}

\begin{exercise}
    Let \(f\) be multiplicative. 
    Show that
    \[D(f,s) = \prod_p\left(\sum_{\nu=0}^\infty f(p^\nu)p^{-\nu s}\right).\]
\end{exercise}

\begin{exercise}
    If 
    \[\zeta(s) = D(1,s) = \sum_{n=1}^\infty\frac{1}{n^s},\]
    show that 
    \[D(\mu,s)=\frac{1}{\zeta(s)}.\]
\end{exercise}

\begin{exercise}
    Show that
    \[D(\Lambda,s) = \sum_{n=1}^\infty\frac{\Lambda(n)}{n^s} = -\frac{\zeta'}{\zeta}(s),\]
    where \(-\zeta'(s) = \sum_{n=1}^\infty(\log n)n^{-s}\).
\end{exercise}

\end{document}