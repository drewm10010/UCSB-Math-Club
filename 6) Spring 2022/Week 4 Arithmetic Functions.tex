\documentclass{article}

\usepackage{../mathclub}

\title{Arithmetic Functions} 
\author{}
\date{April 18, 2022}

\begin{document}

\section{Introduction}

Welcome to week 4 of Math Club!
Today we will be talking about some functions that tend to appear when studying analytic number theory.
These are functions \(f(n)\) which express some number-theoretic fact about \(n\).
We provide some examples of such functions
\begin{enumerate}
    \item \(\nu(n) =\) the number of distinct prime divisors of \(n\).
    \item \(\Omega(n) =\) the number of prime divisors of \(n\) counted with multiplcity (e.g., \(\Omega(4)=2\)).
    \item The \(s\)-fold divisor function \(\sigma_s(n) = \sum_{d\mid n}d^s\) for \(s\in\CC\). If \(s=0\), write \(\sigma_s(n)=d(n)\), the number of divisors of \(n\).
    \item The M\"obius function 
    \[\mu(n) = \begin{cases}
        (-1)^{\nu(n)} & \textrm{if }n\textrm{ is squarefree}\\
        0 & \textrm{otherwise,}
    \end{cases}\]
    along with \(\mu(1)=1\).
    \item Euler's totient function, or \(\phi\)-function
    \[\phi(n) = n\prod_{p\mid n}\left(1-\frac{1}{p}\right).\]
    \item The von Mangoldt function
    \[\Lambda(n) = \begin{cases}
        \log p & \textrm{if }n=p^\alpha\textrm{ for some }\alpha\geq 1\textrm{ and }p\textrm{ prime}\\
        0 & \textrm{otherwise.}
    \end{cases}\]
\end{enumerate}

\section{Exercises}

For each of these exercises, you may use the results from the previous exercises.

\begin{exercise}
    Prove that
    \[\sum_{d\mid n}\mu(d)=\begin{cases}
        1 & \textrm{if }n=1\\
        0 & \textrm{otherwise}
    \end{cases}\]
\end{exercise}

\begin{exercise}[M\"obius Inversion]
    Show that
    \[f(n) = \sum_{d\mid n}g(d)\qquad \forall n\in\NN\]
    if and only if
    \[g(n) = \sum_{d\mid n}\mu(d)f(n/d)\qquad \forall n\in\NN.\]
\end{exercise}

\end{document}