\documentclass{article}

\usepackage{../mathclub}

\title{Generating Functions}
\author{}
\date{January 18, 2022}

\begin{document}

\section{Introduction}

In the words of Herbert Wilf, author of \textit{generatingfunctionology}, ``a generating function is a clothesline'' on which we hang up a sequence of numbers for display.
Think of a sequence of numbers \(a_0,a_1,a_2,\ldots\).
A \textbf{generating function} for this sequence is 
\[\sum_{n=0}^\infty a_n x^n = a_0 + a_1x + a_2x^2 + \cdots + a_k x^k + \cdots.\]
For example, if you thought about the Fibonacci numbers defined by \(F_0=1,\ F_1=1,\ F_{n+2}=F_{n+1}+F_n\), the generating function would be 
\[f(x) = 1 + x + 2x^2 + 3x^3 + 5x^4 + 8x^5 + 13x^6 + \cdots.\]
It's a fact that we can also write this power series as the expansion of the function
\[f(x) = \frac{1}{1-x-x^2}\]
about the origin.
We would call this expression the closed form of the series.
Local information about the coefficients in the generating function can give us global information about the closed form and vice versa, so converting between the expanded and closed forms is often helpful to our understanding of the function (or the combinatorial properties it models).

\section{Problems}

\begin{exercise}
Prove that 
\[\frac{1}{1-x-x^2} = \sum_{n=0}^\infty F_n x^n,\]
where \(F_n\) is the \(n\)th Fibonacci number. (Hint: 3)
\end{exercise}

\begin{exercise}
In how many ways can we roll three dice to get a sum to 9?
\end{exercise}

\begin{exercise}
Suppose four friends go to a carnival and each plays a game in which they can win 2 dollars, lose 1 dollar, or lose 2 dollars.
In how many ways can the friends collectively break even?
\end{exercise}

\begin{exercise}
In how many different ways can I collect a total of 20 dollars from four different children and three different adults if each child can contribute up to 6 dollars and each adult can contribute up to 10 dollars?
\end{exercise}

\begin{exercise}
Write the generating function for the number of ways we can form \(n\) cents using pennies, nickels, and dimes.
\end{exercise}

\begin{exercise}
Write the generating function for the number of partitions of \(n\) into even parts.
\end{exercise}

\begin{exercise}
Show that each positive integer has a unique base-2 representation (using generating functions, ideally).
\end{exercise}

\begin{exercise}
Using generating functions, prove that 
\[\binom{n}{0}^2 + \binom{n}{1}^2 + \cdots + \binom{n}{n}^2 = \binom{2n}{n}.\]
(Hint: 2)
\end{exercise}

\begin{exercise}
Prove that for a given positive integer $k$ each positive integer $n$ has a unique representation of the form $n=\binom{b_1}{1}+\binom{b_2}{2}+\cdots+\binom{b_k}{k}$ where $0\leqs b_1<\cdots< b_k$. 
\end{exercise}

\begin{exercise}
Alberto places \(N\) checkers in a circle. Some, perhaps all, are black; the others are white.
(The distribution of colors is random.) Betul places new checkers between pairs of adjacent checkers in
Alberto’s ring: she places a white checker between every two that are of the same color and a black checker
between every pair of opposite color. She then removes Alberto’s original checkers to leave a new ring of \(N\)
checkers in a circle. Alberto then performs the same operation on Betul’s ring of checkers following the same
rules. The two players alternately perform this maneuver over and over again. Show that if \(N\) is a power
of two, then all the checkers will eventually be white, no matter the arrangement of colors Alberto initially
puts down. Are there any interesting observations to be made if \(N\) is not a power of two?
\end{exercise}

\section{Attacking congruence problems with generating functions}

Suppose I have a set $S$ of non-negative integers, and I ask you to determine how many times you can write the non-negative integer $n$ as a sum of $m$ elements of $S$. Using generating functions, this is easy: set $f(x)=\sum_{k\in S} x^k$ and find the coefficient of $x^n$ in $f(x)^m$. But what if I ask you instead ask you how many times the sum of $k$ elements of $S$ is \emph{congruent} to $n$ modulo a prime $p$? It turns out that plugging in roots of unity is really helpful in retrieving arithmetic information like this. The next few exercises explore this theme. 

\begin{exercise}
\begin{enumerate}
    \item[(a)] Let \(\zeta=\cos\frac{2\pi}{n} + i\sin\frac{2\pi}{n}\) be an \(n\)th root of unity (satisfies \(\zeta^n=1\)).
    Show that the sum
    \[1 + \zeta^k + \zeta^{2k} + \cdots + \zeta^{(n-1)k}\]
    is \(n\) or 0, depending on whether \(k\) is a multiple of \(n\).
    \item[(b)] Compute
    \[\sum_{k=0}^{\lfloor n/3\rfloor}\binom{n}{3k}.\]
\end{enumerate}
\end{exercise}

\begin{exercise}
A finite sequence \(a_1,a_2,\ldots,a_n\) is called \textbf{\(p\)-balanced} if any sum of the form
\[a_k + a_{k+p} + a_{k+2p} + \cdots\]
is the same for any \(k=1,2,\ldots,p\).
Prove that if a sequence of 50 members is \(p\)-balanced for each of \(p=3,5,7,11,13,17\), then all its members are equal to 0. (Hint: 1)
\end{exercise}
The following lemma may come in handy (if you know some algebra, try to prove it!)
\begin{lemma}
If $p$ is a prime number and $a_0,a_1,\dots,a_{p-1}$ are rational numbers satisfying $a_0+a_1\zeta+a_2\zeta^2+\cdots+a_{p-1}\zeta^{p-1}=0$, where $\zeta=e^{2\pi i /p}$, then $a_0=a_1=\cdots=a_{p-1}$. 
\end{lemma}

\begin{exercise}[1987 IMO Shortlist]
Three persons $A$, $B$, $C$ play the following game: a subset with $k$ elements of the set $\{1,2,\dots,1986\}$ is selected randomly, all selections having the same probability. The winner is $A$, $B$, or $C$, according to whether the sum of the elements of the selected subset is congruent to $0$, $1$, or $2$ modulo $3$. Find all values of $k$ for which $A$, $B$, $C$ have equal chances of winning.
\end{exercise}
\begin{exercise}
Let $p>2$ be a prime. How many subsets of $\{1,2,\dots,p-1\}$ have the sum of their elements divisible by $p$?
\end{exercise}

\section{Analysis}
Operations involving power series can be carried out entirely formally, that is, in the ring of power series. However, the most profound insights regarding a sequence of numbers can often only be obtained by investigating the analytic properties of the generating function itself.

Perhaps the most profound takeaway from generating functions is that properties of sequences $\{a_n\}$ are intimately related to analytic properties of the various power series 
\[\sum_{n=0}^\infty a_nz^n,\ \ \ \ \sum_{n=0}^\infty a_n\frac{z^n}{n!},\ \ \ \ \sum_{n=1}^\infty \frac{a_n}{n^s},\ \ \ \ \sum_{n=1}^\infty a_nq^n\] we associate to them. 

For a cool example of this, check out the second chapter of Newman's \emph{Analytic Number Theory}.
\begin{exercise}
Can the positive integers be partitioned into at least two arithmetic progressions such that they all have different common differences?
\end{exercise}
\begin{exercise}
The numbers $0,2,5,6$ have the property that their positive differences are the numbers $1,2,3,4,5,6$ each taken on once. Can this phenomenon occur for some number above $6$?
\end{exercise}
\begin{exercise}
Each bus ticket has a six digit number. We call a ticket \emph{lucky} if the sum of the first three digits equals the sum of the last three digits. Prove that the number of lucky tickets is 
\[\frac{1}{\pi}\int_{-\frac{\pi}{2}}^{\frac{\pi}{2}}\left(\frac{\sin(10\theta)}{\sin\theta}\right)^6d\theta\]
\end{exercise}

\section{Aside on Dirichlet Series}
Let $f(n)$, $g(n)$ be functions from the natural numbers $\NN\to \CC$. Define the \emph{convolution} of $f$ and $g$ to be 
\[(f*g)(n) = \sum_{d\mid n} f(d)g\left(\frac{n}{d}\right)\] 

To $f$ and $g$, we may associate their corresponding \emph{Dirichlet series} 
\[\sum_{n=1}^\infty \frac{f(n)}{n^s},\ \ \ \ \sum_{n=1}^\infty \frac{g(n)}{n^s}\] where $s$ is the variable here. Verify by multiplying out these series that 
\[\sum_{n=1}^\infty \frac{(f*g)(n)}{n^s} = \left(\sum_{n=1}^\infty \frac{f(n)}{n^s}\right)\left(\sum_{n=1}^\infty \frac{g(n)}{n^s}\right)\]
\begin{exercise}
Define the \emph{Möbius function} $\mu:\NN\to \CC$ by 
\[\mu(n)=\begin{cases} 1 & n=1 \\ (-1)^k & \text{if $n=p_1\cdots p_k$ is square-free} \\ 0 & \text{otherwise}\end{cases}\]
Prove that its associated Dirichlet series $\sum_{n=1}^\infty \frac{\mu(n)}{n^s}$ is $1/\zeta(s)$, where $\zeta(s)$ is the Riemann zeta function. 
\end{exercise}
It is a consequence of this that an equivalent formulation of the Riemann Hypothesis involves merely proving a property of the function $\mu$.


\newpage

\section{Hints}

\begin{enumerate}
    \item Let \(f(x)=\sum_{n>0}a_nx^n\). If \(a_1,\ldots,a_n\) is \(p\)-balanced, then \(f(e^{2\pi i/p})=0\).
    Try to find a polynomial with too many roots so that the polynomial is identically 0.
    \item The strategy is to find a generating function for the LHS and one for the RHS and show that they turn out to be the same function.
    \item Show that \((1-x-x^2)\sum_{n\geq 0}F_nx^n=1\) by manipulating the LHS.
\end{enumerate}

\end{document}