\documentclass{article}

\usepackage{../mathclub}

\title{Algebraic Techniques}
\author{}
\date{January 3, 2022}

\begin{document}

\section{Introduction}

Welcome back to Week 1 of Math Club! Hope you guys had a great Winter Break. Today we will be taking a look at some common Algebra tricks. By Algebra, we mean the classic $y = mx + b$ kind from High School.
On the third page, we included a list of hints associated to select problems.
Try to solve the problems without the hints, but if you get stuck, take a look.
Without further ado, let's start doing some algebra! \\\\
There are three main tricks to consider when proving algebraic identities:
\begin{enumerate}[label=\Roman*.]
    \item Add \(0\) (adding and subtracting the same thing): \[a^4+4b^4 = a^4 + 4b^4 + 0 = a^4 + 4b^4 + 4b^2-4b^2 = (a^4+ 4b^2 + 4b^4) - 4b^2\]
    This adding and subtracting of $4b^2$ might seem useless, but it is actually very powerful, because both terms on the right hand side are squares, so it combines very well with the next trick.
    (Note that multiplying by 1 in a clever way is also often helpful.)
    \item Factoring and Expanding: 
    \[(a^4 + 4b^2 + 4b^4) - 4b^2 = (a^2+2b^2)^2 - (2b)^2 = (a^2 + 2b^2 + 2b)(a^2+2b^2-2b)\]
    First we used the square-of-sum expansion, $(x+y)^2 = x^2 + 2xy + y^2$, with $x=a^2$, $y=2b^2$, and then we used the difference-of-squares factorization, $x^2-y^2 = (x+y)(x-y)$, with $x=a^2+2b^2$ and $y=2b$. This concludes the proof of the \emph{Sophie-Germain identity}: $a^4+4b^4 = (a^2+2b^2 + 2b)(a^2+2b^2-2b)$.
    
    An essential part of this kind of algebraic manipulation is \emph{knowing how to create and recognize squares and cubes}. Examples are given.
    \begin{enumerate}[label=(\alph*)]
    \item $x^n - y^n = (x-y)(x^{n-1}+x^{n-2}y + \cdots + xy^{n-2}+y^{n-1})$ 
    \item If $n$ is odd, $x^n + y^n=(x+y)(x^{n-1}-x^{n-2}y+x^{n-3}y^2-\cdots + x^2y^{n-3}-xy^{n-2}+y^{n-1})$
    \item $x^{4} + 4y^{4} = (x^2 + 2y^2 + 2y)(x^2 + 2y^2-2y)$ 
    \item $x^3 + y^3 + z^3 - 3xyz = (x + y + z)(x^2 + y^2 + z^2 -xy - xz - yz)$\\$ = \frac{1}{2}(x+y+z)\left[ (x-y)^2 + (y-z)^2 + (x-z)^2\right]$ 
    \item $(x+y)^2 =x^2+2xy+y^2$
    \item $(x-y)^2 = x^2 -2xy+y^2$
    \item $(x+y)^3 = x^3 + 3x^2y +3xy^2 + y^3 = x^3 + y^3 +3xy(x+y)$
    \item $(x-y)^3 = x^3 - 3x^2y+3xy^2-y^3 = x^3 - y^3 -3xy(x-y)$
    \item (The Binomial Theorem) $(x+y)^n = \sum_{k=0}^n \binom{n}{k} x^{n-k}y^k$
\end{enumerate}
    \item Substitution and Simplification
    This final trick is often used to vastly simplify complicated-looking equations. The heuristic here is, \emph{always move in the direction of greater beauty or simplicity}. In other words, \emph{be lazy!} For example, supposed you're asked to find the product of the solutions to the equation \begin{equation}x^2+18x+30 = 2\sqrt{x^2 + 18x + 45}\end{equation} While you could certainly square both sides and solve the resulting quartic (ugh) equation, it would be much easier to let $y=\sqrt{x^2 + 18x + 45}$, and find all possible solutions for $y$. The equation would become
    \[y^2-15 = 2y\]
    Now it's easy: solve the quadratic equation for $y$ (by factoring or the quadratic formula), and once you have the value of $y$, you can plug back into $y=\sqrt{x^2 + 18x + 45}$ to find the solutions for $x$. \emph{We essentially just reduced the problem of solving a quartic equation into that of solving two quadratic equations}.
\end{enumerate}
% \section{Resources}

% Here are some resources.
% \begin{enumerate}
%     some resources
% \end{enumerate}

\section{Problems} 
\begin{exercise}[Hint: 2]
If $xy=x+y = 3$, find $x^3+y^3$.
\end{exercise}

\begin{exercise}[Hint: 3]
    Show that for no positive integer $n$ can both $n+3$ and $n^2 +3n+3$ be perfect cubes.
\end{exercise}

\begin{exercise}[Hint: 1]
Verify that
\[\sqrt[3]{20+14\sqrt{2}}+\sqrt[3]{20-14\sqrt{2}}=4\]
\end{exercise}

\begin{exercise}[Hint: 6]
Given $x^2 + y^2 + z^2 = 1$, find the minimum value of $xy+xz+yz$. No calculus!
\end{exercise}

\begin{exercise}[Simon's favorite factoring trick, Hint: 4]
How many integer solutions \((a,b)\) does \(ab-3b-2a=7\) have?
\end{exercise}

\begin{exercise}[2018 A1, Hint: 11]
Find all ordered pairs \((a,b)\) of positive integers for
which
\begin{align*} 
    \frac{1}{a} + \frac{1}{b} = \frac{3}{2018}
\end{align*}
\end{exercise}

\begin{exercise}
If the expression 
\[(x^3-x^2y+xy^2+y^3)^5\] 
is expanded and simplified, what is the sum of all the coefficients of the resulting polynomial?
\end{exercise}

\begin{exercise}[Hint: 10]
Prove that for any non-negative number $n$, the number 
\begin{align*}
    5^{5^{n+1}} + 5^{5^n} + 1
\end{align*}
is not a prime.
\end{exercise}

\begin{exercise}[Hint: 7]
Let \(x,y,z\) be distinct real numbers.
Prove that
\[\sqrt[3]{x-y} + \sqrt[3]{y-z} + \sqrt[3]{z-x} \neq 0.\]
\end{exercise}

\begin{exercise}[2019 A1, Hint: 8]
Determine all possible values of the expression
\[A^3 + B^3 + C^3 - 3ABC,\]
where \(A\), \(B\), and \(C\) are nonnegative integers.
\end{exercise}

\begin{exercise}[Hint: 9]
Factor $5^{1985}-1$ into three integers, each one larger than $5^{100}$.
\end{exercise}

\begin{exercise}[Hint: 5]
Prove that the number
\[\frac{5^{125}-1}{5^{25}-1}\]
is not prime.
\end{exercise}

\begin{exercise}[1977 A2]
Determine all solutions in real numbers $x,y,z,w$ of the system
\begin{align*}
    x+y+z&= w \\
    \frac{1}{x}+\frac{
    1}{y}+\frac{1}{z} &= \frac{1}{w}
\end{align*}
\end{exercise}

\newpage 
\section{Hints}

\begin{enumerate}
    \item Cubing stuff
    \item Cubing stuff
    \item Cubing stuff
    \item Factorize
    \item Similar to form to previous exercise. Consider $(25x^4+15x^2+1)^2-(5x)^2(5x^2+1)^2$.
    \item Squaring stuff
    \item Find a factorization of \(a^3+b^3+c^3-3abc\) involving the term \((a+b+c)\)
    \item Use the factorization from hint 7.
    \item $625x^8 + 125x^6 + 25x^4 + 5x^2 + 1 = (25x^4-25x^3+15x^2-5x+1)(25x^4+25x^3+15x^2+5x+1)$
    \item A cool substitution can do the trick.
    \item Previous exercise!
\end{enumerate}

\end{document}